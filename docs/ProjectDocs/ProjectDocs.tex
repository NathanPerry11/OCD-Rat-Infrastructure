\documentclass[12pt]{article}
\usepackage[utf8]{inputenc}
\usepackage{hyperref}
\usepackage{enumitem}

\title{12 Behavioral Data Analysis Platform: Building Research Infrastructure for Animal Models of OCD (SE/CS)}
\author{}
\date{}

\begin{document}

\maketitle

\textbf{Supervisor(s):} Dr. Henry Szechtman and Dr. Anna Dvorkin-Gheva\\
\textbf{Organization:} Dept. of Psychiatry and Behavioural Neurosciences, and Dept. of Pathology and Molecular Medicine, McMaster\\
\textbf{E-mail(s):} \href{mailto:szechtma@mcmaster.ca}{szechtma@mcmaster.ca}, \href{mailto:dvorkin@mcmaster.ca}{dvorkin@mcmaster.ca}\\
\textbf{Selected:} Available

\section*{Description}

\subsection*{Project Overview}
This project focuses on creating a comprehensive data analysis platform for one of the most extensive behavioral neuroscience datasets—nearly 20,000 trials of rat behavioral data from studies of Obsessive-Compulsive Disorder (OCD). The dataset contains rich spatial-temporal tracking data (x, y, t coordinates), corresponding video recordings of rat behavior, and additional research files that researchers may use to study psychiatric disorders.

\subsection*{The Challenge}
While this valuable dataset exists in a public repository, there is currently no user-friendly way for researchers to search, filter, download, or analyze specific subsets of data. Moreover, researchers cannot easily view behavioral trajectories alongside their corresponding videos, a feature that would add to the dataset’s scientific utility.

\subsection*{Project Goals}
Your team will build a modern web-based platform that makes this behavioral dataset truly accessible to the global research community:

\begin{enumerate}[label=\arabic*.]
    \item \textbf{Database Design \& Backend Development}
    \begin{itemize}
        \item Design a robust PostgreSQL database schema for behavioral data, metadata, video files, and additional research files
        \item Implement REST API endpoints for data queries and downloads
        \item Create efficient indexing for large-scale spatial-temporal data and file management system for video content
    \end{itemize}

    \item \textbf{Interactive Web Interface}
    \begin{itemize}
        \item Build a React-based frontend with intuitive search and filtering capabilities
        \item Implement natural language querying—allow researchers to ask questions like “show me trials with strong checking behavior after 5 injections” or “find sessions where rats showed compulsive patterns”
        \item Create data visualization tools for trajectory plotting and behavioral metrics
        \item Implement synchronized video playback—allow researchers to view behavioral trajectories alongside corresponding video recordings
        \item Design user-friendly interfaces for researchers without programming expertise
    \end{itemize}

    \item \textbf{Data Processing Pipeline}
    \begin{itemize}
        \item Develop Python tools for processing coordinate data into meaningful behavioral measures
        \item Implement algorithms for detecting key behavioral patterns (home-base behavior, checking routes, exploration metrics)
        \item Create automated analysis workflows for common research tasks
        \item Design extensible architecture for future video analysis capabilities
    \end{itemize}
\end{enumerate}

\subsection*{Technical Stack}
\begin{itemize}
    \item \textbf{Backend:} Python (FastAPI), PostgreSQL, Redis caching
    \item \textbf{Frontend:} React, modern JavaScript, data visualization libraries
    \item \textbf{Data Processing:} Python (NumPy, Pandas, scientific computing libraries)
    \item \textbf{NLP Integration:} Language model APIs for natural language querying
    \item \textbf{Media Handling:} Video streaming, file management systems
\end{itemize}

\subsection*{Why This Project Matters}
\begin{itemize}
    \item \textbf{Real Impact:} Your work will be used by OCD researchers worldwide, potentially accelerating discoveries that help millions of people
    \item \textbf{Open Science:} You’ll contribute to making scientific data more accessible and reusable
    \item \textbf{Technical Challenge:} Handle big data problems with spatial-temporal complexity, large video files, and AI-assisted querying
    \item \textbf{Professional Experience:} Build a production system that serves an international research community
\end{itemize}

\subsection*{Learning Outcomes}
\begin{itemize}
    \item Full-stack web development with modern technologies
    \item Database design for scientific data and multimedia content management
    \item API development and optimization for large file handling
    \item Data visualization, video integration, and user experience design
    \item Working with real scientific datasets and research workflows
\end{itemize}

\subsection*{Deliverables}
\begin{itemize}
    \item Functional web platform deployed and accessible online
    \item Natural language query system for intuitive data exploration
    \item Synchronized trajectory-video viewing capabilities
    \item Python analysis tools packaged as open-source libraries
    \item Extensible file management system for diverse research data
    \item User documentation and tutorials for researchers
\end{itemize}

\subsection*{Team Structure (Suggested)}
\begin{itemize}
    \item Backend/Database specialist
    \item Frontend/UI developer
    \item Data processing/algorithms developer
    \item NLP/AI integration specialist
    \item Media/video systems developer
    \item DevOps/deployment specialist
\end{itemize}

\subsection*{Links}
\begin{itemize}
    \item \href{https://academic.oup.com/gigascience/article/doi/10.1093/gigascience/giac092/6756450}{GigaScience Article}
    \item \href{https://www.frdr-dfdr.ca/repo/collection/szechtmanlab}{FRDR Szechtman Lab Repository}
\end{itemize}

\end{document}
