\begin{enumerate}
  \item What went well while writing this deliverable? 
  \item What pain points did you experience during this deliverable, and how did
  you resolve them?
  \item How many of your requirements were inspired by speaking to your
  client(s) or their proxies (e.g. your peers, stakeholders, potential users)?
  \item Which of the courses you have taken, or are currently taking, will help
  your team to be successful with your capstone project.
  \item What knowledge and skills will the team collectively need to acquire to
  successfully complete this capstone project?  Examples of possible knowledge
  to acquire include domain specific knowledge from the domain of your
  application, or software engineering knowledge, mechatronics knowledge or
  computer science knowledge.  Skills may be related to technology, or writing,
  or presentation, or team management, etc.  You should look to identify at
  least one item for each team member.
  \item For each of the knowledge areas and skills identified in the previous
  question, what are at least two approaches to acquiring the knowledge or
  mastering the skill?  Of the identified approaches, which will each team
  member pursue, and why did they make this choice?

\end{enumerate}

  \section{Group Reflection:}

  \begin{enumerate}

   \item What knowledge and skills will the team collectively need to acquire to
  successfully complete this capstone project?  Examples of possible knowledge
  to acquire include domain specific knowledge from the domain of your
  application, or software engineering knowledge, mechatronics knowledge or
  computer science knowledge.  Skills may be related to technology, or writing,
  or presentation, or team management, etc.  You should look to identify at
  least one item for each team member.

    \begin{itemize}
      \item \textbf{REST APIs: } Familiarize ourselves with REST APIs as knowing how they work will be essential to properly understand how to extract data from the 
    FRDR in a way that is well structured for ease of use in our system and for users.

      \item \textbf{LLM Training: } Read up on how to train a basic LLM model for a specific purpose. We should learn how to prepare
    datasets for the LLM to learn off of so that it can accurately categorize rat behaviour as well as interpret natural language into queries.

    \item \textbf{Aggregate and Manipulate Metadata: } Given that the metadata is stored in one large file as is not yet directly associated with the data objects
    we need to identify a reliable and efficient way to read the metadata from the seperate file and relate it to the data objects. This could be done
    by developing some sort of algorithm to match this metadata with a data object but research must also be down on what this metadata will look like
    when it is queried raw.

    \end{itemize}


  \item For each of the knowledge areas and skills identified in the previous
  question, what are at least two approaches to acquiring the knowledge or
  mastering the skill?  Of the identified approaches, which will each team
  member pursue, and why did they make this choice?

  \begin{itemize}
      \item \textbf{Leo Vugert - REST APIs: }The first way to acquire this knowledge would be to start trying to use the FRDR API to get a feel for
      how it works. Learning about REST APIs in this way is the most applicable to our scenario and thus likely the most useful. A second option
      would be research online for a specific breakdown of this topic, likely through a service such as W3Schools or tutorialspoint. An example can
      be found linked here: https://www.tutorialspoint.com/restful/index.htm. Leo took this as he is the lest familiar with the topic and thus
      can gain the most from reviewing it.

      \item \textbf{Timothy Pokanai, Nathan Perry, Jeremy Orr - LLM Training: }Given that this is an area of technology that our group has very
      little experience with, we decided that it would be good for multiple team members to take on this knowledge area. Much like the previous knowledge
      area, the internet is a great resource for learning these things. One knowledge source that could be used is linked here: https://www.datacamp.com/tutorial/fine-tuning-large-language-models.
      Additionally, another good knowledge source can be found in the form of a rat pose identification library that is publicly available. This
      can be studied or referenced and would be a great thing for us to use to hone this skill for our purposes. Linked here: https://github.com/DeepLabCut/DeepLabCut.
      The specific team members were chosen because of their lack of experience in this field, Aidan has the most knowledge of LLMs so we thought it best
      to give him a different knowledge area.

    \item \textbf{Aidan Goodyer - Aggregate and Manipulate Metadata} The obvious place to start for gaining knowledge in this area is by querying the FRDR metadata
    using the provided API to investigate how the metadata is given to us, how difficult it will be to access and possible ways to manipulate or 
    make use of it. From there, it would good to review relevant algorithmic approaches to sort through this metadata and relate it to the specific
    data objects so that our database schema is coherent. The internet is a good tool for this as well as potentially reviewing content from our previous
    data structures and algorithms course. Aidan was assigned this area as he is the most proficient in the other two areas and thus has the most
    room to grow.

\end{itemize}

\end{enumerate}

\section{Nathan Perry Reflection:}

\begin{enumerate}
  \item What went well while writing this deliverable? 

  \par{ Understanding the expectations for this deliverable went well, especially relative to the previous deliverable. In the goals and problem statement
  document, I felt that we had a general idea of what we needed to do but not a concrete guide. Using the Volere template, we were able to find very helpful
  documentation which provided an overview of specifically what was expected to be put in each section. This made writing the document feel much more structured
  and provided confidence we were on the right track, especially since the document itself gives very little to go off of aside from section titles.}

  \item What pain points did you experience during this deliverable, and how did
  you resolve them?

  \par{ Far and away the biggest and nearly only pain point for this deliverable was the delegation of work among the members. A few of us like the frontload
  the effort over the deadline window while others like to backload the effort and thus, a few members ended up starting early and needing to take on more
  than their fair share of tasks as an inevitable scramble to complete the deliverable ensued. To try to resolve this, the members who did not start earlier offered
  to take on the rest of the work, telling the others they don't need to continue to contribute. This was somewhat helpful but the deliverable needs to get done
  so everyone ended up helping to some extent anyways. In the future, we decided to resolve this by delegating tasks before work begins. For the SRS
  we used a 'take-as-you-go' approach to issues whereas from now on we will be assigning everything ahead of time. }


  \item How many of your requirements were inspired by speaking to your
  client(s) or their proxies (e.g. your peers, stakeholders, potential users)?

\par{I cannot give a concrete number of requirements that were inspired by speaking to our client but we did have a meeting with Dr. Henry Szechtman
in the week leading up to the writing of our SRS document. The biggest requirement that he directly inspired was the idea to pre-package queries
and display them like a webstore so that users who may not even know exactly what they are looking for can derive value from our system. Furthermore,
our filtering requirements were guided by a reference website created by the National Cancer Institute which
was provided by our client as a reference. \href{https://portal.gdc.cancer.gov/analysis_page?app=CohortBuilder&tab=general.}. Requirements
to make the project maintainable by people other than the original developers and to be open source was also inspired by our client meeting as Dr. Szechtman wants the
project to carry on after we graduate.
}

  \item Which of the courses you have taken, or are currently taking, will help
  your team to be successful with your capstone project.

  \par{ Our project is very multi disciplinary as it is a full stack project and thus plenty of what we learned will go into
  our development. First and foremost, our databases class will provide the knowledge needed to design a database schema as a foundation
  for our entire system. Additionally, the Human-Computer interfaces class we are currently taking will be helpful in creating a user interface
  that is approachable for a non-technical audience. Furthermore, all of the very coding heavy courses including object oriented programming, system
  design and software development will help us to implement the core business logic of the application and to connect the different components.
  Finally, I will add that the our data structures class will help us to efficiently store, package and represent the data objects being transmitted
  in our system. }
\end{enumerate}

\section{Timothy Pokanai Reflection}

\begin{enumerate}
  \item What went well while writing the deliverable?
  \par{I would say that we were able to smoothly and consistently deliver our preliminary SRS document. What I mean is as we worked throughout the deliverable 
  our individual ideas and interpretations of our project description, through the forms of requirements, were aligned and consistent. To be fair we may have had 
  a commit or two where something I wrote was modified to better suit the context of our project, but that was the only occurence in terms of peer performed changes. 
  I would attribute our team's ability to work individually with seamless integration to the fact that we all acquired a great grasp of the scope of the project and 
  the constraints associated with it early on. A lot of this can be attributed to our supervisors with their involvement and a bit from the public nature of the project.}

  \item What pain points did you experience during this deliverable, and how did
  you resolve them?
  \par{What caused some turbulence while working on this deliverable was not related to the deliverable at all, it was how we delegated work throughout our team. 
  We used a first-come-first-serve approach when it came to picking up issues related to our deliverable, which had both its pros and cons. It allowed some 
  members of our team to pick what they wanted to do earlier on and often times they would do it early. This allowed some members to complete their 
  work earlier on in the project timeline which is definitely a pro. On the flip side, members that start their work later on don't have as much flexibility in 
  their choice of work issues. This resulted in an imbalance of workload between group members where some would be left with multiple issues with more weight in 
  work when compared to previously completed issues, and at times people who did their share of work needed to complete extra for the sake of time. Our resolution plan 
  moving forward is to delegate tasks in a different manner, which will be dividing sections so everyone has equal weight in work before anyone starts working 
  on a deliverable.
  }

  \item How many of your requirements were inspired by speaking to your
  client(s) or their proxies (e.g. your peers, stakeholders, potential users)?
  \par{Its hard to keep track of how many requirements exactly were inspired by our conversations with our clients and their proxies, but I'll provide the 
  general requirement section names which were mostly or fully completely based on our discussions. Our mandated constraints were heavily based on context 
  provided to us about the current and future system implementation. The functional requirements we created were directly inspired by what features our clients wanted 
  to see us deliver with the project. The requirement sections related to the usability and user experience of the software, which were look and feel 
  and usability and humanity requirements, were inspired by our clients goals of creating easy-to-use and globally accessible software. Lastly, since our clients are 
  actively involved with the current system in place, we were able to clearly interpret their requirements in the operational and environmental domain, 
  as well as security and compliance requirements.}

  \item Which of the courses you have taken, or are currently taking, will help
  your team to be successful with your capstone project.
  \par{Some elements from the project-based courses, most notably 1P13 and 2PX3, will definitely help with the pacing and iterative nature in terms of development and 
  documentation of our project. Also from those project based courses, we can use principles we have learned about working in a group and doing presentations which will 
  be more prevalent further in the project timeline. On the topic of workflows and development cycles, I think we will need to adopt one or two development cycles like 
  Agile Methodology coupled with Test Driven Development for example. We have each had much practice with some kind of development cycle from our Software Design I and III 
  courses where we focused on building larger-scale software during a time period, and we will definitely need to apply what we have learned from those for our documentation 
  and development. Furthermore, one of our main focus points for this project is to create a user friendly and accessible interface for users of our software. We are all 
  currently in a Human Computer Interfaces class where we will need to apply what we learn about user elicitation, design, and experience. Lastly, our project will heavily rely 
  on data that will be visualized and possible pre-processed by users. This will be a perfect opportunity to apply our learnings from our Database Design course and our Concurrent 
  System Design course, both of which go hand-in-hand when designing efficient, scalable, and safe databases.}
\end{enumerate}

\section{Jeremy Orr Reflection:}

\begin{enumerate}
 \item What went well while writing this deliverable?

    \par{I think our team worked effectively to complete a large and detailed document in an organized manner. We communicated clearly and consistently throughout the process, which helped us stay aligned with our goals. Personally, I was able to complete my assigned sections early and efficiently. As a group, we also did a good job discussing challenges openly and collaborating to resolve them.}

    \par{One specific event that went particularly well was our collaborative editing session before submission. During that session, we reviewed the entire document together, finalized remaining sections, and made meaningful improvements. This not only strengthened the overall quality and consistency of the deliverable but also ensured that everyone’s contributions were well integrated. Overall, our teamwork and communication during that session demonstrated how effectively we could work together under a deadline.}

  \item What pain points did you experience during this deliverable, and how did you resolve them?

    \par{One of the main challenges we faced was dividing the work fairly among team members. Initially, we used a first-come, first-served approach, which led to an uneven distribution of tasks. Additionally, I accidentally started working on a section that a team member had already begun, which caused some inefficiency and duplicated effort.}

    \par{To resolve these issues, we discussed a more structured approach for future work, agreeing to evenly divide tasks before starting on a deliverable. For my specific issue, I committed to reviewing team assignments and communicating with teammates before beginning work on any section, ensuring that efforts are coordinated and that overlap is avoided. This experience highlighted the importance of proactive communication and clear task assignment in collaborative projects, which we will carry forward into future deliverables.}

  \item How many of your requirements were inspired by speaking to your client(s) or their proxies (e.g., your peers, stakeholders, potential users)?

    \par{The primary requirement that was influenced by discussions with our client was clarifying the specific scope and objectives of the project. We held multiple meetings with the clients, during which they shared numerous ideas for future extensions and potential features. These conversations were important to ensure that our team focused only on the deliverables required for the current project and did not extend into future scope items. Other than clarifying the project scope, there were few additional requirements that needed direct input from the client, as much of the remaining work was based on established technical objectives and user needs identified through our research.}

  \item Which of the courses you have taken, or are currently taking, will help
  your team to be successful with your capstone project.

    \par{I believe that several courses will help our team succeed in this capstone project. These include SFWRENG 2AA4: Introduction to Software Development, SFWRENG 3RA3: Software Requirements, SFWRENG 3A04: Software Testing, and the Software Architecture course. These courses have provided us with great knowledge in software development practices, requirement gathering, system design.}


\end{enumerate}

\section{Leo Vugert Reflection:}

\begin{enumerate}
 \item What went well while writing this deliverable?

    \par{Our team’s ability to communicate openly and effectively helped throughout this deliverable. When we were at pain points from dividing the work, not understanding sections, or any help that anyone needed, one or multiple other group members would always be there to help or try to resolve the issue. An example of this for me would be when I wasn’t sure exactly what to put for the functional requirements and how to word them, the group came together with me to brain-storm and we got it done efficiently.}
  
  \item What pain points did you experience during this deliverable, and how did you resolve them?

    \par{A pain point we experienced during this deliverable was when writing each section, we weren’t completely sure what to put for the content. Some either had confusing titles or no guidance of what form the information should be in (table, chart, diagram, etc.) We resolved this by using the a Volere documentation guide that we found online. It gave us the Content, Motivation, Consideration, and Form of each section, this helped us to ensure that our work for each section was aligned with what was expected of us.}

  \item How many of your requirements were inspired by speaking to your client(s) or their proxies (e.g., your peers, stakeholders, potential users)?

    \par{All of the functional requirements were made from our meetings with the client. During the meetings, we gathered all the necessary information for the project about the client’s expectations of the product and us. The non-functional requirements were more of a blend of the professor’s requirements and requirements we put in place with our technical knowledge. It’s difficult to say exactly how many of these are from the client, but they primarily influenced many of the look and feel, and usability requirements.}

  \item Which of the courses you have taken, or are currently taking, will help
  your team to be successful with your capstone project.

    \par{Our project uses a good blend of all the courses we’ve taken in our undergrad. Two big ones that I believe would help the technical aspect would be SFWRENG 3DB3 Databases would help us to work with the large SQL database that the professor has set up, and SFWRENG 4HC3 Human Computer Interfaces will help us make the front-end of our assignment. For the planning, design, and administrative details of project, classes like Software Design 1 to 3, Software Requirements and Security Considerations, and Engineering Design 1 to 3 will help us in the project management and execution process.}

\end{enumerate}

\section{Aidan Goodyer Reflection:}

\begin{enumerate}
  \item What went well while writing this deliverable?
  \par{Overall, the team did a great job about communicating expectations, dividing work, and providing effective peer review and support throughout the requirements analysis process. The group was particularly effective in resolving confusion and coming to consensus on topics and SRS sections that were initially unclear. The structured Volere template and its corresponding section hints also helped us deliver the SRS in a timely manner.   }
  \item What pain points did you experience during this deliverable, and how did you resolve them?

  \par{One challenge we faced particularly with the SRS is resolving ambiguity of requirements. Some aspects of system goals and requirements were hard to resolve from talking to the client. One important strategy we used to mitigate this problem was to note areas of uncertainty in each client meeting, and then the group would take time to formulate questions after and prepare them for our next client consultation. This strategy was effective in helping to resolve some of the more fine grained details of how the system is expected to function. Another common pain point shared by the group is work delegation. Initially, a first-come-first-serve approach was attempted on the SRS, however led to imbalance in work distribution due to the varied schedules among the team. To create a more balanced and predictable delivery schedule, the group agreed to pre-assign work for future deliverables to ensure these challenges are not repeated.    }

  \item How many of your requirements were inspired by speaking to \\ your client(s) or their proxies (e.g., your peers, stakeholders, potential users)?
  
  \par{Our client was instrumental for helping us resolve requirements for the system. The team is building a system for a domain none of us have prior experience with (Neuroscience and Animal Behaviour trials) and thus our clients (Dr. Henry Szechtman and Dr. Anna Dvorkin-Gheva) have been very helpful in bridging the domain gap and expressing their goals for the project. For example, Dr. Szechtman helped shape important look-and-feel requirements regarding the system's web-store like interface, as well as Dr. Dvorkin-Gheva helping to define some of the expected goals for the system's querying interface. While difficult to give an exact number, I would estimate that $>50\%$ of requirements were elicitated via client consultations, with the rest being shaped by our software-specific knowledge.  }

  \item Which of the courses you have taken, or are currently taking, will help your team to be successful with your capstone project.
  \par{Three courses that are proving instrumental to the project are: \\ SFWRENG 3DB3 (Databases), SFWRENG 3AA4 (Large Scale System Design), and SFWRENG 4HC3 (Human Computer Interfaces). Due to the large database and querying component, the background provided by the databases course serves as an important basis of knowledge on how to effectively utilize a database for the animal behaviour dataset. Additionally, the large scale systems design course is relevant in its focus on design patterns for large, real world systems comprised of multiple modules. This course identifies the benefits of specific patterns, like the Client-Server model, that are likely to be relevant in our design. Lastly, human-computer interfaces is valuable in shaping many of our non-functional reqquirements in a way to maximize the value we provide to our end users. It discusses UX design principles and good practices that are highly relevant to our web interface component.  }
\end{enumerate}