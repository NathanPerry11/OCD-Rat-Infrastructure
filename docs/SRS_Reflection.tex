\begin{enumerate}
  \item What went well while writing this deliverable? 
  \item What pain points did you experience during this deliverable, and how did
  you resolve them?
  \item How many of your requirements were inspired by speaking to your
  client(s) or their proxies (e.g. your peers, stakeholders, potential users)?
  \item Which of the courses you have taken, or are currently taking, will help
  your team to be successful with your capstone project.
  \item What knowledge and skills will the team collectively need to acquire to
  successfully complete this capstone project?  Examples of possible knowledge
  to acquire include domain specific knowledge from the domain of your
  application, or software engineering knowledge, mechatronics knowledge or
  computer science knowledge.  Skills may be related to technology, or writing,
  or presentation, or team management, etc.  You should look to identify at
  least one item for each team member.
  \item For each of the knowledge areas and skills identified in the previous
  question, what are at least two approaches to acquiring the knowledge or
  mastering the skill?  Of the identified approaches, which will each team
  member pursue, and why did they make this choice?
\end{enumerate}


\section{Nathan Perry Reflection:}

\begin{enumerate}
  \item What went well while writing this deliverable? 

  \par{ Understanding the expectations for this deliverable went well, especially relative to the previous deliverable. In the goals and problem statement
  document, I felt that we had a general idea of what we needed to do but not a concrete guide. Using the Volere template, we were able to find very helpful
  documentation which provided an overview of specifically what was expected to be put in each section. This made writing the document feel much more structured
  and provided confidence we were on the right track, especially since the document itself gives very little to go off of aside from section titles.}

  \item What pain points did you experience during this deliverable, and how did
  you resolve them?

  \par{ Far and away the biggest and nearly only pain point for this deliverable was the delegation of work among the members. A few of us like the frontload
  the effort over the deadline window while others like to backload the effort and thus, a few members ended up starting early and needing to take on more
  than their fair share of tasks as an inevitable scramble to complete the deliverable ensued. To try to resolve this, the members who did not start earlier offered
  to take on the rest of the work, telling the others they don't need to continue to contribute. This was somewhat helpful but the deliverable needs to get done
  so everyone ended up helping to some extent anyways. In the future, we decided to resolve this by delegating tasks before work begins. For the SRS
  we used a 'take-as-you-go' approach to issues whereas from now on we will be assigning everything ahead of time. }


  \item How many of your requirements were inspired by speaking to your
  client(s) or their proxies (e.g. your peers, stakeholders, potential users)?

\par{I cannot give a concrete number of requirements that were inspired by speaking to our client but we did have a meeting with Dr. Henry Szechtman
in the week leading up to the writing of our SRS document. The biggest requirement that he directly inspired was the idea to pre-package queries
and display them like a webstore so that users who may not even know exactly what they are looking for can derive value from our system. Furthermore,
our filtering requirements were guided by a reference website created by the National Cancer Institute which
was provided by our client as a reference. \href{https://portal.gdc.cancer.gov/analysis_page?app=CohortBuilder&tab=general.}. Requirements
to make the project maintainable by people other than the original developers and to be open source was also inspired by our client meeting as Dr. Szechtman wants the
project to carry on after we graduate.
}

  \item Which of the courses you have taken, or are currently taking, will help
  your team to be successful with your capstone project.

  \par{ Our project is very multi disciplinary as it is a full stack project and thus plenty of what we learned will go into
  our development. First and foremost, our databases class will provide the knowledge needed to design a database schema as a foundation
  for our entire system. Additionally, the Human-Computer interfaces class we are currently taking will be helpful in creating a user interface
  that is approachable for a non-technical audience. Furthermore, all of the very coding heavy courses including object oriented programming, system
  design and software development will help us to implement the core business logic of the application and to connect the different components.
  Finally, I will add that the our data structures class will help us to efficiently store, package and represent the data objects being transmitted
  in our system. }
\end{enumerate}

