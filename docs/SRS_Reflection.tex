\begin{enumerate}
  \item What went well while writing this deliverable? 
  \item What pain points did you experience during this deliverable, and how did
  you resolve them?
  \item How many of your requirements were inspired by speaking to your
  client(s) or their proxies (e.g. your peers, stakeholders, potential users)?
  \item Which of the courses you have taken, or are currently taking, will help
  your team to be successful with your capstone project.
  \item What knowledge and skills will the team collectively need to acquire to
  successfully complete this capstone project?  Examples of possible knowledge
  to acquire include domain specific knowledge from the domain of your
  application, or software engineering knowledge, mechatronics knowledge or
  computer science knowledge.  Skills may be related to technology, or writing,
  or presentation, or team management, etc.  You should look to identify at
  least one item for each team member.
  \item For each of the knowledge areas and skills identified in the previous
  question, what are at least two approaches to acquiring the knowledge or
  mastering the skill?  Of the identified approaches, which will each team
  member pursue, and why did they make this choice?
\end{enumerate}


\section{Nathan Perry Reflection:}

\begin{enumerate}
  \item What went well while writing this deliverable? 

  \par{ Understanding the expectations for this deliverable went well, especially relative to the previous deliverable. In the goals and problem statement
  document, I felt that we had a general idea of what we needed to do but not a concrete guide. Using the Volere template, we were able to find very helpful
  documentation which provided an overview of specifically what was expected to be put in each section. This made writing the document feel much more structured
  and provided confidence we were on the right track, especially since the document itself gives very little to go off of aside from section titles.}

  \item What pain points did you experience during this deliverable, and how did
  you resolve them?

  \par{ Far and away the biggest and nearly only pain point for this deliverable was the delegation of work among the members. A few of us like the frontload
  the effort over the deadline window while others like to backload the effort and thus, a few members ended up starting early and needing to take on more
  than their fair share of tasks as an inevitable scramble to complete the deliverable ensued. To try to resolve this, the members who did not start earlier offered
  to take on the rest of the work, telling the others they don't need to continue to contribute. This was somewhat helpful but the deliverable needs to get done
  so everyone ended up helping to some extent anyways. In the future, we decided to resolve this by delegating tasks before work begins. For the SRS
  we used a 'take-as-you-go' approach to issues whereas from now on we will be assigning everything ahead of time. }


  \item How many of your requirements were inspired by speaking to your
  client(s) or their proxies (e.g. your peers, stakeholders, potential users)?

\par{I cannot give a concrete number of requirements that were inspired by speaking to our client but we did have a meeting with Dr. Henry Szechtman
in the week leading up to the writing of our SRS document. The biggest requirement that he directly inspired was the idea to pre-package queries
and display them like a webstore so that users who may not even know exactly what they are looking for can derive value from our system. Furthermore,
our filtering requirements were guided by a reference website created by the National Cancer Institute which
was provided by our client as a reference. \href{https://portal.gdc.cancer.gov/analysis_page?app=CohortBuilder&tab=general.}. Requirements
to make the project maintainable by people other than the original developers and to be open source was also inspired by our client meeting as Dr. Szechtman wants the
project to carry on after we graduate.
}

  \item Which of the courses you have taken, or are currently taking, will help
  your team to be successful with your capstone project.

  \par{ Our project is very multi disciplinary as it is a full stack project and thus plenty of what we learned will go into
  our development. First and foremost, our databases class will provide the knowledge needed to design a database schema as a foundation
  for our entire system. Additionally, the Human-Computer interfaces class we are currently taking will be helpful in creating a user interface
  that is approachable for a non-technical audience. Furthermore, all of the very coding heavy courses including object oriented programming, system
  design and software development will help us to implement the core business logic of the application and to connect the different components.
  Finally, I will add that the our data structures class will help us to efficiently store, package and represent the data objects being transmitted
  in our system. }
\end{enumerate}

\section{Timothy Pokanai Reflection}

\begin{enumerate}
  \item What went well while writing the deliverable?
  \par{I would say that we were able to smoothly and consistently deliver our preliminary SRS document. What I mean is as we worked throughout the deliverable 
  our individual ideas and interpretations of our project description, through the forms of requirements, were aligned and consistent. To be fair we may have had 
  a commit or two where something I wrote was modified to better suit the context of our project, but that was the only occurence in terms of peer performed changes. 
  I would attribute our team's ability to work individually with seamless integration to the fact that we all acquired a great grasp of the scope of the project and 
  the constraints associated with it early on. A lot of this can be attributed to our supervisors with their involvement and a bit from the public nature of the project.}

  \item What pain points did you experience during this deliverable, and how did
  you resolve them?
  \par{What caused some turbulence while working on this deliverable was not related to the deliverable at all, it was how we delegated work throughout our team. 
  We used a first-come-first-serve approach when it came to picking up issues related to our deliverable, which had both its pros and cons. It allowed some 
  members of our team to pick what they wanted to do earlier on and often times they would do it early. This allowed some members to complete their 
  work earlier on in the project timeline which is definitely a pro. On the flip side, members that start their work later on don't have as much flexibility in 
  their choice of work issues. This resulted in an imbalance of workload between group members where some would be left with multiple issues with more weight in 
  work when compared to previously completed issues, and at times people who did their share of work needed to complete extra for the sake of time. Our resolution plan 
  moving forward is to delegate tasks in a different manner, which will be dividing sections so everyone has equal weight in work before anyone starts working 
  on a deliverable.
  }

  \item How many of your requirements were inspired by speaking to your
  client(s) or their proxies (e.g. your peers, stakeholders, potential users)?
  \par{Its hard to keep track of how many requirements exactly were inspired by our conversations with our clients and their proxies, but I'll provide the 
  general requirement section names which were mostly or fully completely based on our discussions. Our mandated constraints were heavily based on context 
  provided to us about the current and future system implementation. The functional requirements we created were directly inspired by what features our clients wanted 
  to see us deliver with the project. The requirement sections related to the usability and user experience of the software, which were look and feel 
  and usability and humanity requirements, were inspired by our clients goals of creating easy-to-use and globally accessible software. Lastly, since our clients are 
  actively involved with the current system in place, we were able to clearly interpret their requirements in the operational and environmental domain, 
  as well as security and compliance requirements.}

  \item Which of the courses you have taken, or are currently taking, will help
  your team to be successful with your capstone project.
  \par{Some elements from the project-based courses, most notably 1P13 and 2PX3, will definitely help with the pacing and iterative nature in terms of development and 
  documentation of our project. Also from those project based courses, we can use principles we have learned about working in a group and doing presentations which will 
  be more prevalent further in the project timeline. On the topic of workflows and development cycles, I think we will need to adopt one or two development cycles like 
  Agile Methodology coupled with Test Driven Development for example. We have each had much practice with some kind of development cycle from our Software Design I and III 
  courses where we focused on building larger-scale software during a time period, and we will definitely need to apply what we have learned from those for our documentation 
  and development. Furthermore, one of our main focus points for this project is to create a user friendly and accessible interface for users of our software. We are all 
  currently in a Human Computer Interfaces class where we will need to apply what we learn about user elicitation, design, and experience. Lastly, our project will heavily rely 
  on data that will be visualized and possible pre-processed by users. This will be a perfect opportunity to apply our learnings from our Database Design course and our Concurrent 
  System Design course, both of which go hand-in-hand when designing efficient, scalable, and safe databases.}
\end{enumerate}
