\documentclass{article}

\usepackage{booktabs}
\usepackage{tabularx}
\usepackage{hyperref} 

\title{Development Plan\\\progname}

\author{\authname}

\date{}

%% Comments

\usepackage{color}

\newif\ifcomments\commentstrue %displays comments
%\newif\ifcomments\commentsfalse %so that comments do not display

\ifcomments
\newcommand{\authornote}[3]{\textcolor{#1}{[#3 ---#2]}}
\newcommand{\todo}[1]{\textcolor{red}{[TODO: #1]}}
\else
\newcommand{\authornote}[3]{}
\newcommand{\todo}[1]{}
\fi

\newcommand{\wss}[1]{\authornote{magenta}{SS}{#1}} 
\newcommand{\plt}[1]{\authornote{cyan}{TPLT}{#1}} %For explanation of the template
\newcommand{\an}[1]{\authornote{cyan}{Author}{#1}}

%% Common Parts

\newcommand{\progname}{Software Engineering} % PUT YOUR PROGRAM NAME HERE
\newcommand{\authname}{Team \#18, Gouda Engineers 
\\ Aidan Goodyer
\\ Jeremy Orr
\\ Leo Vugert
\\ Nathan Perry
\\ Tim Pokanai} % AUTHOR NAMES                  

\usepackage{hyperref}
    \hypersetup{colorlinks=true, linkcolor=blue, citecolor=blue, filecolor=blue,
                urlcolor=blue, unicode=false}
    \urlstyle{same}
                                


\begin{document}

\maketitle

\begin{table}[hp]
\caption{Revision History} \label{TblRevisionHistory}
\begin{tabularx}{\textwidth}{llX}
\toprule
\textbf{Date} & \textbf{Developer(s)} & \textbf{Change}\\
\midrule
17 September 2025 & Jeremy & Added 2, 3, 5, 6, 7\\
Date2 & Name(s) & Description of changes\\
... & ... & ...\\
\bottomrule
\end{tabularx}
\end{table}

\newpage{}

\wss{Put your introductory blurb here.  Often the blurb is a brief roadmap of
what is contained in the report.}

\wss{Additional information on the development plan can be found in the
\href{https://gitlab.cas.mcmaster.ca/courses/capstone/-/blob/main/Lectures/L02b_POCAndDevPlan/POCAndDevPlan.pdf?ref_type=heads}
{lecture slides}.}

\section{Confidential Information?}

%\wss{State whether your project has confidential information from industry, or
%not.  If there is confidential information, point to the agreement you have in
%place.}

\par{All data that will be used in any matter over the duration of this project is already publicly available on the Federated Research Data Repository (FRDR)
thus no confidential information will be contained in this project.}

\wss{For most teams this section will just state that there is no confidential
information to protect.}
\section{IP to Protect}

\paragraph{There is no IP to protect for the project. The data provided is openly avaiable to use. Likewise, our code will be open source.}

\wss{State whether there is IP to protect.  If there is, point to the agreement.
All students who are working on a project that requires an IP agreement are also
required to sign the ``Intellectual Property Guide Acknowledgement.''}

\section{Copyright License}

\paragraph{The software that we will use will be MIT. The License is linked titled LICENSE and is at the root directory of our repository.}

\wss{What copyright license is your team adopting.  Point to the license in your
repo.}

\section{Team Meeting Plan}

%\wss{How often will you meet? where?}

%\wss{If the meeting is a physical location (not virtual), out of an abundance of
%caution for safety reasons you shouldn't put the location online}

%\wss{How often will you meet with your industry advisor?  when?  where?}

%\wss{Will meetings be virtual?  At least some meetings should likely be
%in-person.}

\par{\indent Meetings will occur every Monday from 9:30 a.m. to 10:30 a.m. and every Friday from 12:30 p.m. to 1:30 p.m. Meetings will
be in person in MDCL 2246 or virtual as needed. If extra meetings are needed, they will be scheduled on an ad-hoc basis with priority given to pre-established capstone timeslots.

Our group will plan to meet with the project supervisors once per week. This is simply a guideline and ad-hoc meetings
will be added as needed. Additionally, all project documents will be forwarded to the supervisors to keep them in the loop and receive any feedback they want to provide.

Tim Pokanai is the group leader and thus will be responsible for leading team meetings. An agenda with the necessary 
discussion items will be collaboratively created ahead of time (likely through text).

A kanban board has been created on github for aiding in communication and work splitting. All action items for each deliverable will
be added to the kanban board so that team members can claim sections they are working on.
}

%\wss{How will the meetings be structured?  There should be a chair for all meetings.  There should be an agenda for all meetings.}

\section{Team Communication Plan}

%\wss{Issues on GitHub should be part of your communication plan.}
Our team will primarily communicate using Microsoft Teams and text messaging for quick, day-to-day coordination. For interactions with the professors associated with our project, we will schedule weekly or ad hoc meetings as needed. Email will be our main channel for formal communication with faculty.

To manage and divide work efficiently, we are utilizing GitHub Issues and a Kanban board. GitHub Issues allow us to assign tasks, track progress, and facilitate transparent communication among team members. The Kanban board complements this by providing a clear visual representation of task status and workflow. This will support project planning and execution.

\wss{Issues on GitHub should be part of your communication plan.}

\section{Team Member Roles}

\paragraph{Roles} 

\begin{description}
    \item[Tim - Team Leader] Oversees members, helps assign tasks, and ensures deadlines are met.
    \item[Nathan - Administrator] Manages documentation, project logistics, and resources.
    \item[Leo - Meeting Chair] Organizes and leads team meetings, sets agendas, and keeps discussions on track.
    \item[Jeremy / Aidan - Software Specalists] Assist with development tasks such as CI/CD setup, code reviews, general project support, and note taking.
\end{description}


\wss{You should identify the types of roles you anticipate, like notetaker,
leader, meeting chair, reviewer.  Assigning specific people to those roles is
not necessary at this stage.  In a student team the role of the individuals will
likely change throughout the year.}

\section{Workflow Plan}

\begin{itemize}
  \item How will you be using git, including branches, pull requests, etc.?

  The team will use Git for version control. Branches will follow the convention:
  \texttt{<type-of-work>/<issue-name>}, where the types of work are:
  \begin{itemize}
      \item documentation
      \item feature
      \item bugfix
  \end{itemize}

  Issue name will follow the name of the issue you are working on.

  For pull requests, GitHub Actions will compile the code and run checks. A reviewer will always be required. Once a branch is merged, it will be deleted.

  \item How will you be managing issues, including template issues, issue classification, etc.?
  
  We will manage tasks and bugs using GitHub Issues. To ensure consistency and clarity, we will use issue templates for common types of issues such as feature requests, bug reports, and documentation tasks. Issues will be classified using labels (e.g., "bug", "enhancement", "in progress", "needs review") to track progress and priority. Each issue will be assigned to a team member and linked to relevant pull requests when applicable. This structured approach will help us maintain transparency, accountability, and effective collaboration throughout the development process.

  \item Use of CI/CD
  
  We will implement Githb actions for CI/CD. This will include automated checks to ensure that the code compiles correctly, passes all tests, and adheres to project standards before being merged into the main branch. By integrating CI/CD, we aim to catch issues early, maintain code quality, and streamline the deployment process.  



\end{itemize}


\section{Project Decomposition and Scheduling}

\begin{itemize}
  \item How will you be using GitHub projects?
  
  We will create GitHub issues for each task, and maintain a Kanban board for issues related to the current deliverable.
  The Kanban board will have 4–5 stages. For now, we will include stages: To Do, In Progress, Peer Review, and Completed. The main use of Github projects is to have good and clear project management.

  \item Include a link to your GitHub project
  
  \href{https://github.com/NathanPerry11/OCD-Rat-Infrastructure}{OCD‑Rat‑Infrastructure on GitHub}
\end{itemize}

\wss{How will the project be scheduled?  This is the big picture schedule, not
details. You will need to reproduce information that is in the course outline
for deadlines.}

\section{Proof of Concept Demonstration Plan}

What is the main risk, or risks, for the success of your project?  What will you
demonstrate during your proof of concept demonstration to convince yourself that
you will be able to overcome this risk?

\par{The main risk arises when dealing with the integration and management of the public OCD rat data set with our application.
 The data set is quite large and heterogeneous, with approximately 20,000 trials worth of data contaning video recordings, trajectory data, 
 and detailed metadata corresponding to the trials. Thus, the main risk stems from our system architecture and database design. 
 If our overall architecture proves to be inadequate, we risk having poor performance in querying and loading data, as well as incorrect 
 and inconsistent results in our natural language querying and when integrating related complex data types. \newline \indent
 Our proof of concept will address these risks by demonstrating that data from a representative subset of the dataset, 
 containing roughly 1000 trials, can be queried and visualized in a manner that is both efficient and user-friendly. 
 We will specifically demonstrate that trial data can be filtered based on metadata corresponding to experimental variables, videos 
 can be synchronized with their corresponding trajectories, all of which can be done through a simple Minimal Viable Product (MVP) 
 user interface. Implementing these features on a small scale will demonstrate evidence that our design will perform and 
 be consistent at scale, and most importantly will lay the foundations to upscale to the full data set while supporting 
 natural language querying.}

\section{Expected Technology}

\wss{What programming language or languages do you expect to use?  What external
libraries?  What frameworks?  What technologies.  Are there major components of
the implementation that you expect you will implement, despite the existence of
libraries that provide the required functionality.  For projects with machine
learning, will you use pre-trained models, or be training your own model?  }

A core module of the project is a web interface for researchers to query and interact with data from past experiments. 
The team anticipates to use some combination of modern web tools, including HTML, JavaScript, CSS, and a web framework such as React to facilitate the development of this module. 

Another core component of the system is the database infrastructure needed to support the querying and storage of the experiment data. The team anticipates using a relational database, such as PostgreSQL or MySQL, to store and manage the data. One or many APIs may be needed to define an interface for interacting with the tools. The team plans to use a framework such as Java Spring Boot to implement these services. 

Further goals include the implementation of data analysis and visualization tools. The team anticipates using Python for this purpose, as this is the lowest common denominator tool for many researchers. Another stakeholder request is the capability for natural language querying of data attributes over the trial datasets. The team anticipates using some combination of pre-trained language models like GPT-5 or equivalent to facilitate this task. 

Testing is another important aspect of the project. As test frameworks are language specific, the team will select an appropriate test framework for the language used. For frontend JavaScript, the team may use Jest. For backend in Java/Python, the team may use JUnit/PyTest in conjunction with other code quality and code coverage tools like SonarQube.

For CI/CD purposes, the team plans to use GitHub Actions to define workflows to build and test the code automatically on each commit and pull request.



\wss{The implementation decisions can, and likely will, change over the course
of the project.  The initial documentation should be written in an abstract way;
it should be agnostic of the implementation choices, unless the implementation
choices are project constraints.  However, recording our initial thoughts on
implementation helps understand the challenge level and feasibility of a
project.  It may also help with early identification of areas where project
members will need to augment their training.}

Topics to discuss include the following:

\begin{itemize}
\item Specific programming language
\item Specific libraries
\item Pre-trained models
\item Specific linter tool (if appropriate)
\item Specific unit testing framework
\item Investigation of code coverage measuring tools
\item Specific plans for Continuous Integration (CI), or an explanation that CI
  is not being done
\item Specific performance measuring tools (like Valgrind), if
  appropriate
\item Tools you will likely be using?
\end{itemize}

\wss{git, GitHub and GitHub projects should be part of your technology.}

\section{Coding Standard}

\wss{What coding standard will you adopt?}

Team members will follow the \href{https://github.com/google/styleguide}{Google Style Guides} for the respective programming languages used in the project. This will help ensure consistencty and define a single source of truth for language conventions. 
Linters will be used to enforce the standards defined in the style guides. All team members will be expected to run the linters locally before pushing code to the repository, as well as configure their linter to follow the defined ruleset. 

Regarding software testing, the team will enforce a minimum of 80\% code coverage for all new code added to the repository. This will be measured using a covererage tool integrated into the CI/CD pipeline. Pull requests will require passing tests and meeting the code coverage threshold before being merged into the main branch.

In addition to the above, all team members will be required to write clear documentation for new functions, classes, and modules, as well as clear commit messages and pull request descriptions. This must include the purpose and scope of the change, as well as an outline of the testing plan. 

\newpage{}

\section*{Appendix --- Reflection}

\wss{Not required for CAS 741}

The purpose of reflection questions is to give you a chance to assess your own
learning and that of your group as a whole, and to find ways to improve in the
future. Reflection is an important part of the learning process.  Reflection is
also an essential component of a successful software development process.  

Reflections are most interesting and useful when they're honest, even if the
stories they tell are imperfect. You will be marked based on your depth of
thought and analysis, and not based on the content of the reflections
themselves. Thus, for full marks we encourage you to answer openly and honestly
and to avoid simply writing ``what you think the evaluator wants to hear.''

Please answer the following questions.  Some questions can be answered on the
team level, but where appropriate, each team member should write their own
response:


\begin{enumerate}
    \item \textbf{Why is it important to create a development plan prior to starting the project?}

    The creation of a development plan prior to the start of a project is crucial as it lays the foundation and aligns the group up to completion about key decisions for project. After our team selected our topic, we were quite unorganized and unsure what the next steps were. We started meeting during our breaks to figure out our plan and work on the development plan. This would eventually be the beginning of our planning for the development process. We started by finding two 1 hour long time slots we could meet and organizing our methods of communication. From this we started assigning our roles and responsibilities and other key information about our project that would build the foundation for the rest of the course. Without the development plan, our group would be starting with 5 different perspectives on the project in all different areas, it was a great way for us to align ourselves on decisions of how the project is going to be built before starting and make sure that we were all on the same page from the beginning.

    \item \textbf{In your opinion, what are the advantages and disadvantages of using CI/CD?}

    \textbf{Advantages:}

        -- CI/CD offers faster code delivery and deployment into production
      
        -- Code is delivered in smaller increments, allowing developers to review code, and fix bugs faster and earlier
      
        -- Tools integrated into the CI/CD pipeline helps developers find and fix bugs earlier, reducing time spent on manual testing
      
        -- Instead of manual intervention, code is fit through an automatic pipeline to be pushed to production
      
        -- Unit tests, integration tests, and other checks help review code and find bugs

    \textbf{Disadvantages:}

        -- Creating fully automated pipelines takes time, resources, and a specific skillset
      
        -- Devs must be trained to understand and familiarize themselves with each specific pipeline 
      
        -- Steep learning curve to learn, requires discipline and commitment to writing high quality code, can’t be rushed
      
        -- Robust infrastructure is imperative for a good CI/CD pipeline, otherwise it can produce bottlenecks or errors

    \item \textbf{What disagreements did your group have in this deliverable, if any, and how did you resolve them?}

	  For the development plan, we had no specific disagreements, we all agreed on most of the plans and processes mentioned. Our biggest disagreements came from selecting a project to at the beginning of our capstone. As a group of 5 different people with different ideas, we couldn’t come to one single project we wanted to do straight away. Our process to resolve this was by going through the project list as a group and writing down any project that had any interest in it. At the end we also brainstormed a couple project ideas of our own to add to the list. From this we ranked our top three projects and assigned 1, 2, and 3 points to them. The projects with the highest amount of total points were kept and the rest were discarded. We repeated this process until we had 3-4 projects at the top of our list. From this we reached out to the professor and the other supervisors from the other projects. After coming to the conclusion that our top project idea (which was from our brainstorm) would be too difficult to accomplish, as a group we decided we would proceed with this project after the professor accepted us.

\end{enumerate}

\newpage{}

\section*{Appendix --- Team Charter}

\wss{borrows from
\href{https://engineering.up.edu/industry_partnerships/files/team-charter.pdf}
{University of Portland Team Charter}}

\subsection*{External Goals}

% \wss{What are your team's external goals for this project? These are not the
% goals related to the functionality or quality fo the project.  These are the
% goals on what the team wishes to achieve with the project.  Potential goals are
% to win a prize at the Capstone EXPO, or to have something to talk about in
% interviews, or to get an A+, etc.}

\par{1. To have an impressive project to put on a resume and to show off to potential employers. A tangible creation that showcases our abilities
and expertise in the software engineering realm.\newline\newline
2. To create something that has genuine utility in the real world and that is used by people in the real world because it brings value to their lives.
In other words, to create a legitimate engineering project.\newline\newline
3. To hone our skills in the software realm. From backend to frontend, databases to natural language processing, this project should make us feel like experts 
in our domain and provide the confidence that we can create a software project from end to end.\newline\newline
4. To acheive a good mark in the capstone course.}

\subsection*{Attendance}

\subsubsection*{Expectations}

% \wss{What are your team's expectations regarding meeting attendance (being on
% time, leaving early, missing meetings, etc.)?}

\par{All members are expected to attend all scheduled meetings and be reasonably on time (5 or so minutes late). Leaving meetings early is acceptable as long
as all planned action items have already been covered and their presence is no longer necessary. As university students, everyone is very busy and missing
meetings from time to time is both understandable and acceptable, given that a sound reason is provided and given ahead of time. See below for acceptable excuses.}

\subsubsection*{Acceptable Excuse}

% \wss{What constitutes an acceptable excuse for missing a meeting or a deadline?
% What types of excuses will not be considered acceptable?}

\par{Acceptable excuses include any conflicts related to one's academic or professional career. Meetings for other classes, job interviews or
needing to study for a midterm are examples of acceptable excuses, given that reasonable effort was made to prevent or reschedule conflicts (e.g. A team member
neglecting to study for a midterm and then needing to skip a meeting to cram is not a reasonable effort). Personal emergencies are of course also acceptable
excuses. Examples of unacceptable excuses are those which are both preventable and related to a subject that should not take priority over the capstone project.
For example, a team member being too hungover despite knowing of the scheduled meeting or staying up too late playing video games.}

\subsubsection*{In Case of Emergency}

\par{ In the case of an emergency, members will not be expected to attend meetings and can be caught up after the fact once they become available. 
If the emergency results in them being unable to complete their individual work for a deliverable, the group will do their best to pick up the slack
and deliver what needs to be done. Of course, this must be a reasonable task and based on the quantity of work and the amount of time until the deadline,
a decision must be made if it is realistic to pick up the slack. If it is not, professors, TAs and supervisors (if relevant) must be contacted and informed
of the situation and new expectations must be set.}

% \wss{What process will team members follow if they have an emergency and cannot
% attend a team meeting or complete their individual work promised for a team
% deliverable?}

\subsection*{Accountability and Teamwork}

\subsubsection*{Quality} 

% \wss{What are your team's expectations regarding the quality
% of team members' preparation for team meetings and the quality of the
% deliverables that members bring to the team?}

\par{ There is no specific requirement to formally prepare anything for meetings aside from a general agenda of what needs to be discussed which will be
composed beforehand by the group as a collective. However, the quality of preparation should be that each member is aware of what will be discussed and 
has collected thoughts or familiarized themselves on the subject so they can meaningfully contribute to the discussion. If the action item specifically 
refers to a subject one group member is responsible for or has been extensively working on, they should be prepared to lead the disucssion. \newline\newline\indent
Deliverable quality also does not have explicit requirements but it must pass three checks that will be done for each deliverable.\newline\newline
1. All code written must comply with the our groups coding standard.\newline
2. All pull requests must hold up to the scrutiny of a reviewer that is responsible for critiquing the changes and provide feedback as needed (this cannot be the person who made the request).\newline
3. In the case of written deliverables, all work must also hold up to a final document review done collectively by the group at the time of submission.}  

\subsubsection*{Attitude}

% \wss{What are your team's expectations regarding team members' ideas,
% interactions with the team, cooperation, attitudes, and anything else regarding
% team member contributions?  Do you want to introduce a code of conduct?  Do you
% want a conflict resolution plan?  Can adopt existing codes of conduct.}

\par{ All team members already have strong relationships with each other and thus we already have an implicit framework for interaction and cooperation.
We all recognize each other as competent individuals so team members' ideas should and will be openly welcomed but also openly critiqued if a team
member feels it is necessary to push back. Fortunately we have a comfort level that allows us to push back on ideas without fear of being made socially
uncomfortable. This also includes bringing up any unmet expectations within the group. Our relationships and knowledge that each member is willing
and able to do their part allows us to comfortably address anyone that is not doing their part. \newline\newline\indent

As friends, we have also experienced conflict with each other in the past and although it is never enjoyable, we are comfortable with confrontation
and are confident in our ability to resolve conflicts peacefully and thus a code of conduct or conflict resolution plan is not necessary.}

\subsubsection*{Stay on Track}

% \wss{What methods will be used to keep the team on track? How will your team
% ensure that members contribute as expected to the team and that the team
% performs as expected? How will your team reward members who do well and manage
% members whose performance is below expectations?  What are the consequences for
% someone not contributing their fair share?}

% \wss{You may wish to use the project management metrics collected for the TA and
% instructor for this.}

% \wss{You can set target metrics for attendance, commits, etc.  What are the
% consequences if someone doesn't hit their targets?  Do they need to bring the
% coffee to the next team meeting?  Does the team need to make an appointment with
% their TA, or the instructor?  Are there incentives for reaching targets early?}

\par{ The main method to keep the team on track will be the kanban board in the projects of the repository. At the beginning of every
deliverable/iteration, the kanban should be filled with all of the required action items for that deliverable/iteration. This gives visibility 
to all the things that need to be done and due to the assignment feature of the kanban, everyone can see exactly how much work each member
is comitting to take on. This visibility ensures that each group member is comitting to complete a reasonable amount of work.\newline\newline
If a member does not contribute their fair share, the kanban will provide visibility to this and ideally, the team will be on top of them
before the deliverable is due to avoid missing expectations. If a group member does not meet expectations, we do not believe there should be
any specific punishments and we don't believe that any group member should be rewarded and thus encouraged to take on more than their fair share. We believe
that talking to the member that did not meet expectations will be sufficient and if it is a continued pattern we will make the decision of whether or not to
reach out to our capstone professor who will then provide potential consequences.\newline\newline

Note that our main performance metrics will be number of commits and number of kanban issues resolved although performance in these metrics
will not necessarily have any material benefit for any specific group member. We are a team, we do not want to encourage forming a hierarchy 
based on material rewards and punishment.}

\subsubsection*{Team Building}

% \wss{How will you build team cohesion (fun time, group rituals, etc.)? }

\par{ All members of the group already have relatively strong relationships with each other both inside and outside of class. Planned fun time or group rituals are
not necessary as they will happen organically as a part of everyone's general social life.}

\subsubsection*{Decision Making} 

% \wss{How will you make decisions in your group? Consensus?  Vote? How will you
% handle disagreements? }

\par{ Generally, decisions will be made by consensus. If there are
disagreements, we will try to resolve them conversationally to see if a gap can be bridged and a solution devised. If it can not be resolved, we will
resort to a vote in which all team members will participate in and, as a group of five, the majority vote will be the direction the team moves in. }

\end{document}
