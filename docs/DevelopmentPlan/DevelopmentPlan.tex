\documentclass{article}

\usepackage{booktabs}
\usepackage{tabularx}
\usepackage{hyperref} 

\title{Development Plan\\\progname}

\author{\authname}

\date{}

%% Comments

\usepackage{color}

\newif\ifcomments\commentstrue %displays comments
%\newif\ifcomments\commentsfalse %so that comments do not display

\ifcomments
\newcommand{\authornote}[3]{\textcolor{#1}{[#3 ---#2]}}
\newcommand{\todo}[1]{\textcolor{red}{[TODO: #1]}}
\else
\newcommand{\authornote}[3]{}
\newcommand{\todo}[1]{}
\fi

\newcommand{\wss}[1]{\authornote{magenta}{SS}{#1}} 
\newcommand{\plt}[1]{\authornote{cyan}{TPLT}{#1}} %For explanation of the template
\newcommand{\an}[1]{\authornote{cyan}{Author}{#1}}

%% Common Parts

\newcommand{\progname}{Software Engineering} % PUT YOUR PROGRAM NAME HERE
\newcommand{\authname}{Team \#18, Gouda Engineers 
\\ Aidan Goodyer
\\ Jeremy Orr
\\ Leo Vugert
\\ Nathan Perry
\\ Tim Pokanai} % AUTHOR NAMES                  

\usepackage{hyperref}
    \hypersetup{colorlinks=true, linkcolor=blue, citecolor=blue, filecolor=blue,
                urlcolor=blue, unicode=false}
    \urlstyle{same}
                                


\begin{document}

\maketitle

\begin{table}[hp]
\caption{Revision History} \label{TblRevisionHistory}
\begin{tabularx}{\textwidth}{llX}
\toprule
\textbf{Date} & \textbf{Developer(s)} & \textbf{Change}\\
\midrule
17 September 2025 & Jeremy & Added 2, 3, 5, 6, 7\\
Date2 & Name(s) & Description of changes\\
... & ... & ...\\
\bottomrule
\end{tabularx}
\end{table}

\newpage{}

\wss{Put your introductory blurb here.  Often the blurb is a brief roadmap of
what is contained in the report.}

\wss{Additional information on the development plan can be found in the
\href{https://gitlab.cas.mcmaster.ca/courses/capstone/-/blob/main/Lectures/L02b_POCAndDevPlan/POCAndDevPlan.pdf?ref_type=heads}
{lecture slides}.}

\section{Confidential Information?}

%\wss{State whether your project has confidential information from industry, or
%not.  If there is confidential information, point to the agreement you have in
%place.}

\par{All data that will be used in any matter over the duration of this project is already publicly available on the Federated Research Data Repository (FRDR)
thus no confidential information will be contained in this project.}

\wss{For most teams this section will just state that there is no confidential
information to protect.}
\section{IP to Protect}

\paragraph{There is no IP to protect for the project. The data provided is openly avaiable to use. Likewise, our code will be open source.}

\wss{State whether there is IP to protect.  If there is, point to the agreement.
All students who are working on a project that requires an IP agreement are also
required to sign the ``Intellectual Property Guide Acknowledgement.''}

\section{Copyright License}

\paragraph{The software that we will use will be MIT. The License is linked titled LICENSE and is at the root directory of our repository.}

\wss{What copyright license is your team adopting.  Point to the license in your
repo.}

\section{Team Meeting Plan}

%\wss{How often will you meet? where?}

%\wss{If the meeting is a physical location (not virtual), out of an abundance of
%caution for safety reasons you shouldn't put the location online}

%\wss{How often will you meet with your industry advisor?  when?  where?}

%\wss{Will meetings be virtual?  At least some meetings should likely be
%in-person.}

\par{Meetings will occur every Monday from 9:30 a.m. to 10:30 a.m. and every Friday 12:30 p.m. to 1:30 p.m. Meetings will
be in person in MDCL 2246. If extra meetings are needed, they will be scheduled on an ad-hoc basis with priority given to pre-established capstone timeslots.

Our group will plan to meet with the project supervisors once per week. This is simply a guideline and ad-hoc meetings
will be added as needed. Additionally, all project documents will be forwarded to the supervisors to keep them in the loop and receive any feedback they want to provide.

Tim Pokanai is the group leader and thus will be responsible for leading team meetings. An agenda with the necessary 
discussion items will be collaboratively created ahead of time (likely through text).

A kanban board has been created on github for aiding in communication and work splitting. All action items for each deliverable will
be added to the kanban board so that team members can claim sections they are working on.
}

%\wss{How will the meetings be structured?  There should be a chair for all meetings.  There should be an agenda for all meetings.}

\section{Team Communication Plan}

%\wss{Issues on GitHub should be part of your communication plan.}
Our team will primarily communicate using Microsoft Teams and text messaging for quick, day-to-day coordination. For interactions with the professors associated with our project, we will schedule weekly or ad hoc meetings as needed. Email will be our main channel for formal communication with faculty.

To manage and divide work efficiently, we are utilizing GitHub Issues and a Kanban board. GitHub Issues allow us to assign tasks, track progress, and facilitate transparent communication among team members. The Kanban board complements this by providing a clear visual representation of task status and workflow. This will support project planning and execution.

\wss{Issues on GitHub should be part of your communication plan.}

\section{Team Member Roles}

\paragraph{Roles} 

\begin{description}
    \item[Tim - Team Leader] Oversees members, helps assign tasks, and ensures deadlines are met.
    \item[Nathan - Administrator] Manages documentation, project logistics, and resources.
    \item[Leo - Meeting Chair] Organizes and leads team meetings, sets agendas, and keeps discussions on track.
    \item[Jeremy / Aidan - Software Specalists] Assist with development tasks such as CI/CD setup, code reviews, general project support, and note taking.
\end{description}


\wss{You should identify the types of roles you anticipate, like notetaker,
leader, meeting chair, reviewer.  Assigning specific people to those roles is
not necessary at this stage.  In a student team the role of the individuals will
likely change throughout the year.}

\section{Workflow Plan}

\begin{itemize}
  \item How will you be using git, including branches, pull requests, etc.?

  The team will use Git for version control. Branches will follow the convention:
  \texttt{<type-of-work>/<issue-name>}, where the types of work are:
  \begin{itemize}
      \item documentation
      \item feature
      \item bugfix
  \end{itemize}

  Issue name will follow the name of the issue you are working on.

  For pull requests, GitHub Actions will compile the code and run checks. A reviewer will always be required. Once a branch is merged, it will be deleted.

  \item How will you be managing issues, including template issues, issue classification, etc.?
  
  We will manage tasks and bugs using GitHub Issues. To ensure consistency and clarity, we will use issue templates for common types of issues such as feature requests, bug reports, and documentation tasks. Issues will be classified using labels (e.g., "bug", "enhancement", "in progress", "needs review") to track progress and priority. Each issue will be assigned to a team member and linked to relevant pull requests when applicable. This structured approach will help us maintain transparency, accountability, and effective collaboration throughout the development process.

  \item Use of CI/CD
  
  We will implement Githb actions for CI/CD. This will include automated checks to ensure that the code compiles correctly, passes all tests, and adheres to project standards before being merged into the main branch. By integrating CI/CD, we aim to catch issues early, maintain code quality, and streamline the deployment process.  



\end{itemize}


\section{Project Decomposition and Scheduling}

\begin{itemize}
  \item How will you be using GitHub projects?
  
  We will create GitHub issues for each task, and maintain a Kanban board for issues related to the current deliverable.
  The Kanban board will have 4–5 stages. For now, we will include stages: To Do, In Progress, Peer Review, and Completed. The main use of Github projects is to have good and clear project management.

  \item Include a link to your GitHub project
  
  \href{https://github.com/NathanPerry11/OCD-Rat-Infrastructure}{OCD‑Rat‑Infrastructure on GitHub}
\end{itemize}

\wss{How will the project be scheduled?  This is the big picture schedule, not
details. You will need to reproduce information that is in the course outline
for deadlines.}

\section{Proof of Concept Demonstration Plan}

What is the main risk, or risks, for the success of your project?  What will you
demonstrate during your proof of concept demonstration to convince yourself that
you will be able to overcome this risk?

\par{The main risk arises when dealing with the integration and management of the public OCD rat data set with our application.
 The data set is quite large and heterogeneous, with approximately 20,000 trials worth of data contaning video recordings, trajectory data, 
 and detailed metadata corresponding to the trials. Thus, the main risk stems from our system architecture and database design. 
 If our overall architecture proves to be inadequate, we risk having poor performance in querying and loading data, as well as incorrect 
 and inconsistent results in our natural language querying and when integrating related complex data types. \newline \indent
 Our proof of concept will address these risks by demonstrating that data from a representative subset of the dataset, 
 containing roughly 1000 trials, can be queried and visualized in a manner that is both efficient and user-friendly. 
 We will specifically demonstrate that trial data can be filtered based on metadata corresponding to experimental variables, videos 
 can be synchronized with their corresponding trajectories, all of which can be done through a simple Minimal Viable Product (MVP) 
 user interface. Implementing these features on a small scale will demonstrate evidence that our design will perform and 
 be consistent at scale, and most importantly will lay the foundations to upscale to the full data set while supporting 
 natural language querying.}

\section{Expected Technology}

\wss{What programming language or languages do you expect to use?  What external
libraries?  What frameworks?  What technologies.  Are there major components of
the implementation that you expect you will implement, despite the existence of
libraries that provide the required functionality.  For projects with machine
learning, will you use pre-trained models, or be training your own model?  }

A core module of the project is a web interface for researchers to query and interact with data from past experiments. 
The team anticipates to use some combination of modern web tools, including HTML, JavaScript, CSS, and a web framework such as React to facilitate the development of this module. 

Another core component of the system is the database infrastructure needed to support the querying and storage of the experiment data. The team anticipates using a relational database, such as PostgreSQL or MySQL, to store and manage the data. One or many APIs may be needed to define an interface for interacting with the tools. The team plans to use a framework such as Java Spring Boot to implement these services. 

Further goals include the implementation of data analysis and visualization tools. The team anticipates using Python for this purpose, as this is the lowest common denominator tool for many researchers. Another stakeholder request is the capability for natural language querying of data attributes over the trial datasets. The team anticipates using some combination of pre-trained language models like GPT-5 or equivalent to facilitate this task. 

Testing is another important aspect of the project. As test frameworks are language specific, the team will select an appropriate test framework for the language used. For frontend JavaScript, the team may use Jest. For backend in Java/Python, the team may use JUnit/PyTest in conjunction with other code quality and code coverage tools like SonarQube.

For CI/CD purposes, the team plans to use GitHub Actions to define workflows to build and test the code automatically on each commit and pull request.



\wss{The implementation decisions can, and likely will, change over the course
of the project.  The initial documentation should be written in an abstract way;
it should be agnostic of the implementation choices, unless the implementation
choices are project constraints.  However, recording our initial thoughts on
implementation helps understand the challenge level and feasibility of a
project.  It may also help with early identification of areas where project
members will need to augment their training.}

Topics to discuss include the following:

\begin{itemize}
\item Specific programming language
\item Specific libraries
\item Pre-trained models
\item Specific linter tool (if appropriate)
\item Specific unit testing framework
\item Investigation of code coverage measuring tools
\item Specific plans for Continuous Integration (CI), or an explanation that CI
  is not being done
\item Specific performance measuring tools (like Valgrind), if
  appropriate
\item Tools you will likely be using?
\end{itemize}

\wss{git, GitHub and GitHub projects should be part of your technology.}

\section{Coding Standard}

\wss{What coding standard will you adopt?}

Team members will follow the \href{https://github.com/google/styleguide}{Google Style Guides} for the respective programming languages used in the project. This will help ensure consistencty and define a single source of truth for language conventions. 
Linters will be used to enforce the standards defined in the style guides. All team members will be expected to run the linters locally before pushing code to the repository, as well as configure their linter to follow the defined ruleset. 

Regarding software testing, the team will enforce a minimum of 80\% code coverage for all new code added to the repository. This will be measured using a covererage tool integrated into the CI/CD pipeline. Pull requests will require passing tests and meeting the code coverage threshold before being merged into the main branch.

In addition to the above, all team members will be required to write clear documentation for new functions, classes, and modules, as well as clear commit messages and pull request descriptions. This must include the purpose and scope of the change, as well as an outline of the testing plan. 

\newpage{}

\section*{Appendix --- Reflection}

\wss{Not required for CAS 741}

The purpose of reflection questions is to give you a chance to assess your own
learning and that of your group as a whole, and to find ways to improve in the
future. Reflection is an important part of the learning process.  Reflection is
also an essential component of a successful software development process.  

Reflections are most interesting and useful when they're honest, even if the
stories they tell are imperfect. You will be marked based on your depth of
thought and analysis, and not based on the content of the reflections
themselves. Thus, for full marks we encourage you to answer openly and honestly
and to avoid simply writing ``what you think the evaluator wants to hear.''

Please answer the following questions.  Some questions can be answered on the
team level, but where appropriate, each team member should write their own
response:


\begin{enumerate}
    \item Why is it important to create a development plan prior to starting the
    project?
    \item In your opinion, what are the advantages and disadvantages of using
    CI/CD?
    \item What disagreements did your group have in this deliverable, if any,
    and how did you resolve them?
\end{enumerate}

\newpage{}

\section*{Appendix --- Team Charter}

\wss{borrows from
\href{https://engineering.up.edu/industry_partnerships/files/team-charter.pdf}
{University of Portland Team Charter}}

\subsection*{External Goals}

\wss{What are your team's external goals for this project? These are not the
goals related to the functionality or quality fo the project.  These are the
goals on what the team wishes to achieve with the project.  Potential goals are
to win a prize at the Capstone EXPO, or to have something to talk about in
interviews, or to get an A+, etc.}

\subsection*{Attendance}

\subsubsection*{Expectations}

\wss{What are your team's expectations regarding meeting attendance (being on
time, leaving early, missing meetings, etc.)?}

\subsubsection*{Acceptable Excuse}

\wss{What constitutes an acceptable excuse for missing a meeting or a deadline?
What types of excuses will not be considered acceptable?}

\subsubsection*{In Case of Emergency}

\wss{What process will team members follow if they have an emergency and cannot
attend a team meeting or complete their individual work promised for a team
deliverable?}

\subsection*{Accountability and Teamwork}

\subsubsection*{Quality} 

\wss{What are your team's expectations regarding the quality
of team members' preparation for team meetings and the quality of the
deliverables that members bring to the team?}

\subsubsection*{Attitude}

\wss{What are your team's expectations regarding team members' ideas,
interactions with the team, cooperation, attitudes, and anything else regarding
team member contributions?  Do you want to introduce a code of conduct?  Do you
want a conflict resolution plan?  Can adopt existing codes of conduct.}

\subsubsection*{Stay on Track}

\wss{What methods will be used to keep the team on track? How will your team
ensure that members contribute as expected to the team and that the team
performs as expected? How will your team reward members who do well and manage
members whose performance is below expectations?  What are the consequences for
someone not contributing their fair share?}

\wss{You may wish to use the project management metrics collected for the TA and
instructor for this.}

\wss{You can set target metrics for attendance, commits, etc.  What are the
consequences if someone doesn't hit their targets?  Do they need to bring the
coffee to the next team meeting?  Does the team need to make an appointment with
their TA, or the instructor?  Are there incentives for reaching targets early?}

\subsubsection*{Team Building}

\wss{How will you build team cohesion (fun time, group rituals, etc.)? }

\subsubsection*{Decision Making} 

\wss{How will you make decisions in your group? Consensus?  Vote? How will you
handle disagreements? }

\end{document}
