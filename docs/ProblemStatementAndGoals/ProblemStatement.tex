\documentclass{article}

\usepackage{tabularx}
\usepackage{booktabs}

\title{Problem Statement and Goals\\\progname}

\author{\authname}

\date{}

%% Comments

\usepackage{color}

\newif\ifcomments\commentstrue %displays comments
%\newif\ifcomments\commentsfalse %so that comments do not display

\ifcomments
\newcommand{\authornote}[3]{\textcolor{#1}{[#3 ---#2]}}
\newcommand{\todo}[1]{\textcolor{red}{[TODO: #1]}}
\else
\newcommand{\authornote}[3]{}
\newcommand{\todo}[1]{}
\fi

\newcommand{\wss}[1]{\authornote{magenta}{SS}{#1}} 
\newcommand{\plt}[1]{\authornote{cyan}{TPLT}{#1}} %For explanation of the template
\newcommand{\an}[1]{\authornote{cyan}{Author}{#1}}

%% Common Parts

\newcommand{\progname}{Software Engineering} % PUT YOUR PROGRAM NAME HERE
\newcommand{\authname}{Team \#18, Gouda Engineers 
\\ Aidan Goodyer
\\ Jeremy Orr
\\ Leo Vugert
\\ Nathan Perry
\\ Tim Pokanai} % AUTHOR NAMES                  

\usepackage{hyperref}
    \hypersetup{colorlinks=true, linkcolor=blue, citecolor=blue, filecolor=blue,
                urlcolor=blue, unicode=false}
    \urlstyle{same}
                                


\begin{document}

\maketitle

\begin{table}[hp]
\caption{Revision History} \label{TblRevisionHistory}
\begin{tabularx}{\textwidth}{llX}
\toprule
\textbf{Date} & \textbf{Developer(s)} & \textbf{Change}\\
\midrule
Date1 & Name(s) & Description of changes\\
Date2 & Name(s) & Description of changes\\
... & ... & ...\\
\bottomrule
\end{tabularx}
\end{table}

\section{Problem Statement}

\par{Current behavioral neuroscience research on Obsessive-Compulsive Disorder (OCD) is limited by the lack of 
accessible tools for managing and analyzing large-scale animal model data sets. With roughly 20,000 trials worth 
of rat behavioural data constituting one of the most extensive data sets in the field, the data set contains 
video recordings of rat behaviour with corresponding spatial-temporal tracking data, and additional research files. 
Although the data set is publicly accessible and contains rich and diverse forms of data, it is not presented in a 
user-friendly way for researchers to analyze. The current state of the data set impacts researchers by making it 
difficult to synchronize trajectories with video recordings and extract meaningful insights from correlated data types. 
This gap in infrastructure reduces the scientific utility of an extensive data set, ultimately slowing the pace of 
discovery through research.}

\wss{You should check your problem statement with the
\href{https://github.com/smiths/capTemplate/blob/main/docs/Checklists/ProbState-Checklist.pdf}
{problem statement checklist}.} 

\wss{You can change the section headings, as long as you include the required
information.}

\subsection{Problem}

\subsection{Inputs and Outputs}

\wss{Characterize the problem in terms of ``high level'' inputs and outputs.  
Use abstraction so that you can avoid details.}

\par{Inputs from users will consist of various query prompts or data visualization requests. These should be easily done
by users without technical experience. This will include: \newline \newline \indent 1. Submitting queries by applying filter criteria related to the experimental sessions or the data itself.
These could be but are not limited to: type of file, selected study/experiment, drug injection used in the trial,
rat bodyweight, date and time of trial, etc... \newline \newline \indent
2. Submitting queries using natural language. “show me trials with strong checking behavior after 5 injections”
or “find sessions where rats showed compulsive patterns” are examples of this \newline \newline \indent
3. Making requests for data visualizations such as behavioral metrics (seperated by injection type for example) or visually plotting trajectories of the rats
based on their (x,y) coordinates.}
\newline

\par{Outputs to users will be constrained to two main categories.\newline \newline \indent Category 1: raw data records returned from a query for the inputting user to view or extract
\newline \newline \indent Category 2: Data visualizations which include plots of rat trajectories or statistical or graphical displays of behavioural metrics of the rat subjects.}

\subsection{Stakeholders}

\par{The application produced from this project will affect multiple stakeholders, which will either be affected directly or indirectly affected:}

\begin{enumerate}
  \item \textbf{Direct Stakeholders:}
  \begin{itemize}
    \item {Behavioural Neuroscience Researchers: Researchers like Dr. Szechtman and Dr. Dvorkin-Gheva will be 
    end users of the platform and will be the core benefitors from the user-friendly search, visualization, and analysis of data.}
    \item {Data Scientists: This group of users will benefit from the structured database, API query system, 
    and processing tools included in the data processing pipeline.}
    \item {Graduate Students and Lab Members: These will be very frequent users of the application, who will benefit from the simplified 
    preprocessing, visualization, and analysis tools.}
  \end{itemize}
  \item \textbf{Indirect Stakeholders:}
  \begin{itemize}
    \item {OCD Clinicians: Clinicians don't use the platform directly, but they may incorporate researched insights into 
    a clinical understanding of OCD by refining trials and treatments.}
    \item {Patients with OCD: This stakeholder group may indirectly benefit with an improved quality of life
     from research findings, such as new treatments, accelerated by this platform.}
    \item {Global Research Community: These collaborating institutions, potentially involved in different areas of research, 
    may indirectly benefit from the streamlined data and its reusability, even if the platform isnt't built directly for them.}
  \end{itemize}
\end{enumerate}

\subsection{Environment}

\wss{Hardware and Software Environment}

\section{Goals}

  \subsection{Goal 1: Sound and Complete DBMS} 
  
  \par{ A complete database schema must be developed for the purposes
  of th is project. This means that all pieces of data in the FRDR repository have
  been accounted for and are uniquely identified within tables in the DBMS. This also means that
  all metadata related to each piece of data are accessible for querying. Finally,
  the relations between data must be properly represented. Video, track files and trajectory diagrams
  related to the same session must have relationships between them.
  
  Furthermore, a proper query framework must be set up and able to access the DBMS. 
  The MVP form of this would just be writing SQL into a database request and
  recieve the correct result. This MVP would essentially just need Rest APIs set up
  to make a request to the database. However, this should be abstracted with a filtering UI that
  allows for non-technical people to easily make queries.}

  \subsection{Goal 2: Sleek and Non-Technical UI}
  
  \par{The main user base for this project will be academics with a psychiatric or
  animal related background and thus the user interface needs to allow for non-technical
  people to easily make rather complex queries. Our goal is for our front end system to have an intuitive
  interface for filtering and searching for data. This would include features such as adding easily adding filters related to metadata annotations, a certain study,
  or the date of the trial. These filters would then be converted into an equivalent query, sent to the DBMS and return the relevant data. In a similar way,
  this UI will include natural language searching in which a sentence for what the user wants will be similarily converted into an equivalent query and the data returned.}

 \subsection{Goal 3: Data Visualization/Processing Capabilities} 
 
 \par{Due to the large amount of data in the repository, processing and visualization are important in
  helping users more easily interpret and draw conclusions from the data. The first part of this goal is to successfully implement an algorithm that
  can identify behaviours in the trials and group them accordingly. The MVP of this is simply an algorithm that can distinguish between compulsive and non-compulsive
  behaviour.
  
  On the visualization side, graphical displays of important metrics should be an option for users to view through the user interface. The different
  visualizations will likely need to be developed on an ad-hoc basis depending on the user needs but an example of an MVP for this goal would be to provide graphical
  displays of compulsive behavioural metrics based on the type of injection the rat recieved.}


\section{Stretch Goals}

\section{Extras}

\par{The first extra deliverable will be a performance report on areas of the software application where performance should be optimized. 
The second extra deliverable will be a user manual on the intended use of the software application.}

\wss{For CAS 741: State whether the project is a research project. This
designation, with the approval (or request) of the instructor, can be modified
over the course of the term.}

\wss{For SE Capstone: List your extras.  Potential extras include usability
testing, code walkthroughs, user documentation, formal proof, GenderMag
personas, Design Thinking, etc.  (The full list is on the course outline and in
Lecture 02.) Normally the number of extras will be two.  Approval of the extras
will be part of the discussion with the instructor for approving the project.
The extras, with the approval (or request) of the instructor, can be modified
over the course of the term.}

\newpage{}

\section*{Appendix --- Reflection}

\wss{Not required for CAS 741}

The purpose of reflection questions is to give you a chance to assess your own
learning and that of your group as a whole, and to find ways to improve in the
future. Reflection is an important part of the learning process.  Reflection is
also an essential component of a successful software development process.  

Reflections are most interesting and useful when they're honest, even if the
stories they tell are imperfect. You will be marked based on your depth of
thought and analysis, and not based on the content of the reflections
themselves. Thus, for full marks we encourage you to answer openly and honestly
and to avoid simply writing ``what you think the evaluator wants to hear.''

Please answer the following questions.  Some questions can be answered on the
team level, but where appropriate, each team member should write their own
response:


\begin{enumerate}
    \item What went well while writing this deliverable? 
    \item What pain points did you experience during this deliverable, and how
    did you resolve them?
    \item How did you and your team adjust the scope of your goals to ensure
    they are suitable for a Capstone project (not overly ambitious but also of
    appropriate complexity for a senior design project)?
\end{enumerate}  

\end{document}
