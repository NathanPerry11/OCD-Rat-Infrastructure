\documentclass[12pt, titlepage]{article}

\usepackage{fullpage}
\usepackage[round]{natbib}
\usepackage{multirow}
\usepackage{booktabs}
\usepackage{tabularx}
\usepackage{graphicx}
\usepackage{float}
\usepackage{hyperref}
\hypersetup{
    colorlinks,
    citecolor=blue,
    filecolor=black,
    linkcolor=red,
    urlcolor=blue
}

\input{../../Comments}
%% Common Parts

\newcommand{\progname}{Software Engineering} % PUT YOUR PROGRAM NAME HERE
\newcommand{\authname}{Team \#18, Gouda Engineers 
\\ Aidan Goodyer
\\ Jeremy Orr
\\ Leo Vugert
\\ Nathan Perry
\\ Tim Pokanai} % AUTHOR NAMES                  

\usepackage{hyperref}
    \hypersetup{colorlinks=true, linkcolor=blue, citecolor=blue, filecolor=blue,
                urlcolor=blue, unicode=false}
    \urlstyle{same}
                                


\newcounter{acnum}
\newcommand{\actheacnum}{AC\theacnum}
\newcommand{\acref}[1]{AC\ref{#1}}

\newcounter{ucnum}
\newcommand{\uctheucnum}{UC\theucnum}
\newcommand{\uref}[1]{UC\ref{#1}}

\newcounter{mnum}
\newcommand{\mthemnum}{M\themnum}
\newcommand{\mref}[1]{M\ref{#1}}

\begin{document}

\title{Module Guide for \progname{}} 
\author{\authname}
\date{\today}

\maketitle

\pagenumbering{roman}

\section{Revision History}

\begin{tabularx}{\textwidth}{p{3cm}p{2cm}X}
\toprule {\bf Date} & {\bf Version} & {\bf Notes}\\
\midrule
Date 1 & 1.0 & Notes\\
Date 2 & 1.1 & Notes\\
\bottomrule
\end{tabularx}

\newpage

\section{Reference Material}

This section records information for easy reference.

\subsection{Abbreviations and Acronyms}

\renewcommand{\arraystretch}{1.2}
\begin{tabular}{l l} 
  \toprule		
  \textbf{symbol} & \textbf{description}\\
  \midrule 
  AC & Anticipated Change\\
  DAG & Directed Acyclic Graph \\
  M & Module \\
  MG & Module Guide \\
  OS & Operating System \\
  R & Requirement\\
  SC & Scientific Computing \\
  SRS & Software Requirements Specification\\
  \progname & Explanation of program name\\
  UC & Unlikely Change \\
  \wss{etc.} & \wss{...}\\
  \bottomrule
\end{tabular}\\

\newpage

\tableofcontents

\listoftables

\listoffigures

\newpage

\pagenumbering{arabic}

\section{Introduction}

Decomposing a system into modules is a commonly accepted approach to developing
software.  A module is a work assignment for a programmer or programming
team~\citep{ParnasEtAl1984}.  We advocate a decomposition
based on the principle of information hiding~\citep{Parnas1972a}.  This
principle supports design for change, because the ``secrets'' that each module
hides represent likely future changes.  Design for change is valuable in SC,
where modifications are frequent, especially during initial development as the
solution space is explored.  

Our design follows the rules layed out by \citet{ParnasEtAl1984}, as follows:
\begin{itemize}
\item System details that are likely to change independently should be the
  secrets of separate modules.
\item Each data structure is implemented in only one module.
\item Any other program that requires information stored in a module's data
  structures must obtain it by calling access programs belonging to that module.
\end{itemize}

After completing the first stage of the design, the Software Requirements
Specification (SRS), the Module Guide (MG) is developed~\citep{ParnasEtAl1984}. The MG
specifies the modular structure of the system and is intended to allow both
designers and maintainers to easily identify the parts of the software.  The
potential readers of this document are as follows:

\begin{itemize}
\item New project members: This document can be a guide for a new project member
  to easily understand the overall structure and quickly find the
  relevant modules they are searching for.
\item Maintainers: The hierarchical structure of the module guide improves the
  maintainers' understanding when they need to make changes to the system. It is
  important for a maintainer to update the relevant sections of the document
  after changes have been made.
\item Designers: Once the module guide has been written, it can be used to
  check for consistency, feasibility, and flexibility. Designers can verify the
  system in various ways, such as consistency among modules, feasibility of the
  decomposition, and flexibility of the design.
\end{itemize}

The rest of the document is organized as follows. Section
\ref{SecChange} lists the anticipated and unlikely changes of the software
requirements. Section \ref{SecMH} summarizes the module decomposition that
was constructed according to the likely changes. Section \ref{SecConnection}
specifies the connections between the software requirements and the
modules. Section \ref{SecMD} gives a detailed description of the
modules. Section \ref{SecTM} includes two traceability matrices. One checks
the completeness of the design against the requirements provided in the SRS. The
other shows the relation between anticipated changes and the modules. Section
\ref{SecUse} describes the use relation between modules.

\section{Anticipated and Unlikely Changes} \label{SecChange}

This section lists possible changes to the system. According to the likeliness
of the change, the possible changes are classified into two
categories. Anticipated changes are listed in Section \ref{SecAchange}, and
unlikely changes are listed in Section \ref{SecUchange}.

\subsection{Anticipated Changes} \label{SecAchange}

Anticipated changes are the source of the information that is to be hidden
inside the modules. Ideally, changing one of the anticipated changes will only
require changing the one module that hides the associated decision. The approach
adapted here is called design for
change.

\begin{description}
\item[\refstepcounter{acnum} \actheacnum \label{acUIDesign}:] The layout, accessibility, 
  and responsiveness of the front-end interface.
\item[\refstepcounter{acnum} \actheacnum \label{acQueryLLM}:] Fine-tuning or using a 
  different query NLP Model.
\item[\refstepcounter{acnum} \actheacnum \label{acFiltersAndQueryParameters}:] The constraints of filter and query 
  metadata field inputs.
\item[\refstepcounter{acnum} \actheacnum \label{acDataVisualization}:] The framework used for 
  creating visualizations from data and presenting them.
\item[\refstepcounter{acnum} \actheacnum \label{acProcessingAlgorithms}:] The algorithms implemented for 
  behavioural analysis or classification.
\item[\refstepcounter{acnum} \actheacnum \label{acDatasetHosting}:] The virtual hosting location of 
  Dr. Szechtman's dataset.
\item[\refstepcounter{acnum} \actheacnum \label{acAccessControl}:] The implementation of the user authentication 
  and access control mechanism for scalability.
\item[\refstepcounter{acnum} \actheacnum \label{acErrorAndFaultHandling}:] The implementation of 
  error detection and fault recovery mechanisms.
\item[\refstepcounter{acnum} \actheacnum \label{acSecurityAndDependencyUpdates}:] The implementation of 
  application security and dependency monitoring strategies.
\end{description}

\subsection{Unlikely Changes} \label{SecUchange}

The module design should be as general as possible. However, a general system is
more complex. Sometimes this complexity is not necessary. Fixing some design
decisions at the system architecture stage can simplify the software design. If
these decision should later need to be changed, then many parts of the design
will potentially need to be modified. Hence, it is not intended that these
decisions will be changed.

\begin{description}
\item[\refstepcounter{ucnum} \uctheucnum \label{ucReadOnlyDataset}:] The read-only access 
  to Dr. Szechtman's dataset.
\item[\refstepcounter{ucnum} \uctheucnum \label{ucSoleDataSource}:] The dataset hosted on FRDR 
  being our primary and individual data source.
\item[\refstepcounter{ucnum} \uctheucnum \label{ucDataProcessingScope}:] Scope of data 
  processing analysis (Does not generalize rat behaviour trial data to other animal models 
  or domains).
\item[\refstepcounter{ucnum} \uctheucnum \label{ucMVCDesignPattern}:] The software application's MVC design pattern.
\item[\refstepcounter{ucnum} \uctheucnum \label{ucIO}:] Input/Output devices
  (Input: File and/or Keyboard, Output: File, Memory, and/or Screen).
\item[\refstepcounter{ucnum} \uctheucnum \label{ucUserComputingDevice}:] User's computing device 
  (The assumption that the user's device will support modern browsers with Javascript enabled, 
  will not change)
\end{description}

\section{Module Hierarchy} \label{SecMH}

This section provides an overview of the module design. Modules are summarized
in a hierarchy decomposed by secrets in Table \ref{TblMH}. The modules listed
below, which are leaves in the hierarchy tree, are the modules that will
actually be implemented.

\begin{description}
\item [\refstepcounter{mnum} \mthemnum \label{mHH}:] Hardware-Hiding Module
\item [\refstepcounter{mnum} \mthemnum \label{mNLP}:] NLP Query Processor Module
\item [\refstepcounter{mnum} \mthemnum \label{mUI}:] Front-End Interface Module
\item [\refstepcounter{mnum} \mthemnum \label{mAPI}:] API Layer Module
\item [\refstepcounter{mnum} \mthemnum \label{mDSS}:] Data Schema and Storage Module
\item [\refstepcounter{mnum} \mthemnum \label{mDV}:] Data Visualization Module
\item [\refstepcounter{mnum} \mthemnum \label{mDPP}:] Data Processing Pipeline Module
\item [\refstepcounter{mnum} \mthemnum \label{mAAC}:] Authentication and Access Control Module
\item [\refstepcounter{mnum} \mthemnum \label{mFEM}:] Fault and Error Management Module
\item [\refstepcounter{mnum} \mthemnum \label{mSDM}:] Security and Dependency Management Module
\end{description}


\begin{table}[h!]
\centering
\begin{tabular}{p{0.3\textwidth} p{0.6\textwidth}}
\toprule
\textbf{Level 1} & \textbf{Level 2}\\
\midrule

{Hardware-Hiding Module} & ~ \\
\midrule

\multirow{3}{0.3\textwidth}{Behaviour-Hiding Module} & Front-End Interface Module\\
& API Layer Module\\
& Data Schema and Storage Module\\
& Data Visualization Module\\
\midrule

\multirow{4}{0.3\textwidth}{Software Decision Module} & NLP Query Processor \\
& Data Processing Pipeline Module\\
& Authentication and Access Control Module\\
& Fault and Error Management Module\\
& Security and Dependency Management Module\\
\bottomrule

\end{tabular}
\caption{Module Hierarchy}
\label{TblMH}
\end{table}

\section{Connection Between Requirements and Design}

The design of the behavioral neuroscience data analysis platform was guided directly by the requirements defined in the Software Requirements Specification (SRS). Each design decision was made to ensure that researchers can efficiently access, explore, and analyze complex spatial-temporal behavioral datasets through a user-friendly web interface. The system design is modular, allowing for clear traceability between requirements and their implementation in specific architectural components. Table~\ref{tab:req_design} outlines the connection between key requirements and the corresponding design modules.

\begin{table}[H]
\centering
\caption{Connection Between Requirements and Design}
\label{tab:req_design}
\begin{tabular}{|p{4cm}|p{4cm}|p{6cm}|}
\hline
\textbf{Requirement ID} & \textbf{Design Module} & \textbf{Design Decision / Rationale} \\ \hline
FUNC.R.1 -- Store and manage behavioral datasets & Database Design Module & Implemented a PostgreSQL schema supporting spatial-temporal data, video references, and metadata to ensure efficient storage and retrieval. \\ \hline
FUNC.R.2 -- Query and filter trials & API Layer (FastAPI) & Designed REST API endpoints to allow parameterized and natural language-based querying of trial data. \\ \hline
FUNC.R.3 -- Visualize trajectories and metrics & Data Visualization Module & Built React components using visualization libraries to render spatial trajectories, behavioral heatmaps, and statistical summaries. \\ \hline
APP.R.1 -- Enable intuitive researcher interaction & Frontend Interface (React) & Designed accessible interfaces for non-technical users, emphasizing usability, clarity, and responsive design. \\ \hline
APP.R.2 -- Natural language querying & NLP Integration Layer & Integrated a language model API to interpret user queries and translate them into structured data retrieval commands. \\ \hline
SYS.R.1 -- Scalable and performant system & Backend Infrastructure Module & Implemented Redis caching and efficient database indexing to support large dataset queries and concurrent users. \\ \hline
SYS.R.2 -- Secure and reliable data access & Authentication and Access Control Module & Applied user authentication, HTTPS protocols, and access management to protect research data integrity. \\ \hline
SYS.R.3 -- Extensible architecture for future analysis tools & Processing Pipeline Module & Designed modular Python-based pipelines for processing coordinate data and computing behavioral measures, allowing future algorithm integration. \\ \hline
\end{tabular}
\end{table}

The table above illustrates how the high-level design maps to the system’s defined requirements. For example, the \textit{Database Design Module} directly supports the need for structured data management, while the \textit{Video Integration Module} fulfills the requirement to visualize behavioral trajectories alongside synchronized recordings. The decision to use a modular web architecture (FastAPI + React) ensures scalability and maintainability, while also providing researchers with a seamless, interactive environment for exploring and analyzing OCD behavioral datasets.


The table above explicitly documents the connections between system requirements and their corresponding design modules. Each mapping illustrates how specific architectural decisions were made to fulfill the intended functionality, performance, and reliability objectives. For example, to satisfy data security requirements, MinIO access control policies were implemented; to support performance goals, the data conversion pipeline was parallelized; and to meet modularity and maintainability goals, the system was containerized using Docker.

The design of the system is intended to satisfy the requirements developed in
the SRS. In this stage, the system is decomposed into modules. The connection
between requirements and modules is listed in Table~\ref{TblRT}.

\wss{The intention of this section is to document decisions that are made
  ``between'' the requirements and the design.  To satisfy some requirements,
  design decisions need to be made.  Rather than make these decisions implicit,
  they are explicitly recorded here.  For instance, if a program has security
  requirements, a specific design decision may be made to satisfy those
  requirements with a password.}

\section{Module Decomposition} \label{SecMD}

REMOVE********
Modules are decomposed according to the principle of ``information hiding''
proposed by \citet{ParnasEtAl1984}. The \emph{Secrets} field in a module
decomposition is a brief statement of the design decision hidden by the
module. The \emph{Services} field specifies \emph{what} the module will do
without documenting \emph{how} to do it. For each module, a suggestion for the
implementing software is given under the \emph{Implemented By} title. If the
entry is \emph{OS}, this means that the module is provided by the operating
system or by standard programming language libraries.  \emph{\progname{}} means the
module will be implemented by the \progname{} software.

Only the leaf modules in the hierarchy have to be implemented. If a dash
(\emph{--}) is shown, this means that the module is not a leaf and will not have
to be implemented.
*********
\subsection{Hardware Hiding Modules (\mref{mHH})}

\begin{description}
\item[Secrets:] The data storage and computing environment configurations used
  to host the backend services, databases, and frontend web application.
\item[Services:] Provides the physical or virtual computing infrastructure for
  data storage, computation, and web serving. This includes the local machines,
  cloud servers, and Docker containers used to deploy the system.
\item[Implemented By:] OS
\end{description}

\subsection{Behaviour-Hiding Module}

\begin{description}
\item[Secrets:] The high-level system behaviours that define how the platform
  responds to user interactions and fulfills functional requirements.
\item[Services:] Includes all externally visible system behaviours as specified
  in the Software Requirements Specification (SRS). This layer coordinates
  between the hardware-hiding and software decision modules, providing user
  interactions, API responses, and visualization features. Changes to this
  module are driven by changes in user or research requirements.
\item[Implemented By:] --
\end{description}

\subsubsection{Input Format Module (\mref{mInput})}

\begin{description}
\item[Secrets:] The structure, validation rules, and parsing strategy of the
  input behavioural datasets and metadata.
\item[Services:] Converts raw spatial-temporal rat behavioural data, metadata
  files, and video references into standardized records ready for ingestion
  into the database.
\item[Implemented By:] \progname{}
\item[Type of Module:] Abstract Data Type
\end{description}

\subsubsection{Data Ingestion Module (\mref{mIngest})}

\begin{description}
\item[Secrets:] The logic and pipeline configuration for batch data upload and
  registration.
\item[Services:] Imports, validates, and indexes datasets into the PostgreSQL
  database, ensuring consistency and schema compliance.
\item[Implemented By:] \progname{}
\item[Type of Module:] Library
\end{description}

\subsubsection{API and Query Module (\mref{mAPI})}

\begin{description}
\item[Secrets:] The API routing, endpoint structure, and query optimization
  techniques.
\item[Services:] Provides REST endpoints that expose system functionality to
  external clients and the frontend interface. Supports both structured and
  natural language-based data queries.
\item[Implemented By:] FastAPI
\item[Type of Module:] Library
\end{description}

\subsubsection{Natural Language Processing Module (\mref{mNLP})}

\begin{description}
\item[Secrets:] Query interpretation logic and language model integration.
\item[Services:] Translates human-language queries (e.g., ``find trials with
  checking behaviour after 5 injections'') into structured database queries.
\item[Implemented By:] \progname{}
\item[Type of Module:] Abstract Object
\end{description}


\subsubsection{User Interface Module (\mref{mUI})}

\begin{description}
\item[Secrets:] Frontend state management, routing, and layout structure.
\item[Services:] Provides the graphical interface for researchers to search,
  filter, visualize, and download data without coding knowledge.
\item[Implemented By:] React
\item[Type of Module:] Abstract Object
\end{description}

\subsubsection{Authentication and Access Control Module (\mref{mAuth})}

\begin{description}
\item[Secrets:] User authentication logic and credential encryption.
\item[Services:] Manages user sessions, API key access, and role-based
  permissions to ensure secure and compliant data access.
\item[Implemented By:] FastAPI Security Layer
\item[Type of Module:] Library
\end{description}

\subsection{Software Decision Module}

\begin{description}
\item[Secrets:] Internal design decisions for optimization, scalability,
  fault tolerance, and data processing efficiency. These are not visible in the
  SRS but are critical for performance and maintainability.
\item[Services:] Provides the algorithms, data structures, and optimization
  techniques that support system reliability, responsiveness, and scalability.
\item[Implemented By:] --
\end{description}

\subsubsection{Data Processing and Analysis Module (\mref{mProc})}

\begin{description}
\item[Secrets:] Algorithms for behavioural pattern recognition, trajectory
  metrics, and exploratory data analysis.
\item[Services:] Processes raw coordinate data to compute behavioural metrics
  such as exploration duration, home-base frequency, and checking behaviour.
\item[Implemented By:] Python Scientific Stack (NumPy, Pandas)
\item[Type of Module:] Library
\end{description}

\subsubsection{Database Management Module (\mref{mDB})}

\begin{description}
\item[Secrets:] Schema design, indexing strategy, and query execution plan.
\item[Services:] Stores and retrieves behavioural data, metadata, and video
  references efficiently using a relational model.
\item[Implemented By:] PostgreSQL
\item[Type of Module:] Record
\end{description}

\subsubsection{Deployment and Configuration Module (\mref{mDeploy})}

\begin{description}
\item[Secrets:] Container configuration, environment variable structure, and
  network orchestration details.
\item[Services:] Defines reproducible deployment processes using containerized
  environments for backend, frontend, and database components.
\item[Implemented By:] Docker / Kubernetes
\item[Type of Module:] Library
\end{description}

\subsubsection{Logging and Fault Tolerance Module (\mref{mLog})}

\begin{description}
\item[Secrets:] Internal error-handling and monitoring mechanisms.
\item[Services:] Collects logs, detects faults, and enables system recovery
  during execution failures or unexpected input conditions.
\item[Implemented By:] \progname{}
\item[Type of Module:] Library
\end{description}


\section{Traceability Matrix} \label{SecTM}

This section shows two traceability matrices: between the modules and the
requirements and between the modules and the anticipated changes.

% the table should use mref, the requirements should be named, use something
% like fref
\begin{table}[H]
\centering
\begin{tabular}{p{0.2\textwidth} p{0.6\textwidth}}
\toprule
\textbf{Req.} & \textbf{Modules}\\
\midrule
R1 & \mref{mHH}, \mref{mInput}, \mref{mParams}, \mref{mControl}\\
R2 & \mref{mInput}, \mref{mParams}\\
R3 & \mref{mVerify}\\
R4 & \mref{mOutput}, \mref{mControl}\\
R5 & \mref{mOutput}, \mref{mODEs}, \mref{mControl}, \mref{mSeqDS}, \mref{mSolver}, \mref{mPlot}\\
R6 & \mref{mOutput}, \mref{mODEs}, \mref{mControl}, \mref{mSeqDS}, \mref{mSolver}, \mref{mPlot}\\
R7 & \mref{mOutput}, \mref{mEnergy}, \mref{mControl}, \mref{mSeqDS}, \mref{mPlot}\\
R8 & \mref{mOutput}, \mref{mEnergy}, \mref{mControl}, \mref{mSeqDS}, \mref{mPlot}\\
R9 & \mref{mVerifyOut}\\
R10 & \mref{mOutput}, \mref{mODEs}, \mref{mControl}\\
R11 & \mref{mOutput}, \mref{mODEs}, \mref{mEnergy}, \mref{mControl}\\
\bottomrule
\end{tabular}
\caption{Trace Between Requirements and Modules}
\label{TblRT}
\end{table}

\begin{table}[H]
\centering
\begin{tabular}{p{0.2\textwidth} p{0.6\textwidth}}
\toprule
\textbf{AC} & \textbf{Modules}\\
\midrule
\acref{acHardware} & \mref{mHH}\\
\acref{acInput} & \mref{mInput}\\
\acref{acParams} & \mref{mParams}\\
\acref{acVerify} & \mref{mVerify}\\
\acref{acOutput} & \mref{mOutput}\\
\acref{acVerifyOut} & \mref{mVerifyOut}\\
\acref{acODEs} & \mref{mODEs}\\
\acref{acEnergy} & \mref{mEnergy}\\
\acref{acControl} & \mref{mControl}\\
\acref{acSeqDS} & \mref{mSeqDS}\\
\acref{acSolver} & \mref{mSolver}\\
\acref{acPlot} & \mref{mPlot}\\
\bottomrule
\end{tabular}
\caption{Trace Between Anticipated Changes and Modules}
\label{TblACT}
\end{table}

\section{Use Hierarchy Between Modules} \label{SecUse}

In this section, the uses hierarchy between modules is
provided. \citet{Parnas1978} said of two programs A and B that A {\em uses} B if
correct execution of B may be necessary for A to complete the task described in
its specification. That is, A {\em uses} B if there exist situations in which
the correct functioning of A depends upon the availability of a correct
implementation of B.  Figure \ref{FigUH} illustrates the use relation between
the modules. It can be seen that the graph is a directed acyclic graph
(DAG). Each level of the hierarchy offers a testable and usable subset of the
system, and modules in the higher level of the hierarchy are essentially simpler
because they use modules from the lower levels.

\wss{The uses relation is not a data flow diagram.  In the code there will often
be an import statement in module A when it directly uses module B.  Module B
provides the services that module A needs.  The code for module A needs to be
able to see these services (hence the import statement).  Since the uses
relation is transitive, there is a use relation without an import, but the
arrows in the diagram typically correspond to the presence of import statement.}

\wss{If module A uses module B, the arrow is directed from A to B.}

\begin{figure}[H]
\centering
%\includegraphics[width=0.7\textwidth]{UsesHierarchy.png}
\caption{Use hierarchy among modules}
\label{FigUH}
\end{figure}

%\section*{References}

\section{User Interfaces}

\wss{Design of user interface for software and hardware.  Attach an appendix if
needed. Drawings, Sketches, Figma}

\section{Design of Communication Protocols}

\wss{If appropriate}

\section{Timeline}

The project timeline outlines the major milestones, deliverables, and responsibilities
for each development phase. Progress, task assignments, and discussions are tracked
publicly through the project's GitHub repository at:

\begin{center}
\url{https://github.com/OCD-Rats-Capstone/OCD-Rat-Infrastructure/issues}
\end{center}

This repository serves as the primary project management platform, where issues correspond
to individual tasks or features, each labeled with priority, category, and assignee.

Table~\ref{tab:timeline} summarizes the high-level schedule of the project, organized
by development phase and major deliverables.

\begin{table}[h!]
\centering
\caption{Project Timeline and Responsibilities}
\label{tab:timeline}
\begin{tabular}{|p{2.5cm}|p{5.5cm}|p{3.5cm}|p{2cm}|}
\hline
\textbf{Phase} & \textbf{Major Tasks / Deliverables} & \textbf{Responsible Members} & \textbf{Duration} \\ \hline
\textbf{Week 1--2} & Project setup, environment configuration, and repository initialization. Define database schema draft. & Backend + Database Teams & 2 weeks \\ \hline
\textbf{Week 3--5} & Develop and test PostgreSQL schema. Implement initial API endpoints for data ingestion and retrieval. & Backend Team & 3 weeks \\ \hline
\textbf{Week 6--8} & Implement frontend React interface for search and filtering. Integrate with backend API endpoints. & Frontend Team & 3 weeks \\ \hline
\textbf{Week 9--10} & Implement data visualization and synchronized video playback components. & Frontend + Visualization Teams & 2 weeks \\ \hline
\textbf{Week 11--12} & Integrate NLP query functionality and improve search capabilities. & NLP + Backend Teams & 2 weeks \\ \hline
\textbf{Week 13--14} & Conduct system testing, debugging, and performance optimization. Prepare user documentation. & All Members & 2 weeks \\ \hline
\textbf{Week 15} & Final deployment, demonstration, and report submission. & Entire Team & 1 week \\ \hline
\end{tabular}
\end{table}

Ongoing updates, issue tracking, and future work will continue to be maintained through
GitHub to ensure transparency and collaboration with supervisors and contributors.

\bibliographystyle {plainnat}
\bibliography{../../../refs/References}

\newpage{}

\end{document}
