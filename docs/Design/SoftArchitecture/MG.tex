\documentclass[12pt, titlepage]{article}

\usepackage{fullpage}
\usepackage{longtable}
\usepackage[round]{natbib}
\usepackage{multirow}
\usepackage{booktabs}
\usepackage{tabularx}
\usepackage{graphicx}
\usepackage{float}
\usepackage{hyperref}
\hypersetup{
    colorlinks,
    citecolor=blue,
    filecolor=black,
    linkcolor=red,
    urlcolor=blue
}

\input{../../Comments}
%% Common Parts

\newcommand{\progname}{Software Engineering} % PUT YOUR PROGRAM NAME HERE
\newcommand{\authname}{Team \#18, Gouda Engineers 
\\ Aidan Goodyer
\\ Jeremy Orr
\\ Leo Vugert
\\ Nathan Perry
\\ Tim Pokanai} % AUTHOR NAMES                  

\usepackage{hyperref}
    \hypersetup{colorlinks=true, linkcolor=blue, citecolor=blue, filecolor=blue,
                urlcolor=blue, unicode=false}
    \urlstyle{same}
                                


\newcounter{acnum}
\newcommand{\actheacnum}{AC\theacnum}
\newcommand{\acref}[1]{AC\ref{#1}}

\newcounter{ucnum}
\newcommand{\uctheucnum}{UC\theucnum}
\newcommand{\uref}[1]{UC\ref{#1}}

\newcounter{mnum}
\newcommand{\mthemnum}{M\themnum}
\newcommand{\mref}[1]{M\ref{#1}}

\begin{document}

\title{Module Guide for \progname{}} 
\author{\authname}
\date{\today}

\maketitle

\pagenumbering{roman}

\section{Revision History}

\begin{tabularx}{\textwidth}{p{3cm}p{2cm}X}
\toprule {\bf Date} & {\bf Version} & {\bf Notes}\\
\midrule
Date 1 & 1.0 & Notes\\
Date 2 & 1.1 & Notes\\
\bottomrule
\end{tabularx}

\newpage

\section{Reference Material}

This section records information for easy reference.

\subsection{Abbreviations and Acronyms}

\renewcommand{\arraystretch}{1.2}
\begin{longtable}{l l}
  \toprule
  \textbf{Symbol / Abbreviation} & \textbf{Description} \\
  \midrule
  \endfirsthead

  \toprule
  \textbf{Symbol / Abbreviation} & \textbf{Description} \\
  \midrule
  \endhead

  \bottomrule
  \endfoot

  \bottomrule
  \endlastfoot

  % --- Anticipated and General ---
  AC & Anticipated Change \\
  API & Application Programming Interface \\
  APP & Appearance Requirement \\
  CAP & Capacity Requirement \\
  DPP & Data Processing Pipeline (Module) \\
  DSS & Data Schema and Storage (Module) \\
  DV & Data Visualization (Module) \\
  EOU & Ease of Use Requirement \\
  EPE & Expected Physical Environment Requirement \\
  FEM & Fault and Error Management (Module) \\
  FR & Functional Requirement \\
  FRDR & Federated Research Data Repository \\
  FUNC & Functional Requirement (Prefix) \\
  HH & Hardware-Hiding (Module) \\
  HTTPS & Hypertext Transfer Protocol Secure \\
  IMM & Immunity Requirement \\
  INT & Integrity Requirement \\
  IWAS & Interfacing with Adjacent Systems Requirement \\
  JSON & JavaScript Object Notation \\
  LEA & Learning Requirement \\
  LEG & Legal Requirement \\
  LON & Longevity Requirement \\
  LLM & Large Language Model \\
  M & Module \\
  MAI & Maintenance Requirement \\
  MG & Module Guide \\
  NLP & Natural Language Processing \\
  NLPQP & NLP Query Processor (Module) \\
  NFR & Non-Functional Requirement \\
  OCD & Obsessive-Compulsive Disorder \\
  OI & Open Issue \\
  OS & Operating System \\
  POA & Precision or Accuracy Requirement \\
  PROD & Productization Requirement \\
  R & Requirement \\
  REL & Release Requirement \\
  REST & Representational State Transfer \\
  ROFT & Robustness or Fault Tolerance Requirement \\
  SAL & Speed and Latency Requirement \\
  SDM & Security and Dependency Management (Module) \\
  SOE & Scalability or Extensibility Requirement \\
  SRS & Software Requirements Specification \\
  SUP & Supportability Requirement \\
  UAP & Understandability and Politeness Requirement \\
  UC & Unlikely Change \\
  UD & User Documentation Requirement \\
  UI & Front-End Interface (Module) \\
  WE & Wider Environment Requirement \\
  WR & Waiting Room (Deferred Requirement) \\
  AAC & Authentication and Access Control (Module) \\
  ADA & Adaptability Requirement \\
  ACC & Accessibility Requirement \\
  JSON API & Structured data exchange format over HTTP(S) \\

\end{longtable}


\newpage

\tableofcontents

\listoftables

\listoffigures

\newpage

\pagenumbering{arabic}

\section{Introduction}

Decomposing a system into modules is a commonly accepted approach to developing
software.  A module is a work assignment for a programmer or programming
team~\citep{ParnasEtAl1984}.  We advocate a decomposition
based on the principle of information hiding~\citep{Parnas1972a}.  This
principle supports design for change, because the ``secrets'' that each module
hides represent likely future changes.  Design for change is valuable in SC,
where modifications are frequent, especially during initial development as the
solution space is explored.  

Our design follows the rules layed out by \citet{ParnasEtAl1984}, as follows:
\begin{itemize}
\item System details that are likely to change independently should be the
  secrets of separate modules.
\item Each data structure is implemented in only one module.
\item Any other program that requires information stored in a module's data
  structures must obtain it by calling access programs belonging to that module.
\end{itemize}

After completing the first stage of the design, the Software Requirements
Specification (SRS), the Module Guide (MG) is developed~\citep{ParnasEtAl1984}. The MG
specifies the modular structure of the system and is intended to allow both
designers and maintainers to easily identify the parts of the software.  The
potential readers of this document are as follows:

\begin{itemize}
\item New project members: This document can be a guide for a new project member
  to easily understand the overall structure and quickly find the
  relevant modules they are searching for.
\item Maintainers: The hierarchical structure of the module guide improves the
  maintainers' understanding when they need to make changes to the system. It is
  important for a maintainer to update the relevant sections of the document
  after changes have been made.
\item Designers: Once the module guide has been written, it can be used to
  check for consistency, feasibility, and flexibility. Designers can verify the
  system in various ways, such as consistency among modules, feasibility of the
  decomposition, and flexibility of the design.
\end{itemize}

The rest of the document is organized as follows. Section
\ref{SecChange} lists the anticipated and unlikely changes of the software
requirements. Section \ref{SecMH} summarizes the module decomposition that
was constructed according to the likely changes. Section \ref{SecConnection}
specifies the connections between the software requirements and the
modules. Section \ref{SecMD} gives a detailed description of the
modules. Section \ref{SecTM} includes two traceability matrices. One checks
the completeness of the design against the requirements provided in the SRS. The
other shows the relation between anticipated changes and the modules. Section
\ref{SecUse} describes the use relation between modules.

\section{Anticipated and Unlikely Changes} \label{SecChange}

This section lists possible changes to the system. According to the likeliness
of the change, the possible changes are classified into two
categories. Anticipated changes are listed in Section \ref{SecAchange}, and
unlikely changes are listed in Section \ref{SecUchange}.

\subsection{Anticipated Changes} \label{SecAchange}

Anticipated changes are the source of the information that is to be hidden
inside the modules. Ideally, changing one of the anticipated changes will only
require changing the one module that hides the associated decision. The approach
adapted here is called design for
change.

\begin{description}
\item[\refstepcounter{acnum} \actheacnum \label{acUIDesign}:] The layout, accessibility, 
  and responsiveness of the front-end interface.
\item[\refstepcounter{acnum} \actheacnum \label{acQueryLLM}:] Fine-tuning or using a 
  different query NLP Model.
\item[\refstepcounter{acnum} \actheacnum \label{acFiltersAndQueryParameters}:] The constraints of filter and query 
  metadata field inputs.
\item[\refstepcounter{acnum} \actheacnum \label{acDataVisualization}:] The framework used for 
  creating visualizations from data and presenting them.
\item[\refstepcounter{acnum} \actheacnum \label{acProcessingAlgorithms}:] The algorithms implemented for 
  behavioural analysis or classification.
\item[\refstepcounter{acnum} \actheacnum \label{acDatasetHosting}:] The virtual hosting location of 
  Dr. Szechtman's dataset.
\item[\refstepcounter{acnum} \actheacnum \label{acAccessControl}:] The implementation of the user authentication 
  and access control mechanism for scalability.
\item[\refstepcounter{acnum} \actheacnum \label{acErrorAndFaultHandling}:] The implementation of 
  error detection and fault recovery mechanisms.
\item[\refstepcounter{acnum} \actheacnum \label{acSecurityAndDependencyUpdates}:] The implementation of 
  application security and dependency monitoring strategies.
\end{description}

\subsection{Unlikely Changes} \label{SecUchange}

The module design should be as general as possible. However, a general system is
more complex. Sometimes this complexity is not necessary. Fixing some design
decisions at the system architecture stage can simplify the software design. If
these decision should later need to be changed, then many parts of the design
will potentially need to be modified. Hence, it is not intended that these
decisions will be changed.

\begin{description}
\item[\refstepcounter{ucnum} \uctheucnum \label{ucReadOnlyDataset}:] The read-only access 
  to Dr. Szechtman's dataset.
\item[\refstepcounter{ucnum} \uctheucnum \label{ucSoleDataSource}:] The dataset hosted on FRDR 
  being our primary and individual data source.
\item[\refstepcounter{ucnum} \uctheucnum \label{ucDataProcessingScope}:] Scope of data 
  processing analysis (Does not generalize rat behaviour trial data to other animal models 
  or domains).
\item[\refstepcounter{ucnum} \uctheucnum \label{ucMVCDesignPattern}:] The software application's MVC design pattern.
\item[\refstepcounter{ucnum} \uctheucnum \label{ucIO}:] Input/Output devices
  (Input: File and/or Keyboard, Output: File, Memory, and/or Screen).
\item[\refstepcounter{ucnum} \uctheucnum \label{ucUserComputingDevice}:] User's computing device 
  (The assumption that the user's device will support modern browsers with Javascript enabled, 
  will not change)
\end{description}

\section{Module Hierarchy} \label{SecMH}

This section provides an overview of the module design. Modules are summarized
in a hierarchy decomposed by secrets in Table \ref{TblMH}. The modules listed
below, which are leaves in the hierarchy tree, are the modules that will
actually be implemented.

\begin{description}
\item [\refstepcounter{mnum} \mthemnum \label{mHH}:] Hardware-Hiding Module
\item [\refstepcounter{mnum} \mthemnum \label{mNLP}:] NLP Query Processor Module
\item [\refstepcounter{mnum} \mthemnum \label{mUI}:] Front-End Interface Module
\item [\refstepcounter{mnum} \mthemnum \label{mAPI}:] API Layer Module
\item [\refstepcounter{mnum} \mthemnum \label{mDSS}:] Data Schema and Storage Module
\item [\refstepcounter{mnum} \mthemnum \label{mDV}:] Data Visualization Module
\item [\refstepcounter{mnum} \mthemnum \label{mDPP}:] Data Processing Pipeline Module
\item [\refstepcounter{mnum} \mthemnum \label{mAAC}:] Authentication and Access Control Module
\item [\refstepcounter{mnum} \mthemnum \label{mFEM}:] Fault and Error Management Module
\end{description}


\begin{table}[h!]
\centering
\begin{tabular}{p{0.3\textwidth} p{0.6\textwidth}}
\toprule
\textbf{Level 1} & \textbf{Level 2}\\
\midrule

{Hardware-Hiding Module} & ~ \\
\midrule

\multirow{3}{0.3\textwidth}{Behaviour-Hiding Module} & Front-End Interface Module\\
& API Layer Module\\
& Data Schema and Storage Module\\
& Data Visualization Module\\
\midrule

\multirow{3}{0.3\textwidth}{Software Decision Module} & NLP Query Processor \\
& Data Processing Pipeline Module\\
& Authentication and Access Control Module\\
& Fault and Error Management Module\\
\bottomrule

\end{tabular}
\caption{Module Hierarchy}
\label{TblMH}
\end{table}

\section{Connection Between Requirements and Design} \label{SecReqDesign}

The design of the behavioral neuroscience data analysis platform is directly driven by the requirements specified in the SRS. Each module in the hierarchy (\autoref{SecMH}) was designed to fulfill specific functional, application, or system-level requirements. This section documents the decisions made to satisfy requirements and ensures traceability from requirements to design modules.
Table~\ref{tab:req_design_func} and Table~\ref{tab:req_design_sys} map key requirements to the corresponding design modules, explaining the rationale behind each decision.

\begin{table}[H]
\centering
\small
\caption{Mapping Between Functional and Application Requirements and Design Modules}
\label{tab:req_design_func}
\begin{tabularx}{\textwidth}{|p{2.8cm}|p{2.8cm}|X|}
\hline
\textbf{Requirement ID} & \textbf{Design Module(s)} & \textbf{Design Decision / Rationale} \\ \hline
FUNC.R.1 -- Store and manage behavioral datasets & M5: Data Schema and Storage Module & Implemented a PostgreSQL-based schema supporting spatial-temporal data, video references, and metadata to ensure efficient and reliable data storage and retrieval. \\ \hline
FUNC.R.2 -- Query and filter trials & M4: API Layer Module, M2: NLP Query Processor Module & REST API endpoints handle structured queries, while the NLP processor translates natural language requests into database queries for user-friendly data access. \\ \hline
APP.R.1 -- Enable intuitive researcher interaction & M3: Front-End Interface Module & Provides a responsive, accessible web interface for researchers to explore, filter, and download datasets without needing technical expertise. \\ \hline
APP.R.2 -- Support natural language querying & M2: NLP Query Processor Module & Employs NLP models to interpret user questions and map them to database operations for seamless data exploration. \\ \hline
\end{tabularx}
\end{table}

\begin{table}[H]
\centering
\small
\caption{Mapping Between System Requirements and Design Modules (continued)}
\label{tab:req_design_sys}
\begin{tabularx}{\textwidth}{|p{2.8cm}|p{2.8cm}|X|}
\hline
\textbf{Requirement ID} & \textbf{Design Module(s)} & \textbf{Design Decision / Rationale} \\ \hline
SYS.R.1 -- Scalable and performant system & M7: Data Processing Pipeline Module & Uses distributed data processing, efficient indexing, and caching mechanisms to handle large-scale datasets and concurrent requests. \\ \hline
SYS.R.2 -- Secure and reliable data access & M8: Authentication and Access Control Module, M10: Security and Dependency Management Module & Implements user authentication, role-based access, encryption, and dependency auditing to ensure data integrity and confidentiality. \\ \hline
SYS.R.3 -- Fault tolerance and error handling & M9: Fault and Error Management Module & Provides logging, monitoring, and recovery mechanisms to detect, report, and mitigate system faults or data inconsistencies. \\ \hline
SYS.R.4 -- Extensible architecture for future analysis tools & M1: Hardware-Hiding Module, M4: API Layer Module & Hardware abstraction and modular API interfaces support integration of new algorithms, sensors, and external analysis tools without disrupting core functionality. \\ \hline
\end{tabularx}
\end{table}


This mapping makes the relationship between requirements and design explicit, demonstrating how each architectural decision satisfies specific functional, performance, and security objectives.

\section{Module Decomposition} \label{SecMD}

The system is decomposed according to principles of information hiding \citep{ParnasEtAl1984}. Each module is defined by its \textit{Secrets} (hidden design decisions), \textit{Services} (exposed functionality), and the software or technology used for implementation. Only leaf modules in the hierarchy (\autoref{SecMH}) are directly implemented.

\subsection{Hardware-Hiding Module (\mref{mHH})}

\begin{description}
\item[Secrets:] Configuration of computing infrastructure including backend servers, databases, and frontend deployment environments.
\item[Services:] Provides physical and virtual infrastructure for computation, storage, and web services.
\item[Implemented By:] OS / Cloud Infrastructure
\end{description}

\subsection{Behaviour-Hiding Modules}

\subsubsection{Front-End Interface Module (\mref{mUI})}

\begin{description}
\item[Secrets:] Frontend state management, routing, and layout design.
\item[Services:] User interface for dataset search, visualization, filtering, and download.
\item[Implemented By:] React
\end{description}

\subsubsection{API Layer Module (\mref{mAPI})}

\begin{description}
\item[Secrets:] API routing, endpoint definition, query optimization.
\item[Services:] Provides REST API for structured and natural language queries.
\item[Implemented By:] FastAPI
\end{description}

\subsubsection{NLP Query Processor Module (\mref{mNLP})}

\begin{description}
\item[Secrets:] Query interpretation logic and integration with language model APIs.
\item[Services:] Converts natural language queries into structured database commands.
\item[Implemented By:] Custom Python code
\end{description}

\subsubsection{Authentication and Access Control Module (\mref{mAAC})}

\begin{description}
\item[Secrets:] User authentication and role-based access control.
\item[Services:] Manages secure access to data and APIs.
\item[Implemented By:] FastAPI Security Layer
\end{description}

\subsection{Software Decision Modules}

\subsubsection{Data Processing Pipeline Module (\mref{mDPP})}

\begin{description}
\item[Secrets:] Data processing algorithms and pipeline configuration for efficiency and scalability.
\item[Services:] Processes behavioral datasets to compute metrics and prepare data for visualization.
\item[Implemented By:] Python (NumPy, Pandas)
\end{description}

\subsubsection{Data Schema and Storage Module (\mref{mDSS})}

\begin{description}
\item[Secrets:] Database schema, indexing strategies, and query optimization.
\item[Services:] Efficient storage and retrieval of behavioral datasets and metadata.
\item[Implemented By:] PostgreSQL
\end{description}

\subsubsection{Data Visualization Module (\mref{mDV})}

\begin{description}
\item[Secrets:] Visualization design and interactive plotting logic.
\item[Services:] Generates charts, trajectories, and heatmaps for research analysis.
\item[Implemented By:] React + Charting Libraries
\end{description}

\subsubsection{Fault and Error Management Module (\mref{mFEM})}

\begin{description}
\item[Secrets:] Error detection, logging, and recovery mechanisms.
\item[Services:] Ensures robustness and reliability during runtime exceptions or unexpected input.
\item[Implemented By:] Python logging / monitoring frameworks
\end{description}

\subsubsection{Security and Dependency Management Module (\mref{mSDM})}

\begin{description}
\item[Secrets:] Internal security protocols, dependency versions, and update management.
\item[Services:] Maintains system integrity, reduces vulnerabilities, and ensures reproducibility.
\item[Implemented By:] OS / Package Management Tools
\end{description}


\section{Traceability Matrix} \label{SecTM}

This section shows two traceability matrices: between the modules and the
requirements and between the modules and the anticipated changes.

% the table should use mref, the requirements should be named, use something
% like fref
\begin{table}[H]
\centering
\begin{tabular}{p{0.2\textwidth} p{0.6\textwidth}}
\toprule
\textbf{Req.} & \textbf{Modules}\\
\midrule
R1 & \mref{mHH}, \mref{mInput}, \mref{mParams}, \mref{mControl}\\
R2 & \mref{mInput}, \mref{mParams}\\
R3 & \mref{mVerify}\\
R4 & \mref{mOutput}, \mref{mControl}\\
R5 & \mref{mOutput}, \mref{mODEs}, \mref{mControl}, \mref{mSeqDS}, \mref{mSolver}, \mref{mPlot}\\
R6 & \mref{mOutput}, \mref{mODEs}, \mref{mControl}, \mref{mSeqDS}, \mref{mSolver}, \mref{mPlot}\\
R7 & \mref{mOutput}, \mref{mEnergy}, \mref{mControl}, \mref{mSeqDS}, \mref{mPlot}\\
R8 & \mref{mOutput}, \mref{mEnergy}, \mref{mControl}, \mref{mSeqDS}, \mref{mPlot}\\
R9 & \mref{mVerifyOut}\\
R10 & \mref{mOutput}, \mref{mODEs}, \mref{mControl}\\
R11 & \mref{mOutput}, \mref{mODEs}, \mref{mEnergy}, \mref{mControl}\\
\bottomrule
\end{tabular}
\caption{Trace Between Requirements and Modules}
\label{TblRT}
\end{table}

\begin{table}[H]
\centering
\begin{tabular}{p{0.2\textwidth} p{0.6\textwidth}}
\toprule
\textbf{AC} & \textbf{Modules}\\
\midrule
\acref{acHardware} & \mref{mHH}\\
\acref{acInput} & \mref{mInput}\\
\acref{acParams} & \mref{mParams}\\
\acref{acVerify} & \mref{mVerify}\\
\acref{acOutput} & \mref{mOutput}\\
\acref{acVerifyOut} & \mref{mVerifyOut}\\
\acref{acODEs} & \mref{mODEs}\\
\acref{acEnergy} & \mref{mEnergy}\\
\acref{acControl} & \mref{mControl}\\
\acref{acSeqDS} & \mref{mSeqDS}\\
\acref{acSolver} & \mref{mSolver}\\
\acref{acPlot} & \mref{mPlot}\\
\bottomrule
\end{tabular}
\caption{Trace Between Anticipated Changes and Modules}
\label{TblACT}
\end{table}

\section{Use Hierarchy Between Modules} \label{SecUse}

In this section, the uses hierarchy between modules is
provided. \citet{Parnas1978} said of two programs A and B that A {\em uses} B if
correct execution of B may be necessary for A to complete the task described in
its specification. That is, A {\em uses} B if there exist situations in which
the correct functioning of A depends upon the availability of a correct
implementation of B.  Figure \ref{FigUH} illustrates the use relation between
the modules. It can be seen that the graph is a directed acyclic graph
(DAG). Each level of the hierarchy offers a testable and usable subset of the
system, and modules in the higher level of the hierarchy are essentially simpler
because they use modules from the lower levels.

\wss{The uses relation is not a data flow diagram.  In the code there will often
be an import statement in module A when it directly uses module B.  Module B
provides the services that module A needs.  The code for module A needs to be
able to see these services (hence the import statement).  Since the uses
relation is transitive, there is a use relation without an import, but the
arrows in the diagram typically correspond to the presence of import statement.}

\wss{If module A uses module B, the arrow is directed from A to B.}

\begin{figure}[H]
\centering
%\includegraphics[width=0.7\textwidth]{UsesHierarchy.png}
\caption{Use hierarchy among modules}
\label{FigUH}
\end{figure}

%\section*{References}

\section{User Interfaces}

This section presents the main user interface designs for \textbf{RatBat2}, 
it shows the structure and layout of the system’s main pages. The following 
figures show a revised user interface design of the application, developed 
to guide implementation and maintain a consistent user experience across the program.

\begin{figure}[H]
  \centering
  \includegraphics[width=\textwidth]{UIScreenshots/HomepageF1.png}
  \caption{RatBat2 Homepage – Initial Layout Frame 1}
\end{figure}

\begin{figure}[H]
  \centering
  \includegraphics[width=\textwidth]{UIScreenshots/HomepageF2.png}
  \caption{RatBat 2 Homepage – Initial Layout Frame 2}
\end{figure}

\begin{figure}[H]
  \centering
  \includegraphics[width=\textwidth]{UIScreenshots/HomepageF3.png}
  \caption{RatBat2 Homepage – Trial Data Display}
\end{figure}

\begin{figure}[H]
  \centering
  \includegraphics[width=\textwidth]{UIScreenshots/HomepageF4.png}
  \caption{RatBat2 Homepage – NLP Information}
\end{figure}

\begin{figure}[H]
  \centering
  \includegraphics[width=\textwidth]{UIScreenshots/QueryPage.png}
  \caption{Query Page – Search and Filter Interface}
\end{figure}

\begin{figure}[H]
  \centering
  \includegraphics[width=\textwidth]{UIScreenshots/QueryPageTest.png}
  \caption{Query Page – Test of Sample Prompt}
\end{figure}

\begin{figure}[H]
  \centering
  \includegraphics[width=\textwidth]{UIScreenshots/ExperimentPage.png}
  \caption{Experiment Page – Experiment Management and Overview}
\end{figure}

\begin{figure}[H]
  \centering
  \includegraphics[width=\textwidth]{UIScreenshots/AboutPageF1.png}
  \caption{About Page – Team Section}
\end{figure}

\begin{figure}[H]
  \centering
  \includegraphics[width=\textwidth]{UIScreenshots/AboutPageF2.png}
  \caption{About Page – Frequently Asked Questions (FAQ) and Background Information}
\end{figure}

\begin{figure}[H]
  \centering
  \includegraphics[width=\textwidth]{UIScreenshots/ErrorPage.png}
  \caption{Error Page – 404 Not Found Screen}
\end{figure}


\section{Design of Communication Protocols}

The communication protocols define how system modules exchange information and coordinate data flow between the user interface, processing components, and data sources. The goal is to maintain consistent, secure, and modular interactions across all subsystems.

\subsection{Overall Communication Architecture}
All communication follows a client–server model where the Front-End Interface Module (\mref{mUI}) acts as the client and the API Layer Module (\mref{mAPI}) serves as the central hub between client requests and backend modules.  
Data is exchanged in JSON format over HTTPS to ensure interoperability and security.

\subsection{External Communication}
The front-end sends RESTful HTTP(S) requests to the API layer for querying behavioural datasets, submitting NLP-based searches, and requesting visualizations.  
Responses are returned as JSON objects containing results, metadata, or error information.

\subsection{Internal Communication}
The API Layer (\mref{mAPI}) intermediates between modules:
\begin{itemize}
    \item Sends parsed queries to the NLP Query Processor (\mref{mNLP}).
    \item Retrieves structured data from the Data Schema and Storage Module (\mref{mDSS}).
    \item Forwards subsets to the Data Processing Pipeline (\mref{mDPP}) for analysis.
    \item Provides outputs to the Data Visualization Module (\mref{mDV}).
    \item Coordinates with Authentication and Access Control (\mref{mAAC}) for validation.
    \item Works with Fault and Error Management (\mref{mFEM}) for monitoring and recovery.
    \item Integrates with Security and Dependency Management (\mref{mSDM}) to enforce security policies.
\end{itemize}

\subsection{Communication Reliability and Standards}
All communication uses HTTPS/TLS encryption.  
Backend interactions are stateless, enabling independent module deployment.  
Standard HTTP response codes are used for consistency (e.g., 200, 400, 401, 500).  
Asynchronous queues handle long-running tasks to prevent blocking.

\subsection{Scalability and Fault Tolerance}
Modules support horizontal scaling through containerization.  
The Fault and Error Management Module (\mref{mFEM}) provides retry and fallback mechanisms.  
The Hardware-Hiding Module (\mref{mHH}) abstracts hosting details, allowing migration across environments without affecting communication.


\section{Timeline}

The project timeline outlines the major milestones, deliverables, and responsibilities
for each development phase. Progress, task assignments, and discussions are tracked
publicly through the project's GitHub repository at:

\begin{center}
\url{https://github.com/OCD-Rats-Capstone/OCD-Rat-Infrastructure/issues}
\end{center}

This repository serves as the primary project management platform, where issues correspond
to individual tasks or features, each labeled with priority, category, and assignee.

Table~\ref{tab:timeline} summarizes the high-level schedule of the project, organized
by development phase and major deliverables.

\begin{table}[h!]
\centering
\caption{Project Timeline and Responsibilities}
\label{tab:timeline}
\begin{tabular}{|p{2.5cm}|p{5.5cm}|p{3.5cm}|p{2cm}|}
\hline
\textbf{Phase} & \textbf{Major Tasks / Deliverables} & \textbf{Responsible Members} & \textbf{Duration} \\ \hline
\textbf{Week 1--2} & Project setup, environment configuration, and repository initialization. Define database schema draft. & Backend + Database Teams & 2 weeks \\ \hline
\textbf{Week 3--5} & Develop and test PostgreSQL schema. Implement initial API endpoints for data ingestion and retrieval. & Backend Team & 3 weeks \\ \hline
\textbf{Week 6--8} & Implement frontend React interface for search and filtering. Integrate with backend API endpoints. & Frontend Team & 3 weeks \\ \hline
\textbf{Week 9--10} & Implement data visualization and synchronized video playback components. & Frontend + Visualization Teams & 2 weeks \\ \hline
\textbf{Week 11--12} & Integrate NLP query functionality and improve search capabilities. & NLP + Backend Teams & 2 weeks \\ \hline
\textbf{Week 13--14} & Conduct system testing, debugging, and performance optimization. Prepare user documentation. & All Members & 2 weeks \\ \hline
\textbf{Week 15} & Final deployment, demonstration, and report submission. & Entire Team & 1 week \\ \hline
\end{tabular}
\end{table}

Ongoing updates, issue tracking, and future work will continue to be maintained through
GitHub to ensure transparency and collaboration with supervisors and contributors.

\bibliographystyle {plainnat}
\bibliography{../../../refs/References}

\newpage{}

\end{document}
