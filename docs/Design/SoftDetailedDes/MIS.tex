\documentclass[12pt, titlepage]{article}

\usepackage{amsmath, mathtools}

\usepackage[round]{natbib}
\usepackage{amsfonts}
\usepackage{amssymb}
\usepackage{graphicx}
\usepackage{colortbl}
\usepackage{xr}
\usepackage{hyperref}
\usepackage{longtable}
\usepackage{xfrac}
\usepackage{tabularx}
\usepackage{float}
\usepackage{siunitx}
\usepackage{booktabs}
\usepackage{multirow}
\usepackage[section]{placeins}
\usepackage{caption}
\usepackage{fullpage}

\hypersetup{
bookmarks=true,     % show bookmarks bar?
colorlinks=true,       % false: boxed links; true: colored links
linkcolor=red,          % color of internal links (change box color with linkbordercolor)
citecolor=blue,      % color of links to bibliography
filecolor=magenta,  % color of file links
urlcolor=cyan          % color of external links
}

\usepackage{array}

\allowdisplaybreaks

\externaldocument{../../SRS/SRS}

\input{../../Comments}
%% Common Parts

\newcommand{\progname}{Software Engineering} % PUT YOUR PROGRAM NAME HERE
\newcommand{\authname}{Team \#18, Gouda Engineers 
\\ Aidan Goodyer
\\ Jeremy Orr
\\ Leo Vugert
\\ Nathan Perry
\\ Tim Pokanai} % AUTHOR NAMES                  

\usepackage{hyperref}
    \hypersetup{colorlinks=true, linkcolor=blue, citecolor=blue, filecolor=blue,
                urlcolor=blue, unicode=false}
    \urlstyle{same}
                                


\begin{document}

\title{Module Interface Specification for \progname{}}

\author{\authname}

\date{\today}

\maketitle

\pagenumbering{roman}

\section{Revision History}

\begin{tabularx}{\textwidth}{p{3cm}p{2cm}X}
\toprule {\bf Date} & {\bf Version} & {\bf Notes}\\
\midrule
Date 1 & 1.0 & Notes\\
Date 2 & 1.1 & Notes\\
\bottomrule
\end{tabularx}

~\newpage

\section{Symbols, Abbreviations and Acronyms}

See SRS Documentation at \url{https://github.com/OCD-Rats-Capstone/OCD-Rat-Infrastructure/blob/main/docs/SRS-Volere/SRS.pdf}

\wss{Also add any additional symbols, abbreviations or acronyms}

\newpage

\tableofcontents

\newpage

\pagenumbering{arabic}

\section{Introduction}

The following document details the Module Interface Specifications for
\textbf{RatBat2}, a data analysis web application designed to visualize, query, and process behavioural data from experiments involving rats with Obsessive-Compulsive Disorder (OCD). The system enables researchers to upload experimental trial data, perform natural language–based searches, and generate dynamic visualizations for behavioural comparisons and trend analysis.

Complementary documents include the \textit{System Requirements Specification (SRS)} and \textit{Module Guide (MG)}.  
The full documentation and implementation can be found at  
\url{https://github.com/OCD-Rats-Capstone/OCD-Rat-Infrastructure}.


\section{Notation}

This document adopts a formal notation and structural convention to describe the architecture and module interface specifications (MIS) for the Behavioral Data Analysis Platform for Animal Models of OCD. 

The structure of each MIS follows the framework of \citet{HoffmanAndStrooper1995}, extended to incorporate template modules as described by \citet{GhezziEtAl2003}. The mathematical and logical notation is consistent with Chapter 3 of \citet{HoffmanAndStrooper1995}, with domain-specific adaptations for data processing and behavioral event analysis.

\subsection*{Mathematical and Logical Conventions}

\begin{itemize}
    \item The symbol \texttt{:=} denotes an assignment or multiple assignment statement.  
    For example, \texttt{session.avg\_latency := sum(latency) / count(latency)}.

    \item Conditional expressions follow the form:
    \[
        (c_1 \Rightarrow r_1 \;|\; c_2 \Rightarrow r_2 \;|\; \dots \;|\; c_n \Rightarrow r_n)
    \]
    For instance:
    \[
        (\text{event.type = "drug 1"} \Rightarrow \text{increment(count)} \;|\; \text{event.type = "drug 2"} \Rightarrow \text{record(duration)})
    \]
    where the rule corresponding to the first true condition is applied.

    \item Logical connectives are used as follows:
    \begin{itemize}
        \item \(\land\) — logical AND  
        \item \(\lor\) — logical OR  
        \item ` — logical NOT  
        \item \(\Rightarrow\) — implication
    \end{itemize}

    \item Set notation follows standard mathematical conventions:  
    \(\{x \mid P(x)\}\) represents the set of all \(x\) satisfying predicate \(P(x)\).  
    Example:  
    \(\{ e \in Events \mid e.duration > 10s \}\) denotes the set of long-duration events.

    \item Ranges are denoted as \([a..b]\), representing all integer time indices \(t\) such that \(a \leq t \leq b\).

    \item Function definitions are expressed as mappings:  
    \[
        f : Input \rightarrow Output
    \]
    Example:  
    \[
        \texttt{computeMetrics} : SessionData \rightarrow BehavioralSummary
    \]
\end{itemize}

\subsection*{Data and Type Notation}

\begin{itemize}
    \item \(Session\) — a structured dataset representing a single experimental trial.  
    \item \(Event\) — a tuple of attributes describing a behavioral observation, e.g. \((type, timestamp, duration)\).
    \item \(Metric\) — a computed quantitative value derived from one or more events.
    \item \(AnimalID\) — a unique identifier for a subject.
    \item \(TrialSet := \{ s_1, s_2, ..., s_n \}\) — the set of all sessions recorded for a given animal.
\end{itemize}

\subsection*{Units and Measurement Conventions}
\begin{itemize}
    \item Time values are expressed in seconds (s).  
    \item Counts and frequencies are represented as integers.  
    \item Statistical metrics (e.g., mean, standard deviation, z-score) are expressed as real numbers (\(\mathbb{R}\)).
\end{itemize}

These conventions are used consistently throughout the system specification to ensure mathematical clarity and facilitate unambiguous interpretation of the behavioral data models, algorithms, and transformations.


\section{Module Decomposition}

The following table is taken directly from the Module Guide document for this project.

\begin{table}[h!]
  \centering
  \begin{tabular}{p{0.3\textwidth} p{0.6\textwidth}}
  \toprule
  \textbf{Level 1} & \textbf{Level 2}\\
  \midrule
  
  {Hardware-Hiding Module} & ~ \\
  \midrule
  
  \multirow{3}{0.3\textwidth}{Behaviour-Hiding Module} & Front-End Interface Module\\
  & API Layer Module\\
  & Data Schema and Storage Module\\
  & Data Visualization Module\\
  \midrule
  
  \multirow{4}{0.3\textwidth}{Software Decision Module} & NLP Query Processor \\
  & Data Processing Pipeline Module\\
  & Authentication and Access Control Module\\
  & Fault and Error Management Module\\
  \bottomrule
  
  \end{tabular}
  \caption{Module Hierarchy}
  \label{TblMH}
  \end{table}

\newpage
~\newpage

\section{MIS of M1: Hardware-Hiding Module} \label{Module 1}

\subsection{Module}

\par{The hardware hiding module will be implemented by the native OS of the machine running our system. This will not be designed by us. For
detailed design implementation please see the breakdown of the OS of the local machine running our system's web page (client-side) or the 
Linux-based OS of the server hosting our system's backend and database.}

\section{MIS of M2: Query Processor Module} \label{Module 2}

\subsection{Module}

\wss{Short name for the module}

NLPModule

\subsection{Uses}


\subsection{Syntax}

\subsubsection{Exported Constants}

\par

\subsubsection{Exported Access Programs}

\begin{center}
\begin{tabular}{p{4cm} p{4cm} p{4cm} p{4cm}}
\hline
\textbf{Name} & \textbf{In} & \textbf{Out} & \textbf{Exceptions} \\
\hline
NLP Processor & String & String & incoherent prompt, empty input \\
\hline
\end{tabular}
\end{center}

\subsection{Semantics}

\subsubsection{State Variables}

\par{None}

\subsubsection{Environment Variables}

\begin{itemize}

  \item{$LLM\_API\_KEY$: String}
  \item{$System\_Prompt$: String}
  \item{$Schema\_Attributes$: String{}}
  \item{$Valid\_Operators$: String{}}

\end{itemize}

\subsubsection{Assumptions}

\begin{itemize}

  \item{The system prompt environment variable will provide enough information to the LLM to ensure that the returned response string conforms exactly
  to the structure of the needed input format for a database query.}
  \item{The large language model used to process the natural language input into database commands will require the use of an API key and will not
  be locally hosted, thus the need for an API key environment variable.}


\end{itemize}


\subsubsection{Access Routine Semantics}

\noindent $NLP\_Processor(String: natural\_language) \rightarrow String$:
\begin{itemize}
\item output: out := if $(is\_valid\_output(natural\_language))~then~
\\ \{LLM\_API\_Response(natural\_language)\}~else~\{raise~incoherent\_prompt\}$ 
\item exceptions: $(natural\_language$ IS NULL $=> empty\_input), (!is\_valid\_output(natural\_language) => incoherent\_prompt)$
\end{itemize}

\subsubsection{Local Functions}

\noindent $is\_valid\_output(String: LLM\_Response) \rightarrow bool$
\begin{itemize}
\item output: out := $($b such that b $\in ([x \in Schema\_Attributes];[y \in Valid\_Operators];[<Value>]))$
\end{itemize}

\noindent $LLM\_API\_Call(String: natural\_language, String: System\_Prompt) \rightarrow String$
\begin{itemize}
\item $output: LLM\_Response := (Call~LLM~API~where: \\ Prompt input = natural\_language~prepended~by~System\_Prompt)$
\end{itemize}



\section{MIS of M3: Front-End Interface Module} \label{Module 3}

\wss{You can reference SRS labels, such as R\ref{R_Inputs}.}

\wss{It is also possible to use \LaTeX for hypperlinks to external documents.}

\subsection{Module}

FrontEnd

\subsection{Uses}

\subsection{Syntax}

\subsubsection{Exported Constants}

\par{N/A}

\subsubsection{Exported Access Programs}

\par{N/A}

\subsection{Semantics}

\subsubsection{State Variables}

\begin{itemize}
  \item{NavPage: String (URL)}
  \item{HTTPStatusCode: int}
  \item{UserInterfaceComponents: {bool, String, int, char} (Variables which provide values to UI components such as a checkbox being checked or the value
  in an input box. Aggregated as it would be a waste of time to list each individual component and it's state variable(s))}
  \item{DataSet: JSON Object}
\end{itemize}

\subsubsection{State Invariant}

\begin{itemize}
  \item{$NavPage \in \{FrontEndURLs\}$ }
  \item{$HTTPStatusCode \in \{$Standard HTTP Status Codes$\}$}
\end{itemize}

\subsubsection{Environment Variables}

  \begin{itemize}
    \item{BackEndLocation: String (URL for HTTP request destinations)}
    \item{FrontEndURLs: String{}}
  \end{itemize}

\subsubsection{Assumptions}

\begin{itemize}
  \item{Only one backend location will be necessary. HTTP request body will contain sufficient information to call the correct corresponding
  module and serve the proper response.}
  \item{Front end module will be free of explicit business logic and will simply package user input and serve output responded. Aggregating user input,
  formatting HTTP requests and serving responses are the extent of the front end requirements.}
\end{itemize}
\subsubsection{Access Routine Semantics}

\par{N/A}

\subsubsection{Local Functions}

\par{$HTTP\_Send\_Request(HTTP~Request: HTTP\_request) \rightarrow HTTP\_Response$: }
\begin{itemize}
  \item{output: $out := HTTP\_response~from~backend~modules$}
  \item{exception: None}
\end{itemize}

\par{$Nav\_Page(String: Nav\_URL):$ }
\begin{itemize}
  \item{$transition: NavPage := Nav\_URL$}
  \item{$exception: exc := (NavPage = bad~URL => page\_not\_found)$}
\end{itemize}

\par{$HTTP\_Error\_Handle(HTTP~Response: HTTP\_response):$ }
\begin{itemize}
  \item{$transition: HTTPStatusCode := HTTP\_response.StatusCode, NavPage := \\'Initial~Landing~Page', UserInterfaceComponents := \{$component initial values$\}$ }
  \item{exception: None}
\end{itemize}

\par{$HTTP\_Serve\_Data($JSON Object: DataSet): }
\begin{itemize}
  \item{$transition: DataSet := HTTP\_Send\_Request(HTTP\_request).body$}
  \item{$exception: exc := (HTTP\_Send\_Request(HTTP\_request).body == NULL =>$ Empty HTTP Response)}

\end{itemize}

\section{MIS of M4: API Layer Module} \label{Module 4}

\subsection{Module}

APILayer

\subsection{Uses}

\subsection{Syntax}

\subsubsection{Exported Constants}

\par{N/A}

\subsubsection{Exported Access Programs}

\begin{center}
\begin{tabular}{p{4cm} p{5cm} p{3cm} p{4cm}}
\hline
\textbf{Name} & \textbf{In} & \textbf{Out} & \textbf{Exceptions} \\
\hline
Get DataSet & None & JSON Object & empty dataset \\
Request RawFiles & DataSet\{"FileLocations"\} & File[] & empty dataset, bad URL\\
\hline
\end{tabular}
\end{center}

\subsection{Semantics}

\subsubsection{State Variables}

\begin{itemize}
  \item{$QueryType: String \in \{'NaturalLanguage','Structured'\}$}
  \item{ModuleHasData: bool}
  \item{Dataset: JSON Object}
  \item{RequestedQuery: SQL Query}


\end{itemize}

\subsubsection{Environment Variables}

  \begin{itemize}
    \item{HostedDatabaseEndPoint: String (URL for Database Connection)}
    \item{DatabaseAccessKey: String}
    \item{FRDRDataSetEndPoint: String (URL EndPoint for Dr.Szechtman's Data Sets in the FRDR repository)}
    \item{BaseQuery: SQL Query (To be augmented by requested filters)}
  \end{itemize}

\subsubsection{Assumptions}

\begin{itemize}
  \item{Only one instance of our database will be hosted and thus, one hardcoded endpoint will be sufficient for handling our database connection. Additionally,
  a single access key will be sufficient to provide all users access to the database connection.}
  \item{The Endpoint for accessing Dr.Szechtman's datasets on the FRDR website will not change.}
  \item{The disk space available on the application server will be sufficient to temporarily download several .CSV or .MPG files files from FRDR for
  processing purposes.}
  \item{A boilerplate query for the database will be sufficient for every unique query, only needing to be augmented algorithmically through the requested
  filters.}
\end{itemize}
\subsubsection{Access Routine Semantics}

\noindent $Get\_Dataset() \rightarrow File[]$:
\begin{itemize}
\item output: out := DataSet
\item exceptions: exc:= (HasData := False $=>$ empty\_dataset), exc := ($HTTP\_Response \in \{4++\}$)
\end{itemize}

\noindent $Request\_RawFiles(JSON~Object: DataSet): \rightarrow File[]$
\begin{itemize}
  \item{$transition: out := $Array of files provided via automatic downloads faciliatated by \\ HTTP requests to the DataSet\{"FileLocations"\} Attribute}
  \item{$exception: exc := (HasData := False => Empty\_Result), \\ exc := (Database~Connection~Unsuccessful => Connection\_Refused)$}

\end{itemize}

\subsubsection{Local Functions}

\par{$Choose\_Query\_Process(String: QueryType) \rightarrow JSON~Object$: }
\begin{itemize}
  \item{transition: $ DataSet := if (QueryType == 'NaturalLanguage') \{NLP\_Query(HTTP\_Request)\} \\ elseif (QueryType == 'Structured') \{Structured\_Query(HTTP\_Request)\}$}
  \item{exception: $exc := (\not(QueryType \in \{'NaturalLanguage','Structured'\}) => Bad\_Query\_Type$)}
\end{itemize}

\noindent{$NLP\_Query(HTTP~Request: HTTP\_Request): \rightarrow SQL~Query$ }
\begin{itemize}
  \item{$transition: RequestedQuery := $ Augmented Base Query Based on Output of NLPModule.NLP\_Processor(HTTP\_Request.body)}
  \item{$exception: exc := (HTTP\_Request.body~IS~NULL => empty\_filters)$}
\end{itemize}

\noindent{$Structured\_Query(HTTP~Response: HTTP\_response): \rightarrow SQL~Query$ }
\begin{itemize}
  \item{$transition: RequestedQuery := $ Augmented Base Query Based on filters in HTTP\_Request.body}
  \item{$exception: exc := (HTTP\_Request.body~IS~NULL => empty\_filters)$}
\end{itemize}

\noindent{$Request\_DataSet(SQL~Query: RequestedQuery, String: HostedDatabaseEndPoint): \rightarrow JSON~Object$}
\begin{itemize}
  \item{$transition: DataSet := $JSON Object generated from result of RequestedQuery API request at URL of HostedDatabaseEndPoint}
  \item{$transition: if(DataSet~IS~NOT~NULL) \{HasData := True\}$}
  \item{$exception: exc := ( HasData := False => Empty\_Result), exc := \\ (Database Connection Unsuccessful => Connection\_Refused)$}

\end{itemize}

\section{MIS of M5: Data Schema and Storage Module} \label{Module 5}

\subsection{Module}

DataBaseSchema \\

\par{\textbf{Note: }It does not make sense to outline this as a typical module with functions, variables etc... since this module consists of our database schema
	for querying the experimental data relevant to this project. As a result, this module will be outlined via the schema definition of our database.
  Also note that our team and Dr.Szechtman have already worked to develop a schema that makes sense based on Dr.Szechtman's understanding of the datasets.
  Thus, this will be very detailed and specific, more so than is likely necessary but since the work has already been done, the full design will be
  outlined.}

\subsection{Schema Overview}

\subsubsection{Core Entity Tables}

\begin{itemize}

	\item{1. apparatuses - Testing Equipment}
	\item{2. rats - Subject Animals}
	\item{3. brain\_mainpulations - Surgical interventions}
	\item{4. drug\_rx - Drug regimen combinations}
	\item{5. drug\_rx\_details - Individual compound specifications}

\end{itemize}

\subsubsection{Reference Tables}

\begin{itemize}

	\item{6. light\_cycles - Colony room light/dark cycles}
	\item{7. session\_types - Session classification codes}
	\item{8. testers - Experimenter identifiers}
	\item{9. apparatus\_patterns - Object arrangement configurations}
	\item{10. drugs - Pharmacological agents}
	\item{11. brain\_regions - Brain region codes}
	\item{12. testing\_rooms - Physical testing locations}

\end{itemize}

\subsubsection{Experimental Session Tables}

\begin{itemize}

	\item{13. experimental\_sessions - All 20,491 experimental sessions}
	\item{14. session\_drug\_details - Drug administration for experimental sessions}

\end{itemize}

\subsubsection{Histology and Video Tables}

\begin{itemize}

	\item{15. histology\_results – Post-mortem lesion verification}
	\item{16. movie\_files – Video file metadata}

\end{itemize}

\subsubsection{Data File Locations Table}

\begin{itemize}

	\item{17. data\_file\_locations – Physical storage locations (336,507 records)}

\end{itemize}

\subsubsection{Session Data Files and Observations Tables}

\begin{itemize}

	\item{18. session\_data\_files – Session-to-data-file linking}
	\item{19. session\_observations – Quality control and notes}

\end{itemize}

\subsection{Schema Definition}

\begin{itemize}

\item 1. apparatuses - Testing Equipment
\begin{itemize}
	\item{\textbf{Schema Definition: }CREATE TABLE apparatuses (apparatus\_id SERIAL PRIMARY KEY, apparatus\_name VARCHAR(100), apparatus\_code VARCHAR(10),
	      testing\_room\_id INTEGER REFERENCES testing\_rooms(room\_id),
	      \\ apparatus\_location\_in\_room VARCHAR(50),
	      apparatus\_notes TEXT);}
	\item{\textbf{Referenced By: }experimental\_sessions \textbf{Foreign Key: }apparatus\_id (Many-to-One: many sessions use same apparatus)}
\end{itemize}
\item 2. rats - Subject Animals
\begin{itemize}
	\item{\textbf{Schema Definition: }CREATE TABLE rats (
	      rat\_id SERIAL PRIMARY KEY,
	      legacy\_rat\_id VARCHAR(20) UNIQUE,
	      sex VARCHAR(10),
	      light\_cycle\_id INTEGER REFERENCES light\_cycles(light\_cycle\_id),
	      strain VARCHAR(50),
	      birth\_date DATE
	      );
	      }
	\item{\textbf{Referenced By: }experimental\_sessions \textbf{Foreign Key: }rat\_id (One-to-Many: one rat, many sessions)}
	\item{\textbf{Referenced By: }brain\_manipulations \textbf{Foreign Key: }rat\_id (One-to-Many: one rat can have multiple manipulations)}
\end{itemize}
\item 3. brain\_mainpulations - Surgical interventions
\begin{itemize}
	\item{\textbf{Schema Definition: }CREATE TABLE brain\_manipulations (
	      manipulation\_id SERIAL PRIMARY KEY,
	      rat\_id INTEGER REFERENCES rats(rat\_id),
	      surgery\_type VARCHAR(50),
	      target\_region\_id INTEGER REFERENCES brain\_regions(region\_id),
	      surgery\_date DATE
	      );

	      }
	\item{\textbf{Referenced By: }histology\_results table \textbf{Foreign Key: }manipulation\_id (One-to-One: each lesion has histology assessment)}
\end{itemize}
\item 4. drug\_rx - Drug regimen combinations
\begin{itemize}
	\item{\textbf{Schema Definition: }CREATE TABLE drug\_rx (
	      drug\_rx\_id SERIAL PRIMARY KEY,
	      rx\_label VARCHAR(200),
	      num\_drugs INTEGER,
	      notes\_drug\_rx TEXT
	      );

	      }
	\item{\textbf{Referenced By: }experimental\_sessions \textbf{Foreign Key: }drug\_rx\_id (Many-to-One: many sessions use same regimen)}
	\item{\textbf{Referenced By: }drug\_rx\_details  \textbf{Foreign Key: }drug\_rx\_id (One-to-Many: each regimen has multiple detail records)}
\end{itemize}

\item 5. drug\_rx\_details - Individual compound specifications

\begin{itemize}
	\item{\textbf{Schema Definition: }CREATE TABLE drug\_rx\_details (
	      drug\_rx\_detail\_id SERIAL PRIMARY KEY,
	      drug\_rx\_id INTEGER REFERENCES drug\_rx(drug\_rx\_id),
	      drug\_id INTEGER REFERENCES drugs(drug\_id),
	      prescribed\_dose NUMERIC(10,6),
	      dose\_unit VARCHAR(50),
	      route VARCHAR(50),
	      time\_before\_session\_hours NUMERIC(10,2)
	      );


	      }
\end{itemize}

\item light\_cycles
\begin{itemize}
	\item{\textbf{Schema Definition: }CREATE TABLE light\_cycles (light\_cycle\_id SERIAL PRIMARY KEY, cycle\_name VARCHAR(50) NOT NULL, lights\_on\_time TIME, lights\_off\_time TIME);}
	\item{\textbf{Referenced By: }rats \textbf{Foreign Key: }light\_cycle\_id (Many-to-One: many rats share same light cycle)}
\end{itemize}
\item session\_types - Session classification codes
\begin{itemize}
	\item{\textbf{Schema Definition: }CREATE TABLE session\_types (session\_type\_id SERIAL PRIMARY KEY, type\_name VARCHAR(100) NOT NULL, session\_types\_notes TEXT);}
	\item{\textbf{Referenced By: }experimental\_sessions \textbf{Foreign Key: }session\_type\_id (Many-to-One: many sessions can be the same type)}
\end{itemize}
\item 8. testers - Experimenter identifiers
\begin{itemize}
	\item{\textbf{Schema Definition: }CREATE TABLE testers (tester\_id SERIAL PRIMARY KEY, first\_last\_name VARCHAR(100), initials VARCHAR(10));}
	\item{\textbf{Referenced By: }experimental\_sessions \textbf{Foreign Key: }tester\_id (Many-to-One: each tester conducted many sessions)}
\end{itemize}
\item 9. apparatus\_patterns - Object arrangement configurations
\begin{itemize}
	\item{\textbf{Schema Definition: }CREATE TABLE apparatus\_patterns (pattern\_id SERIAL PRIMARY KEY, pattern\_description TEXT);}
	\item{\textbf{Referenced By: }experimental\_sessions \textbf{Foreign Key: }pattern\_id (Many-to-One: many sessions use the same pattern)}
\end{itemize}
\item 10. drugs - Pharmacological agents
\begin{itemize}
	\item{\textbf{Schema Definition: }CREATE TABLE drugs (drug\_id SERIAL PRIMARY KEY, drug\_abbreviation VARCHAR(20), drug\_name
	      VARCHAR(200), drug\_is\_active BOOLEAN, dose\_unit VARCHAR(20), drugs\_notes TEXT);}
	\item{\textbf{Referenced By: }drug\_rx\_details \textbf{Foreign Key: }drug\_id (Many-to-One: specifies drugs in prescription regimens)}
	\item{\textbf{Referenced By: }session\_drug\_details \textbf{Foreign Key: }drug\_id (Many-to-One: records actual drugs administered)}
\end{itemize}
\item 11. brain\_regions - Brain region codes
\begin{itemize}
	\item{\textbf{Schema Definition: }CREATE TABLE brain\_regions (region\_id SERIAL PRIMARY KEY, region\_name VARCHAR(100), region\_abbreviation VARCHAR(20));}
	\item{\textbf{Referenced By: }brain\_manipulations \textbf{Foreign Key: }target\_region\_id (Many-to-One: many rats received surgery targeting the same brain region)}
\end{itemize}
\item 12. testing\_rooms - Physical testing locations
\begin{itemize}
	\item{\textbf{Schema Definition: }CREATE TABLE testing\_rooms (room\_id SERIAL PRIMARY KEY, room\_name VARCHAR(50), room\_notes
	      TEXT);}
	\item{\textbf{Referenced By: }apparatuses \textbf{Foreign Key: }testing\_room\_id (Many-to-One: multiple apparatus can be in the same room)}
\end{itemize}

\item 13. experimental\_sessions - All 20,491 experimental sessions
\begin{itemize}
	\item{\textbf{Schema Definition: }CREATE TABLE experimental\_sessions (session\_id SERIAL PRIMARY KEY, legacy\_session\_id VARCHAR(7)
	      UNIQUE, session\_type\_id INTEGER REFERENCES session\_types(session\_type\_id), rat\_id INTEGER
	      REFERENCES rats(rat\_id), body\_weight\_grams NUMERIC(6,2), drug\_rx\_id INTEGER REFERENCES
	      drug\_rx(drug\_rx\_id), tester\_id INTEGER REFERENCES testers(tester\_id), effective\_manipulation\_id
	      INTEGER REFERENCES \\ brain\_manipulations(manipulation\_id), apparatus\_id INTEGER REFERENCES
	      apparatuses(apparatus\_id), pattern\_id INTEGER REFERENCES \\ apparatus\_patterns(pattern\_id),
	      locale\_in\_room VARCHAR(20), room\_id INTEGER REFERENCES testing\_rooms(room\_id),
	      testing\_lights\_on SMALLINT, session\_timestamp TIMESTAMP, data\_trial\_id VARCHAR(28),
	      cumulative\_drug\_injection\_number INTEGER);
	      }
	\item{\textbf{Referenced By: }session\_drug\_details \textbf{Foreign Key: } session\_id (One-to-Many. Each session has 1-4 drug detail records specifying actual drugs and doses administered)}
	\item{\textbf{Referenced By: }session\_data\_files \textbf{Foreign Key: }  session\_id (One-to-Many. Each session links to multiple data files (videos, tracks, plots))}
	\item{\textbf{Referenced By: }session\_observations \textbf{Foreign Key: }  session\_id (One-to-Many. Quality control notes and behavioral observations for sessions)}
\end{itemize}
\item 14. session\_drug\_details - Drug administration for experimental sessions

\begin{itemize}
	\item{\textbf{Schema Definition: }CREATE TABLE session\_drug\_details (
	      detail\_id SERIAL PRIMARY KEY,
	      session\_id INTEGER REFERENCES experimental\_sessions(session\_id),
	      drug\_id INTEGER REFERENCES drugs(drug\_id),
	      dose\_given NUMERIC(12,5),
	      dose\_unit VARCHAR(20),
	      route VARCHAR(50),
	      time\_before\_session\_hours NUMERIC(10,2),
	      cumulative\_injections\_count\_for\_regimen INTEGER,
	      \\ cumulative\_apparatus\_exposure\_number INTEGER
	      );
	      }
\end{itemize}

\item 15. histology\_results – Post-mortem lesion verification
\begin{itemize}
	\item{\textbf{Schema Definition: }CREATE TABLE histology\_results (
	      histology\_id SERIAL PRIMARY KEY,
	      manipulation\_id INTEGER REFERENCES \\ brain\_manipulations(manipulation\_id),
	      histology\_status\_id INTEGER REFERENCES lkp\_histology\_status(histology\_status\_id),
	      intact\_status\_id INTEGER REFERENCES functional\_intact\_status(intact\_status\_id),
	      left\_percent\_damage DOUBLE PRECISION,
	      right\_percent\_damage DOUBLE PRECISION,
	      notes\_histology\_results TEXT
	      );}
\end{itemize}
\item 16. movie\_files – Video file metadata
\begin{itemize}
	\item{\textbf{Schema Definition: }CREATE TABLE movie\_files (
	      movie\_file\_id SERIAL PRIMARY KEY,
	      movie\_file\_name VARCHAR(255),
	      legacy\_movie\_file\_name VARCHAR(255),
	      file\_size\_mb DOUBLE PRECISION,
	      is\_file\_available BOOLEAN,
	      notes\_movie\_files TEXT
	      );}
	\item{\textbf{Referenced By: }session\_data\_files \textbf{Foreign Key: }.movie\_file\_id (Many-to-One: Multiple sessions can reference the same movie file (video multiplexing))}
\end{itemize}
\item 17. data\_file\_locations – Physical storage locations (336,507 records)
\begin{itemize}
	\item{\textbf{Schema Definition: }CREATE TABLE data\_file\_locations (
	      data\_file\_location\_id SERIAL PRIMARY KEY,
	      data\_file\_id INTEGER REFERENCES session\_data\_files(data\_file\_id),
	      repository\_type\_id INTEGER REFERENCES lkp\_repository\_type(repository\_type\_id),
	      repo\_dataset\_doi VARCHAR(100),
	      repo\_dataset\_name VARCHAR(255),
	      repo\_dataset\_version VARCHAR(24),
	      repo\_file\_url TEXT,
	      is\_primary\_copy BOOLEAN,
	      date\_linked TIMESTAMP,
	      api\_endpoint VARCHAR(255),
	      notes\_data\_file\_locations TEXT
	      );
	      }
\end{itemize}
\item 18. session\_data\_files – Session-to-data-file linking
\begin{itemize}
	\item{\textbf{Schema Definition: }CREATE TABLE session\_data\_files (
	      data\_file\_id SERIAL PRIMARY KEY,
	      session\_id INTEGER REFERENCES experimental\_sessions(session\_id),
	      movie\_file\_id INTEGER REFERENCES movie\_files(movie\_file\_id),
	      start\_frame INTEGER,
	      object\_type\_id INTEGER REFERENCES lkp\_data\_file\_object\_type(object\_type\_id),
	      file\_extension VARCHAR(10),
	      file\_name VARCHAR(255),
	      file\_size\_mb DOUBLE PRECISION,
	      notes\_session\_data\_files TEXT
	      );
	      }
	\item{\textbf{Referenced By: }data\_file\_locations \textbf{Foreign Key: }..data\_file\_id (One-to-Many. Each file record has multiple location records (local, FRDR v1, FRDR v2, backups))}
\end{itemize}
\item 19. session\_observations – Quality control and notes

\begin{itemize}
	\item{\textbf{Schema Definition: }CREATE TABLE session\_observations (
	      observation\_id SERIAL PRIMARY KEY,
	      session\_id INTEGER REFERENCES \\ experimental\_sessions(session\_id),
	      observation\_text TEXT,
	      num\_falls\_during\_test INTEGER,
	      falls\_during\_test\_time\_when\_fell\_str VARCHAR(255)
	      );
	      }
\end{itemize}

\end{itemize}

\newpage

\section{MIS of M6: Data Visualization Module} \label{Module 6} 

\subsection{Module}

\textbf{Visualization Engine}


\subsection{Uses}

\par{
  This module is responsible for rendering interactive charts and graphs using the D3.js library. it accepts raw or processed data, as well a Visualization type specifier and generates the necessary JavaScript/SVG code to render the visualization.
}

\subsection{Syntax}

\subsubsection{Exported Constants}

This module exports a set of predefined visualization/chart type constants.

\begin{itemize}
  \item \textbf{VTYPE\_BAR} String constant representing bar chart. 
  \item \textbf{VTYPE\_PLOT} String constant representing a general plot area (points, curves). 
  \item \textbf{VTYPE\_PIE} String constant representing a pie chart. 
  \item \textbf{VTYPE\_HEATMAP} String constant representing an (x,y) heatmap. 
  \item \textbf{VTYPE\_TIMESERIES} String constant representing time series (x,y,t) position data. 

\end{itemize}




\subsubsection{Exported Access Programs}

\begin{center}
\begin{tabular}{p{2cm} p{4cm} p{2cm} p{4cm}}
\hline
\textbf{Name} & \textbf{In} & \textbf{Out} & \textbf{Exceptions} \\
\hline
GenerateViz & (VType, DataBlob, ChartConfig)  & Viz\_SVG & InvalidDataSchema \\
\hline
\end{tabular}
\end{center}

\subsection{Semantics}

\subsubsection{State Variables}

The Data Visualization Model acts as a stateless data transformation stage and therefore has no associated state variables. 

\subsubsection{Environment Variables}

\begin{itemize}
  \item \textbf{D3\_Library}: External Dependency 
\end{itemize}
\subsubsection{Assumptions}


\begin{itemize}
  \item All input data provided via DataBlob is in valid JSON form.
  \item There will be continued D3 support and existence of all visualization types exported. 
  \item The client environment supports modern SVG standards. 
\end{itemize}


\subsubsection{Access Routine Semantics}

GenerateViz(VType, DataBlob, ChartConfig):
\begin{itemize}
\item  \textbf{Transition:} \textbf{None}
\item  \textbf{Output:} \textbf{Viz\_SVG} (String) representing the rendered D3.js SVG visualization. 
\item[] Note that the image is represented by the XML path string for the resulting SVG. 
\item exception: \textbf{InvalidDataSchema}: Error raised if the JSON DataBlob is incompatible with the specified visualization type.
\item[] For example, most charts are constructed from a tabular form. If this is violated the exception will be raised. 
\end{itemize}

\subsubsection{Local Functions}

\noindent VerifyDataSchema(Data,VType):

\begin{itemize}
  \item \textbf{Purpose:} to check if the JSON data blob conforms to the input specification of the requested visualization. For example, a scatter plot requires 2 vectors for x and y data, while a bar chart requires categorical (key,value) pairs.
  \item \textbf{Output: Boolean} True if valid, False otherwise.  
\end{itemize}

\noindent ApplyD3Template(Data,Type,Config):

\begin{itemize}
  \item \textbf{Purpose:} Convert the validated input data into the specified D3 template, and generate the SVG output. 
  \item \textbf{Output:} SVG (String). 
\end{itemize}


\section{MIS of M7: Data Processing Pipeline Module} \label{Module 7} 

\subsection{Module}

DataProcessingPipeline

\subsection{Uses}
\begin{itemize}
    \item Data Schema and Storage – for retrieving structured experimental datasets and metadata.
    \item Data Visualization – for providing processed outputs ready for visualization.
    \item API Layer – for receiving processing requests and returning results.
\end{itemize}

\subsection{Syntax}

\subsubsection{Exported Constants}
None.

\subsubsection{Exported Access Programs}

\begin{center}
\begin{tabular}{p{3cm} p{4cm} p{4cm} p{3cm}}
\hline
\textbf{Name} & \textbf{In} & \textbf{Out} & \textbf{Exceptions} \\
\hline
ProcessData & Dataset (CSV / JSON), Parameters (dict) & Processed DataFrame / JSON & invalid format, missing fields \\
ComputeMetrics & Processed DataFrame / JSON & Metrics (dict) & empty input, computation error \\
GenerateSummary & Processed DataFrame / JSON & Summary Statistics (dict) & invalid input \\
\hline
\end{tabular}
\end{center}

\subsection{Semantics}

\subsubsection{State Variables}
None. All computations are stateless and performed on data inputs provided per request.

\subsubsection{Environment Variables}
\begin{itemize}
    \item \textbf{DATA\_CACHE\_PATH:} String – temporary directory for caching intermediate results.
    \item \textbf{MAX\_WORKERS:} Integer – controls the number of concurrent processing threads.
\end{itemize}

\subsubsection{Assumptions}
\begin{itemize}
    \item Input datasets conform to schema definitions from Data Schema and Storage Module.
    \item Parameters for filtering, grouping, or aggregation are valid and correspond to known attributes.
    \item The environment has sufficient memory and CPU resources for parallel processing.
\end{itemize}

\subsubsection{Access Routine Semantics}

\noindent ProcessData(Dataset, Parameters) $\rightarrow$ Processed DataFrame:
\begin{itemize}
    \item \textbf{output:} out := cleaned and structured dataset prepared for computation.
    \item \textbf{exception:} invalid format if dataset fails schema validation; missing fields if required attributes are absent.
\end{itemize}

\noindent ComputeMetrics(Processed DataFrame) $\rightarrow$ Metrics:
\begin{itemize}
    \item \textbf{output:} out := computed behavioral metrics (e.g., grooming duration, trial success rate).
    \item \textbf{exception:} empty input if dataset is empty; computation error if numerical operation fails.
\end{itemize}

\noindent GenerateSummary(Processed DataFrame) $\rightarrow$ Summary Statistics:
\begin{itemize}
    \item \textbf{output:} out := dictionary of aggregated statistics (mean, median, standard deviation).
    \item \textbf{exception:} invalid input if DataFrame structure mismatches expected schema.
\end{itemize}

\subsubsection{Local Functions}

\noindent validateInput(Dataset) $\rightarrow$ bool  
\begin{itemize}
    \item \textbf{output:} true if dataset matches schema attributes and expected datatypes.
\end{itemize}

\noindent computeAggregateMetrics(DataFrame) $\rightarrow$ dict  
\begin{itemize}
    \item \textbf{output:} dictionary of computed metrics (e.g., total grooming time, movement variance).
\end{itemize}

\noindent cleanData(Dataset) $\rightarrow$ DataFrame  
\begin{itemize}
    \item \textbf{output:} dataset with missing values handled and outliers removed.
\end{itemize}


\newpage

\section{MIS of M8: Authentication and Access Control Module} \label{Module 8}

\subsection{Module}

\par{The Authentication and Access Control Module is responsible for verifying 
user identities, assigning access roles, and ensuring scalable entry points for users 
that access the software through the web application or through our API clients.}

\subsection{Uses}

\par{This module interfaces with the following modules:}

\begin{itemize}
  \item{M4: API Layer Module}
  \item{M3: Front-End Interface Module}
\end{itemize}

\par{Additionally, this module will use external libraries, including 
FastAPI's OAuth2PasswordBearer, and JWT libraries for token generation, 
encryption, and validation.}

\subsection{Syntax}

\subsubsection{Exported Constants}

\par{N/A}

\subsubsection{Exported Access Programs}

\begin{center}
\begin{tabular}{p{4cm} p{4cm} p{4cm} p{4cm}}
\hline
\textbf{Name} & \textbf{In} & \textbf{Out} & \textbf{Exceptions} \\
\hline
Login User & String Username, String Password & String JWT Token & InvalidCredentialsError \\
Verify Token & String JWT Token & User Object & TokenExpiredError, InvalidTokenError \\
Get User & String JWT Token & User Object & Authentication Error \\
Authorize Role & User Object, String JWT Token & Boolean & AccessDeniedError \\
Logout User & String JWT Token & Boolean & InvalidTokenError \\
Refresh Token & String JWT Token & String JWT Token & InvalidTokenError, TokenExpiredError \\
\hline
\end{tabular}
\end{center}

\subsection{Semantics}

\subsubsection{State Variables}

\begin{itemize}
  \item{User Credentials: A HashMap of usernames to password hashes.}
  \item{Active Tokens: Dictionary of JWT Token Strings mapped to Expiry Timestamps.}
  \item{Permissions Mapping: Dictionary of Role Strings mapped to Permission Strings.}
\end{itemize}

\subsubsection{Environment Variables}

\begin{itemize}
  \item{JWT SECRET KEY: User key used for token signing and verification.}
  \item{MAX CONCURRENT USERS: Maximum allowed concurrent users.}
  \item{NUM CONCURRENT USERS: Number of current concurrent users.}
  \item{USER DATABASE URL: Connection URL to the credential and role management database.}
  \item{TOKEN EXPIRY: Token lifetime configuration for user session.}
\end{itemize}

\subsubsection{Assumptions}

\begin{itemize}
  \item{Authentication requests may orginiate concurrently from both UI or API clients.}
  \item{Passwords are stored securely using a one-way hash.}
  \item{HTTPS is enforced for all login and token exchange operations.}
\end{itemize}

\subsubsection{Access Routine Semantics}

\noindent loginUser():
\begin{itemize}
\item transition: Verifies credentials from userCredentials.
\item output: JWT Token as a String.
\item exception: InvalidCredentialsError.
\end{itemize}

\noindent verifyToken():
\begin{itemize}
\item transition: Checks JWT Token signature, expiry time, and status in activeTokens.
\item output: User Object if token is valid.
\item exception: InvalidTokenError or TokenExpiredError.
\end{itemize}

\noindent getUser():
\begin{itemize}
\item transition: None.
\item output: User Object.
\item exception: AuthenticationError.
\end{itemize}

\noindent authorizeRole():
\begin{itemize}
\item transition: None.
\item output: True if authorized, False otherwise.
\item exception: AccessDeniedError.
\end{itemize}

\noindent logoutUser():
\begin{itemize}
\item transition: Removes identifier from activeTokens, logging out the user.
\item output: True if successful, False otherwise.
\item exception: InvalidTokenError.
\end{itemize}

\noindent refreshToken():
\begin{itemize}
\item transition: Validates token and reissues a fresh JWT Token with renewed expiry time.
\item output: JWT Token as a String
\item exception: InvalidTokenError or TokenExpiredError.
\end{itemize}

\subsubsection{Local Functions}

\begin{itemize}
  \item{hashPassword(String password): Generates a secure hash of a user password.}
  \item{createJWTToken(User Object, int expiry): Generates a JWT Token containing user data and expiry period.}
  \item{decodeJWTToken(String JWTToken): Decodes and validates a JWT Token, extracting user data and expiry period.}
  \item{validateUser(String username, String password): Authenticates user via stored credentials.}
\end{itemize}

\newpage

\bibliographystyle {plainnat}
\bibliography {../../../refs/References}

\newpage

\section{Appendix} \label{Appendix}

\wss{Extra information if required}

\newpage{}

\section*{Appendix --- Reflection}

\wss{Not required for CAS 741 projects}

The information in this section will be used to evaluate the team members on the
graduate attribute of Problem Analysis and Design.

\input{../../Reflection.tex}

\begin{enumerate}
  \item What went well while writing this deliverable? 
  \item What pain points did you experience during this deliverable, and how
    did you resolve them?
  \item Which of your design decisions stemmed from speaking to your client(s)
  or a proxy (e.g. your peers, stakeholders, potential users)? For those that
  were not, why, and where did they come from?
  \item While creating the design doc, what parts of your other documents (e.g.
  requirements, hazard analysis, etc), it any, needed to be changed, and why?
  \item What are the limitations of your solution?  Put another way, given
  unlimited resources, what could you do to make the project better? (LO\_ProbSolutions)
  \item Give a brief overview of other design solutions you considered.  What
  are the benefits and tradeoffs of those other designs compared with the chosen
  design?  From all the potential options, why did you select the documented design?
  (LO\_Explores)
\end{enumerate}


\end{document}
