\documentclass[12pt, titlepage]{article}

\usepackage{amsmath, mathtools}

\usepackage[round]{natbib}
\usepackage{amsfonts}
\usepackage{amssymb}
\usepackage{graphicx}
\usepackage{colortbl}
\usepackage{xr}
\usepackage{hyperref}
\usepackage{longtable}
\usepackage{xfrac}
\usepackage{tabularx}
\usepackage{float}
\usepackage{siunitx}
\usepackage{booktabs}
\usepackage{multirow}
\usepackage[section]{placeins}
\usepackage{caption}
\usepackage{fullpage}

\hypersetup{
bookmarks=true,     % show bookmarks bar?
colorlinks=true,       % false: boxed links; true: colored links
linkcolor=red,          % color of internal links (change box color with linkbordercolor)
citecolor=blue,      % color of links to bibliography
filecolor=magenta,  % color of file links
urlcolor=cyan          % color of external links
}

\usepackage{array}

\allowdisplaybreaks

\externaldocument{../../SRS/SRS}

\input{../../Comments}
%% Common Parts

\newcommand{\progname}{Software Engineering} % PUT YOUR PROGRAM NAME HERE
\newcommand{\authname}{Team \#18, Gouda Engineers 
\\ Aidan Goodyer
\\ Jeremy Orr
\\ Leo Vugert
\\ Nathan Perry
\\ Tim Pokanai} % AUTHOR NAMES                  

\usepackage{hyperref}
    \hypersetup{colorlinks=true, linkcolor=blue, citecolor=blue, filecolor=blue,
                urlcolor=blue, unicode=false}
    \urlstyle{same}
                                


\begin{document}

\title{Module Interface Specification for \progname{}}

\author{\authname}

\date{\today}

\maketitle

\pagenumbering{roman}

\section{Revision History}

\begin{tabularx}{\textwidth}{p{3cm}p{2cm}X}
\toprule {\bf Date} & {\bf Version} & {\bf Notes}\\
\midrule
Date 1 & 1.0 & Notes\\
Date 2 & 1.1 & Notes\\
\bottomrule
\end{tabularx}

~\newpage

\section{Symbols, Abbreviations and Acronyms}

See SRS Documentation at \wss{give url}

\wss{Also add any additional symbols, abbreviations or acronyms}

\newpage

\tableofcontents

\newpage

\pagenumbering{arabic}

\section{Introduction}

The following document details the Module Interface Specifications for
\wss{Fill in your project name and description}

Complementary documents include the System Requirement Specifications
and Module Guide.  The full documentation and implementation can be
found at \url{...}.  \wss{provide the url for your repo}

\section{Notation}

This document adopts a formal notation and structural convention to describe the architecture and module interface specifications (MIS) for the Behavioral Data Analysis Platform for Animal Models of OCD. 

The structure of each MIS follows the framework of \citet{HoffmanAndStrooper1995}, extended to incorporate template modules as described by \citet{GhezziEtAl2003}. The mathematical and logical notation is consistent with Chapter 3 of \citet{HoffmanAndStrooper1995}, with domain-specific adaptations for data processing and behavioral event analysis.

\subsection*{Mathematical and Logical Conventions}

\begin{itemize}
    \item The symbol \texttt{:=} denotes an assignment or multiple assignment statement.  
    For example, \texttt{session.avg\_latency := sum(latency) / count(latency)}.

    \item Conditional expressions follow the form:
    \[
        (c_1 \Rightarrow r_1 \;|\; c_2 \Rightarrow r_2 \;|\; \dots \;|\; c_n \Rightarrow r_n)
    \]
    For instance:
    \[
        (\text{event.type = "drug 1"} \Rightarrow \text{increment(count)} \;|\; \text{event.type = "drug 2"} \Rightarrow \text{record(duration)})
    \]
    where the rule corresponding to the first true condition is applied.

    \item Logical connectives are used as follows:
    \begin{itemize}
        \item \(\land\) — logical AND  
        \item \(\lor\) — logical OR  
        \item ` — logical NOT  
        \item \(\Rightarrow\) — implication
    \end{itemize}

    \item Set notation follows standard mathematical conventions:  
    \(\{x \mid P(x)\}\) represents the set of all \(x\) satisfying predicate \(P(x)\).  
    Example:  
    \(\{ e \in Events \mid e.duration > 10s \}\) denotes the set of long-duration events.

    \item Ranges are denoted as \([a..b]\), representing all integer time indices \(t\) such that \(a \leq t \leq b\).

    \item Function definitions are expressed as mappings:  
    \[
        f : Input \rightarrow Output
    \]
    Example:  
    \[
        \texttt{computeMetrics} : SessionData \rightarrow BehavioralSummary
    \]
\end{itemize}

\subsection*{Data and Type Notation}

\begin{itemize}
    \item \(Session\) — a structured dataset representing a single experimental trial.  
    \item \(Event\) — a tuple of attributes describing a behavioral observation, e.g. \((type, timestamp, duration)\).
    \item \(Metric\) — a computed quantitative value derived from one or more events.
    \item \(AnimalID\) — a unique identifier for a subject.
    \item \(TrialSet := \{ s_1, s_2, ..., s_n \}\) — the set of all sessions recorded for a given animal.
\end{itemize}

\subsection*{Units and Measurement Conventions}
\begin{itemize}
    \item Time values are expressed in seconds (s).  
    \item Counts and frequencies are represented as integers.  
    \item Statistical metrics (e.g., mean, standard deviation, z-score) are expressed as real numbers (\(\mathbb{R}\)).
\end{itemize}

These conventions are used consistently throughout the system specification to ensure mathematical clarity and facilitate unambiguous interpretation of the behavioral data models, algorithms, and transformations.


\section{Module Decomposition}

The following table is taken directly from the Module Guide document for this project.

\begin{table}[h!]
\centering
\begin{tabular}{p{0.3\textwidth} p{0.6\textwidth}}
\toprule
\textbf{Level 1} & \textbf{Level 2}\\
\midrule

{Hardware-Hiding} & ~ \\
\midrule

\multirow{7}{0.3\textwidth}{Behaviour-Hiding} & Input Parameters\\
& Output Format\\
& Output Verification\\
& Temperature ODEs\\
& Energy Equations\\ 
& Control Module\\
& Specification Parameters Module\\
\midrule

\multirow{3}{0.3\textwidth}{Software Decision} & {Sequence Data Structure}\\
& ODE Solver\\
& Plotting\\
\bottomrule

\end{tabular}
\caption{Module Hierarchy}
\label{TblMH}
\end{table}

\newpage
~\newpage

\section{MIS of M1: Hardware-Hiding Module} \label{Module 1}

\subsection{Module}

\par{The hardware hiding module will be implemented by the native OS of the machine running our system. This will not be designed by us. For
detailed design implementation please see the breakdown of the OS of the local machine running our system's web page (client-side) or the 
Linux-based OS of the server hosting our system's backend and database.}

\section{MIS of M2: Query Processor Module} \label{Module 2}

\subsection{Module}

\wss{Short name for the module}

NLPModule

\subsection{Uses}


\subsection{Syntax}

\subsubsection{Exported Constants}

\par

\subsubsection{Exported Access Programs}

\begin{center}
\begin{tabular}{p{4cm} p{4cm} p{4cm} p{4cm}}
\hline
\textbf{Name} & \textbf{In} & \textbf{Out} & \textbf{Exceptions} \\
\hline
NLP Processor & String & String & incoherent prompt, empty input \\
\hline
\end{tabular}
\end{center}

\subsection{Semantics}

\subsubsection{State Variables}

\par{None}

\subsubsection{Environment Variables}

\begin{itemize}

  \item{$LLM\_API\_KEY$: String}
  \item{$System\_Prompt$: String}
  \item{$Schema\_Attributes$: String{}}
  \item{$Valid\_Operators$: String{}}

\end{itemize}

\subsubsection{Assumptions}

\begin{itemize}

  \item{The system prompt environment variable will provide enough information to the LLM to ensure that the returned response string conforms exactly
  to the structure of the needed input format for a database query.}
  \item{The large language model used to process the natural language input into database commands will require the use of an API key and will not
  be locally hosted, thus the need for an API key environment variable.}


\end{itemize}


\subsubsection{Access Routine Semantics}

\noindent $NLP\_Processor(String: natural\_language)$:
\begin{itemize}
\item output: out := if $(is\_valid\_output(natural\_language))~then~
\\ \{LLM\_API\_Response(natural\_language)\}~else~\{raise~incoherent\_prompt\}$ 
\item exceptions: $(natural\_language$ IS NULL $=> empty\_input), (!is\_valid\_output(natural\_language) => incoherent\_prompt)$
\end{itemize}

\subsubsection{Local Functions}

\noindent $is\_valid\_output(String: LLM\_Response) \rightarrow bool$
\begin{itemize}
\item output: out := $($b such that b $\in ([x \in Schema\_Attributes];[y \in Valid\_Operators];[<Value>]))$
\end{itemize}

\noindent $LLM\_API\_Call(String: natural\_language, String: System\_Prompt) \rightarrow String$
\begin{itemize}
\item $output: LLM\_Response := (Call~LLM~API~where: \\ Prompt input = natural\_language~prepended~by~System\_Prompt)$
\end{itemize}



\section{MIS of M3: Front-End Interface Module} \label{Module 3}

\wss{You can reference SRS labels, such as R\ref{R_Inputs}.}

\wss{It is also possible to use \LaTeX for hypperlinks to external documents.}

\subsection{Module}

FrontEnd

\subsection{Uses}

\subsection{Syntax}

\subsubsection{Exported Constants}

\par{N/A}

\subsubsection{Exported Access Programs}

\par{N/A}

\subsection{Semantics}

\subsubsection{State Variables}

\begin{itemize}
  \item{NavPage: String (URL)}
  \item{HTTPStatusCode: int}
  \item{UserInterfaceComponents: {bool, String, int, char} (Variables which provide values to UI components such as a checkbox being checked or the value
  in an input box. Aggregated as it would be a waste of time to list each individual component and it's state variable(s))}
  \item{DataSet: JSON Object}
\end{itemize}

\subsubsection{State Invariant}

\begin{itemize}
  \item{$NavPage \in \{FrontEndURLs\}$ }
  \item{$HTTPStatusCode \in \{$Standard HTTP Status Codes$\}$}
\end{itemize}

\subsubsection{Environment Variables}

  \begin{itemize}
    \item{BackEndLocation: String (URL for HTTP request destinations)}
    \item{FrontEndURLs: String{}}
  \end{itemize}

\subsubsection{Assumptions}

\begin{itemize}
  \item{Only one backend location will be necessary. HTTP request body will contain sufficient information to call the correct corresponding
  module and serve the proper response.}
  \item{Front end module will be free of explicit business logic and will simply package user input and serve output responded. Aggregating user input,
  formatting HTTP requests and serving responses are the extent of the front end requirements.}
\end{itemize}
\subsubsection{Access Routine Semantics}

\par{N/A}

\subsubsection{Local Functions}

\par{$HTTP\_Send\_Request(HTTP~Request: HTTP\_request) \rightarrow HTTP\_Response$: }
\begin{itemize}
  \item{output: $out := HTTP\_response~form~backend~modules$}
  \item{exception: None}
\end{itemize}

\par{$Nav\_Page(String: Nav\_URL):$ }
\begin{itemize}
  \item{$transition: NavPage := Nav\_URL$}
  \item{$exception: exc := (NavPage = bad~URL => page\_not\_found)$}
\end{itemize}

\par{$HTTP\_Error\_Handle(HTTP~Response: HTTP\_response):$ }
\begin{itemize}
  \item{$transition: HTTPStatusCode := HTTP\_response.StatusCode, NavPage := \\'Initial~Landing~Page', UserInterfaceComponents := \{$component initial values$\}$ }
  \item{exception: None}
\end{itemize}

\par{$HTTP\_Serve\_Data($JSON Object: DataSet): }
\begin{itemize}
  \item{$transition: DataSet := HTTP\_Send\_Request(HTTP\_request).body$}
  \item{$exception: exc := (HTTP\_Send\_Request(HTTP\_request).body == NULL =>$ Empty HTTP Response)}

\end{itemize}

\newpage

\bibliographystyle {plainnat}
\bibliography {../../../refs/References}

\newpage

\section{Appendix} \label{Appendix}

\wss{Extra information if required}

\newpage{}

\section*{Appendix --- Reflection}

\wss{Not required for CAS 741 projects}

The information in this section will be used to evaluate the team members on the
graduate attribute of Problem Analysis and Design.

\input{../../Reflection.tex}

\begin{enumerate}
  \item What went well while writing this deliverable? 
  \item What pain points did you experience during this deliverable, and how
    did you resolve them?
  \item Which of your design decisions stemmed from speaking to your client(s)
  or a proxy (e.g. your peers, stakeholders, potential users)? For those that
  were not, why, and where did they come from?
  \item While creating the design doc, what parts of your other documents (e.g.
  requirements, hazard analysis, etc), it any, needed to be changed, and why?
  \item What are the limitations of your solution?  Put another way, given
  unlimited resources, what could you do to make the project better? (LO\_ProbSolutions)
  \item Give a brief overview of other design solutions you considered.  What
  are the benefits and tradeoffs of those other designs compared with the chosen
  design?  From all the potential options, why did you select the documented design?
  (LO\_Explores)
\end{enumerate}


\end{document}