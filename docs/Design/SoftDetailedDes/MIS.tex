\documentclass[12pt, titlepage]{article}

\usepackage{amsmath, mathtools}

\usepackage[round]{natbib}
\usepackage{amsfonts}
\usepackage{amssymb}
\usepackage{graphicx}
\usepackage{colortbl}
\usepackage{xr}
\usepackage{hyperref}
\usepackage{longtable}
\usepackage{xfrac}
\usepackage{tabularx}
\usepackage{float}
\usepackage{siunitx}
\usepackage{booktabs}
\usepackage{multirow}
\usepackage[section]{placeins}
\usepackage{caption}
\usepackage{fullpage}

\hypersetup{
bookmarks=true,     % show bookmarks bar?
colorlinks=true,       % false: boxed links; true: colored links
linkcolor=red,          % color of internal links (change box color with linkbordercolor)
citecolor=blue,      % color of links to bibliography
filecolor=magenta,  % color of file links
urlcolor=cyan          % color of external links
}

\usepackage{array}

\externaldocument{../../SRS/SRS}

\input{../../Comments}
%% Common Parts

\newcommand{\progname}{Software Engineering} % PUT YOUR PROGRAM NAME HERE
\newcommand{\authname}{Team \#18, Gouda Engineers 
\\ Aidan Goodyer
\\ Jeremy Orr
\\ Leo Vugert
\\ Nathan Perry
\\ Tim Pokanai} % AUTHOR NAMES                  

\usepackage{hyperref}
    \hypersetup{colorlinks=true, linkcolor=blue, citecolor=blue, filecolor=blue,
                urlcolor=blue, unicode=false}
    \urlstyle{same}
                                


\begin{document}

\title{Module Interface Specification for \progname{}}

\author{\authname}

\date{\today}

\maketitle

\pagenumbering{roman}

\section{Revision History}

\begin{tabularx}{\textwidth}{p{3cm}p{2cm}X}
\toprule {\bf Date} & {\bf Version} & {\bf Notes}\\
\midrule
Date 1 & 1.0 & Notes\\
Date 2 & 1.1 & Notes\\
\bottomrule
\end{tabularx}

~\newpage

\section{Symbols, Abbreviations and Acronyms}

See SRS Documentation at \wss{give url}

\wss{Also add any additional symbols, abbreviations or acronyms}

\newpage

\tableofcontents

\newpage

\pagenumbering{arabic}

\section{Introduction}

The following document details the Module Interface Specifications for
\wss{Fill in your project name and description}

Complementary documents include the System Requirement Specifications
and Module Guide.  The full documentation and implementation can be
found at \url{...}.  \wss{provide the url for your repo}

\section{Notation}

This document adopts a formal notation and structural convention to describe the architecture and module interface specifications (MIS) for the Behavioral Data Analysis Platform for Animal Models of OCD. 

The structure of each MIS follows the framework of \citet{HoffmanAndStrooper1995}, extended to incorporate template modules as described by \citet{GhezziEtAl2003}. The mathematical and logical notation is consistent with Chapter 3 of \citet{HoffmanAndStrooper1995}, with domain-specific adaptations for data processing and behavioral event analysis.

\subsection*{Mathematical and Logical Conventions}

\begin{itemize}
    \item The symbol \texttt{:=} denotes an assignment or multiple assignment statement.  
    For example, \texttt{session.avg\_latency := sum(latency) / count(latency)}.

    \item Conditional expressions follow the form:
    \[
        (c_1 \Rightarrow r_1 \;|\; c_2 \Rightarrow r_2 \;|\; \dots \;|\; c_n \Rightarrow r_n)
    \]
    For instance:
    \[
        (\text{event.type = "drug 1"} \Rightarrow \text{increment(count)} \;|\; \text{event.type = "drug 2"} \Rightarrow \text{record(duration)})
    \]
    where the rule corresponding to the first true condition is applied.

    \item Logical connectives are used as follows:
    \begin{itemize}
        \item \(\land\) — logical AND  
        \item \(\lor\) — logical OR  
        \item ` — logical NOT  
        \item \(\Rightarrow\) — implication
    \end{itemize}

    \item Set notation follows standard mathematical conventions:  
    \(\{x \mid P(x)\}\) represents the set of all \(x\) satisfying predicate \(P(x)\).  
    Example:  
    \(\{ e \in Events \mid e.duration > 10s \}\) denotes the set of long-duration events.

    \item Ranges are denoted as \([a..b]\), representing all integer time indices \(t\) such that \(a \leq t \leq b\).

    \item Function definitions are expressed as mappings:  
    \[
        f : Input \rightarrow Output
    \]
    Example:  
    \[
        \texttt{computeMetrics} : SessionData \rightarrow BehavioralSummary
    \]
\end{itemize}

\subsection*{Data and Type Notation}

\begin{itemize}
    \item \(Session\) — a structured dataset representing a single experimental trial.  
    \item \(Event\) — a tuple of attributes describing a behavioral observation, e.g. \((type, timestamp, duration)\).
    \item \(Metric\) — a computed quantitative value derived from one or more events.
    \item \(AnimalID\) — a unique identifier for a subject.
    \item \(TrialSet := \{ s_1, s_2, ..., s_n \}\) — the set of all sessions recorded for a given animal.
\end{itemize}

\subsection*{Units and Measurement Conventions}
\begin{itemize}
    \item Time values are expressed in seconds (s).  
    \item Counts and frequencies are represented as integers.  
    \item Statistical metrics (e.g., mean, standard deviation, z-score) are expressed as real numbers (\(\mathbb{R}\)).
\end{itemize}

These conventions are used consistently throughout the system specification to ensure mathematical clarity and facilitate unambiguous interpretation of the behavioral data models, algorithms, and transformations.


\section{Module Decomposition}

The following table is taken directly from the Module Guide document for this project.

\begin{table}[h!]
  \centering
  \begin{tabular}{p{0.3\textwidth} p{0.6\textwidth}}
  \toprule
  \textbf{Level 1} & \textbf{Level 2}\\
  \midrule
  
  {Hardware-Hiding Module} & ~ \\
  \midrule
  
  \multirow{3}{0.3\textwidth}{Behaviour-Hiding Module} & Front-End Interface Module\\
  & API Layer Module\\
  & Data Schema and Storage Module\\
  & Data Visualization Module\\
  \midrule
  
  \multirow{4}{0.3\textwidth}{Software Decision Module} & NLP Query Processor \\
  & Data Processing Pipeline Module\\
  & Authentication and Access Control Module\\
  & Fault and Error Management Module\\
  & Security and Dependency Management Module\\
  \bottomrule
  
  \end{tabular}
  \caption{Module Hierarchy}
  \label{TblMH}
  \end{table}

\newpage
~\newpage

\section{MIS of \wss{Module Name}} \label{Module} \wss{Use labels for
  cross-referencing}

\wss{You can reference SRS labels, such as R\ref{R_Inputs}.}

\wss{It is also possible to use \LaTeX for hypperlinks to external documents.}

\subsection{Module}

\wss{Short name for the module}

\subsection{Uses}


\subsection{Syntax}

\subsubsection{Exported Constants}

\subsubsection{Exported Access Programs}

\begin{center}
\begin{tabular}{p{2cm} p{4cm} p{4cm} p{2cm}}
\hline
\textbf{Name} & \textbf{In} & \textbf{Out} & \textbf{Exceptions} \\
\hline
\wss{accessProg} & - & - & - \\
\hline
\end{tabular}
\end{center}

\subsection{Semantics}

\subsubsection{State Variables}

\wss{Not all modules will have state variables.  State variables give the module
  a memory.}

\subsubsection{Environment Variables}

\wss{This section is not necessary for all modules.  Its purpose is to capture
  when the module has external interaction with the environment, such as for a
  device driver, screen interface, keyboard, file, etc.}

\subsubsection{Assumptions}

\wss{Try to minimize assumptions and anticipate programmer errors via
  exceptions, but for practical purposes assumptions are sometimes appropriate.}

\subsubsection{Access Routine Semantics}

\noindent \wss{accessProg}():
\begin{itemize}
\item transition: \wss{if appropriate} 
\item output: \wss{if appropriate} 
\item exception: \wss{if appropriate} 
\end{itemize}

\wss{A module without environment variables or state variables is unlikely to
  have a state transition.  In this case a state transition can only occur if
  the module is changing the state of another module.}

\wss{Modules rarely have both a transition and an output.  In most cases you
  will have one or the other.}

\subsubsection{Local Functions}

\wss{As appropriate} \wss{These functions are for the purpose of specification.
  They are not necessarily something that is going to be implemented
  explicitly.  Even if they are implemented, they are not exported; they only
  have local scope.}

\newpage

\bibliographystyle {plainnat}
\bibliography {../../../refs/References}

\newpage

\section{Appendix} \label{Appendix}

\wss{Extra information if required}

\newpage{}

\section*{Appendix --- Reflection}

\wss{Not required for CAS 741 projects}

The information in this section will be used to evaluate the team members on the
graduate attribute of Problem Analysis and Design.

\input{../../Reflection.tex}

\begin{enumerate}
  \item What went well while writing this deliverable? 

  The team was much more focused this deliverable than we had been for the previous deliverable (VnV Plan). Now that
  midterms are over, we had more time to really spend a lot of time trying to do this deliverable well. The previous deliverable
  was less of a priority due to the midterms dwarfing it in terms of grade weight. We think this resulted in a more well done,
  coherent and complete deliverable that puts us in a good spot as we begin the implementation of our system. Additionally, the teamwork
  aspect of this deliverable was also quite good this deliverable. This likely also ties back into having more time in general but group
  members were more willing to take on more parts and help others that needed it. Two members even focused largely on the PoC rather than
  writing the design document because the other three members had the capacity to shoulder a larger documentation load.

  \item What pain points did you experience during this deliverable, and how
    did you resolve them?

    There were not too many major pain points experienced during this deliverable, it generally went quite well. The first pain point of note
    would be ensuring consistency across all parts and between documents. Although we work as a team, we mainly complete the section individually
    and thus we need to ensure that two members don't write things in different sections that are not aligned with each other. This deliverable's sections
    were the most interconnected yet and thus it will require a pretty thorough document review to ensure everything is consistent. The second pain point
    for this deliverable was the MIS design of each module. This was the final section to be completed as we were not quite sure how to split it up or exactly
    what was required. As it turns out this is not just one section but rather a section for each module (so essentially 10 additional sections of the document)
    which was quite a large workload to complete, especially since it was left right until the end. This was simply resolved by splitting up the modules based
    on who had the capacity to take them on and to commit the needed time to completing it. As mentioned earlier, members generally had time to do this so it wasn't
    a terribly painful process.

  \item Which of your design decisions stemmed from speaking to your client(s)
  or a proxy (e.g. your peers, stakeholders, potential users)? For those that
  were not, why, and where did they come from?

    First and foremost, essentially all of module 5 was developed by speaking with our client, Dr. Henry Szechtman. He
    is the definitive expert on these datasets and thus he is most equipped to develop a schema for our system. Luckily, he
    has enough technical knowledge to develop this for us to look over and decide if it is workable, which we believe it is.
    Additionally, the data-processing module, the visualization module and the front-end module were all largely influenced by our discussions
    with Dr.Szechtman, specficially through him describing to us what specific functions he wants the system to have (e.g. data processing abilities,
    front-end design, available visualizations). Outside of these design aspects, most of the rest of the design was created outside of
    much influence from our client. This includes things like the natural language processing module, the fault and error management module, the API
    layer module etc... The reason that these decisions were not made based on client interactions is because they are more purely part
    of what could be called the 'intermediate technical design' rather than user specifications. For example, the natural language processing
    module requires it to be designed such that natural language can be accurately converted into a SQL query. Alternatively, the error management
    module simply regulates UI, API and query inputs to prevent any undesired behaviour. Both of these design outlines do not require
    much input from the client/users as they are basically purely technical implementation (there are not really many different ways to specify natural
    language into a SQL query). As such, they are left to the technical experts which are the members of our group.
  \item While creating the design doc, what parts of your other documents (e.g.
  requirements, hazard analysis, etc), it any, needed to be changed, and why?
  \item What are the limitations of your solution?  Put another way, given
  unlimited resources, what could you do to make the project better? (LO\_ProbSolutions)
  \item Give a brief overview of other design solutions you considered.  What
  are the benefits and tradeoffs of those other designs compared with the chosen
  design?  From all the potential options, why did you select the documented design?
  (LO\_Explores)
\end{enumerate}


\end{document}
