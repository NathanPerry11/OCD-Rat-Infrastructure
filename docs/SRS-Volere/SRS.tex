% THIS DOCUMENT IS FOLLOWS THE VOLERE TEMPLATE BY Suzanne Robertson and James Robertson
% ONLY THE SECTION HEADINGS ARE PROVIDED
%
% Initial draft from https://github.com/Dieblich/volere
%
% Risks are removed because they are covered by the Hazard Analysis
\documentclass[12pt]{article}

\usepackage{booktabs}
\usepackage{tabularx}
\usepackage{hyperref}
\hypersetup{
    bookmarks=true,         % show bookmarks bar?
      colorlinks=true,      % false: boxed links; true: colored links
    linkcolor=red,          % color of internal links (change box color with linkbordercolor)
    citecolor=green,        % color of links to bibliography
    filecolor=magenta,      % color of file links
    urlcolor=cyan           % color of external links
}

\newcommand{\lips}{\textit{Insert your content here.}}

%% Comments

\usepackage{color}

\newif\ifcomments\commentstrue %displays comments
%\newif\ifcomments\commentsfalse %so that comments do not display

\ifcomments
\newcommand{\authornote}[3]{\textcolor{#1}{[#3 ---#2]}}
\newcommand{\todo}[1]{\textcolor{red}{[TODO: #1]}}
\else
\newcommand{\authornote}[3]{}
\newcommand{\todo}[1]{}
\fi

\newcommand{\wss}[1]{\authornote{magenta}{SS}{#1}} 
\newcommand{\plt}[1]{\authornote{cyan}{TPLT}{#1}} %For explanation of the template
\newcommand{\an}[1]{\authornote{cyan}{Author}{#1}}

%% Common Parts

\newcommand{\progname}{Software Engineering} % PUT YOUR PROGRAM NAME HERE
\newcommand{\authname}{Team \#18, Gouda Engineers 
\\ Aidan Goodyer
\\ Jeremy Orr
\\ Leo Vugert
\\ Nathan Perry
\\ Tim Pokanai} % AUTHOR NAMES                  

\usepackage{hyperref}
    \hypersetup{colorlinks=true, linkcolor=blue, citecolor=blue, filecolor=blue,
                urlcolor=blue, unicode=false}
    \urlstyle{same}
                                


\begin{document}

\title{Software Requirements Specification for \progname: subtitle describing software} 
\author{\authname}
\date{\today}
	
\maketitle

~\newpage

\pagenumbering{roman}

\tableofcontents

~\newpage

\section*{Revision History}

\begin{tabularx}{\textwidth}{p{3cm}p{2cm}X}
\toprule {\textbf{Date}} & {\textbf{Version}} & {\textbf{Notes}}\\
\midrule
Date 1 & 1.0 & Notes\\
Date 2 & 1.1 & Notes\\
\bottomrule
\end{tabularx}

~\\

~\newpage
\section{Purpose of the Project}
\subsection{User Business}

\par{The business of the user is to use the system to aid in the
creation and analysis of experiments relating to a large but fragmented data set
of rat trials that investigate compulsive behaviour of rats and how they relate to various 
factors such as drug injections or brain legions. This data set is comprised of videos of rats
moving on a flat surface, files of their (x,y) coordinates and plots that show the trajectories
of these coordinates. Users may want to
extract a concise set of pre-existing data that can be applicable to
their hypothesis to streamline their workflow and avoid the necessity of carrying out trials.
Users may also want to use the system to provide analysis on the data set as a means of
further supplementing ther academic work. This can include drawing behavioural metrics of the rats
based on various parameters (i.e. rate of compulsion based on injection type) or data visualizations
based similarily on compulsive behaviours as they relate to attributes of the rats.}

\lips
\subsection{Goals of the Project}

\par{The high-level goal of the project is to provide users in the field of behavioural sciences,
a unified and non-technical way of drawing from the vast amount of available data on OCD rat trials
to aid in their academic work and experiments. This will be accomplished via the following sub-goals:}

\begin{itemize}
    \item Create a DBMS that provides query access to the entirety of the data set of rat trials as well as all metadata associated with the data.
    \item Provide a UI that makes querying this data approachable to non-technical users by incorporating familiar and intuitive filtering and searching techniques
    (Attribute filters, Natural Language search bar). Further the UI must provide users with pre-generated query options that are likely to
    be useful for users that may not know where to start. This could include selecting all rat trials of a particular drug injection along with
    all saline control data.
    \item Provide an algorithm that can identify compulsive behaviour by looking at each rat trial
    and provide options for behavioural metrics and data visualizations based on various attributes that the user can extract 
    for their purposes.
\end{itemize}


\lips
\section{Stakeholders}
\subsection{Client}

\par{The clients for this project are Dr. Henry Szechtman and Dr. Anna Dvorkin-Gheva, two professors at McMaster University who are experts in
the fields of Psychiatry and Behavioural Neurosciences and Bioinformatics, respectively. They have worked extensively with these OCD rat trials and
were responsible for generating and depositing this data in the FRDR repository where it currently is stored. Dr. Szechtman and Dr. Dvorkin-Gheva want
to drastically improve the availability of the vast data set they have created so that the trials they conducted can become useful to other students and academics in related fields.}
\lips
\subsection{Customer}

\par{Since our application is intended to be open-source and accessible to the public for free, our customers will simply be a group of end-users
of our application. Our end users will primarily involve behavioural neuroscience researchers like Dr. Henry Szechtman and Dr. Anna Dvorkin-Gheva, 
as well as graduate students and lab members, who will all benefit from the user-friendly and accessible functionality of the platform. 
Additionally we will have data scientists as customers, who will benefit from our application architecture anf offered functionality for 
their purposes. Lastly, collaborating institutions and organizations from the open research community will be users of our application, 
since the application is intended to be free-use and accessible to all.}
\lips
\subsection{Other Stakeholders}
\lips
\subsection{Hands-On Users of the Project}
\lips
\subsection{Personas}
\lips
\subsection{Priorities Assigned to Users}
\lips
\subsection{User Participation}
\lips
\subsection{Maintenance Users and Service Technicians}
\lips

\section{Mandated Constraints}
\subsection{Solution Constraints}
\lips
\subsection{Implementation Environment of the Current System}
\lips
\subsection{Partner or Collaborative Applications}
\lips

\begin{tabular}{|m{5cm}|m{10cm}|}
    \hline
    Collaborative System & System Overview \\
    \hline
    FRDR Repository & The Federated Research Data Repository is a 'bilingual bilingual publishing 
    platform for sharing and 
    preserving Canadian research data. 
    It is a curated, general-purpose repository, 
    custom built for large datasets.' This is where the data set of the rat trials is physically located
    and is dispersed across 29 independent datasets. Our system will provide a unified database schema but will
    pull data from this repository.\\
    \hline
    ratbat.mcmaster.ca & While not a directly collaborative system, this is a system made by a previous years' capstone
    team to address the same problem that our system seeks to. It will be a collaborative system in the sense that ir provides
    a reference of potential ways to approach our solution, ideas that work well and can be carried forward
    and providing visibility to shortcomings of the system will help us to avoid repeating mistakes. \\
    \hline
\end{tabular}

\subsection{Off-the-Shelf Software}

\begin{itemize}
    \item The system shall make use of the FRDR API in order to fetch the relevant data from the data set as we will not be hosting a seperate database to physically store the data for our system
    \item In the case of implementing natural language processing, an Off-the-Shelf LLM will be used to accomplish this (In other words we will not be writing our own LLM model for this purpose).
    Note that this model may need to be trained to provide results that are acceptable for our purposes.
\end{itemize}

\subsection{Anticipated Workplace Environment}
\lips
\subsection{Schedule Constraints}

\par{Due to the nature of the capstone course, the scheduling constraints
map directly to dates in which major deliverable related to the project are due in the capstone course.
They are laid out below:}

\begin{tabular}{|m{5cm}|m{10cm}|}
    \hline
    Project Milestone & Scheduling Constraint\\
    \hline
    Software Requirements Specification & Oct 6 2025\\
    \hline
    Verification and Validation Plan & Oct 27 2025\\
    \hline
    Design Document Revision -1 & Nov 10 2025\\
    \hline
    Proof of Concept Demonstration & Nov 17-28 2025\\
    \hline
    Design Documentation Revision 0 & Jan 19 2025\\
    \hline
    Revision 0 Design Demonstration & Feb 2-13 2025\\
    \hline
    Verification and Validation Report & Mar 9 2025\\
    \hline
    Extras (Performance Report + User Manual) & Mar 9 2025\\
    \hline
    Final System Demonstration & Mar 23-29 2025\\
    \hline
    Final Documentation & April 6 2025\\
    \hline
\end{tabular}




\subsection{Budget Constraints}

\par{There is very limited budget available for this project. The department of Computing and Software at McMaster University 
will provide \$125 CAD for approved expenses. Outside of that funding, we are asked not to exceed spending of \$500 CAD
culmulatively as a team. Thus, the total budget constraints for this project are \$500 CAD in total and \$375 of our team's personal funding.}

\subsection{Enterprise Constraints}
\lips

\section{Naming Conventions and Terminology}
\subsection{Glossary of All Terms, Including Acronyms, Used by Stakeholders
involved in the Project}
\lips

\section{Relevant Facts And Assumptions}
\subsection{Relevant Facts}
\lips
\subsection{Business Rules}
\lips
\subsection{Assumptions}
\lips

\section{The Scope of the Work}
\subsection{The Current Situation}
\lips
\subsection{The Context of the Work}
\lips
\subsection{Work Partitioning}
\lips
\subsection{Specifying a Business Use Case (BUC)}
\lips

\section{Business Data Model and Data Dictionary}
\subsection{Business Data Model}
\lips
\subsection{Data Dictionary}
\lips

\section{The Scope of the Product}
\subsection{Product Boundary}
\lips
\subsection{Product Use Case Table}
\lips
\subsection{Individual Product Use Cases (PUC's)}
\lips

\section{Functional Requirements}
\subsection{Functional Requirements}
\lips

\section{Look and Feel Requirements}



\subsection{Appearance Requirements}

\begin{itemize}
    \item The system shall resemble a webstore or online boutique in which possibly useful queries are packaged and presented for selection to guide 
    a non-technical user audience.
    \item The system shall not overwhelm the user with too many or overly complex visible features as to not intimidate a non-technical user audience.
    \item The system shall present the filtering and searching capabilities of the system in a way that is familiar and intuitive, such as the 
    filtering provided in an online shopping storefront.
\end{itemize}

\subsection{Style Requirements}
\lips

\section{Usability and Humanity Requirements}



\subsection{Ease of Use Requirements}

\begin{itemize}
    \item The system shall have a very low ease of use floor. This floor being scrolling through
    pre-packaged query options and clicking on the one most relevant to their needs.
    \item The system shall have a reasonably low ease of use ceiling with intuitive an intuitive filtering interface
    as well as a simple natural language search interface and provided options for generating metrics and visualizations.
\end{itemize}

\subsection{Personalization and Internationalization Requirements}
\lips



\subsection{Learning Requirements}

\begin{itemize}
    \item Assuming that the user is familiar with nature of the data itself, the system should
    have a learning curve of less than 30 minutes. Users shall easily be able to find and use searching
    mechanisms and querying pre-packaged and custom searches shall be very intuitive.
    \item The system shall provide options to generate available metrics or visualizations based on the queried
    data so the user does not need to learn to create their own.
\end{itemize}

\subsection{Understandability and Politeness Requirements}
\lips
\subsection{Accessibility Requirements}
\lips

\section{Performance Requirements}

\subsection{Speed and Latency Requirements}
\begin{itemize}
    \item The system shall return all query results within 2 seconds for result sets of up to 5,000 records.
    \item All network requests shall have latency below 100 ms under normal operating conditions.
\end{itemize}

\subsection{Safety-Critical Requirements}
\begin{itemize}
    \item There are no safety-critical operations for this system, as it handles only user interface and data processing with no direct impact on human safety.
\end{itemize}

\subsection{Precision or Accuracy Requirements}
\begin{itemize}
    \item All query results shall be accurate and filtered according to user specifications.
    \item There shall be no inconsistencies between the source database and query results.
\end{itemize}

\subsection{Robustness or Fault-Tolerance Requirements}
\begin{itemize}
    \item The system shall log and report all input errors without crashing.
    \item The backend shall handle errors gracefully, storing detailed logs for debugging and monitoring purposes.
\end{itemize}

\subsection{Capacity Requirements}
\begin{itemize}
    \item The system shall support up to 250 concurrent users and handle 5,000 transactions per hour.
    \item The database and associated controllers shall be able to manage up to 11 TB of data.
\end{itemize}

\subsection{Scalability or Extensibility Requirements}
\begin{itemize}
    \item The system shall allow additional modules to be integrated without requiring major changes to existing features.
    \item Under heavy user loads, the system shall maintain at least 80\% of its performance efficiency.
\end{itemize}

\subsection{Longevity Requirements}
\begin{itemize}
    \item The system shall be designed to operate reliably for at least 5 years, with multiple teams and developers able to maintain and extend it.
    \item The system shall be compatible with any operating system when running offline.
\end{itemize}


\section{Operational and Environmental Requirements}
\subsection{Expected Physical Environment}
\begin{itemize}
    \item The system shall be able to run on a standard desktop with atleast 12 GB RAM and average processor.
    \item The software shall function correctly on Windows, macOS, and Linux operating systems.
    \item The software should be able to run in normal office environment conditions which includes temperatures between 15°C and 30°C. 
\end{itemize}

\subsection{Wider Environment Requirements}

\begin{itemize}
    \item The system shall be able to run the top three popular browers.
\end{itemize}


\subsection{Requirements for Interfacing with Adjacent Systems}

\begin{itemize}
    \item Data exchanged with adjacent systems shall use JSON format and be transmitted over HTTPS.
\end{itemize}

\subsection{Productization Requirements}
\begin{itemize}
    \item The system will use Docker and Kubernetes as it's envoirment for assebility to run on all systems.
    \item The system will provide all code changes and software updates on the GitHub Repository. 
\end{itemize}

\subsection{Release Requirements}
\begin{itemize}
    \item The system releases shall follow the versioning defined by MAJOR\#.MINOR\#.PATCH\#.
    \item The system will provide all code changes and software updates on the GitHub Repository. Code then can be pulled and tagged from the appraite GitHub release.
\end{itemize}


\section{Maintainability and Support Requirements}
\subsection{Maintenance Requirements}
\lips
\subsection{Supportability Requirements}
\lips
\subsection{Adaptability Requirements}
\lips

\section{Security Requirements}
\subsection{Access Requirements}

\par{There are no access requirements for our system, since it will be publicly accessible.}

\subsection{Integrity Requirements}

\par{The product shall prevent itself, the databases, and other files within it from intentional abuse.}

\subsection{Privacy Requirements}

\par{The system does not collect or use any user-related data that could be considered private and/or confidential. \newline \indent
The system does not store any personal or sensitive personal data, it only stores and presents the data which is already 
publicly available through FRDR. \newline \indent There are no privacy requirements for the system.}

\subsection{Audit Requirements}
\lips
\subsection{Immunity Requirements}
\lips

\section{Cultural Requirements}
\subsection{Cultural Requirements}
\par{There are no cultural requirements for the system.}

\section{Compliance Requirements}
\subsection{Legal Requirements}

\par{There are no legal requirements to be considered for the development of this project. The project does not involve the storage or 
use of user data, it only offers the searching and presentation of an existing public data set. \newline \indent We have consent from our 
supervisors, Dr. Henry Szechtman and Dr. Anna Dvorkin-Gheva to use their data set in the development of the project. Additionally, 
we are allowed to reference the work of the capstone group, ratbat.mcmaster.ca, which they supervised during the previous year.}

\subsection{Standards Compliance Requirements}
\lips

\section{Open Issues}
\lips

\section{Off-the-Shelf Solutions}
\subsection{Ready-Made Products}
\lips
\subsection{Reusable Components}
\lips
\subsection{Products That Can Be Copied}
\lips

\section{New Problems}
\subsection{Effects on the Current Environment}
\lips
\subsection{Effects on the Installed Systems}
\lips
\subsection{Potential User Problems}

\par{An important note to make about the project is that it is not a mandatory replacement for the existing data set and infrastructure 
available through FRDR. Users besides the commissioners of our project, our supervisors, can simply keep using the existing system 
if that better suits their needs. \newline \indent The comissioners of our project, and external users of the project due to its public 
availability, could have a potential issue with the Natural Language Processing (NLP) used in querying raw data files. Since natural language 
will be converted to a query statement using an AI model, it cannot be guaranteed the model will be accurate 100\% of the time when 
creating these queries from natural language.}

\subsection{Limitations in the Anticipated Implementation Environment That May
Inhibit the New Product}
\lips
\subsection{Follow-Up Problems}
\lips

\section{Tasks}
\subsection{Project Planning}
\lips
\subsection{Planning of the Development Phases}
\lips

\section{Migration to the New Product}
\subsection{Requirements for Migration to the New Product}

\par{There are no requirements for migration to the new product other than that the user must familiarize themselves with the new domain and UI if they are used to using the FRDR repository.}


\subsection{Data That Has to be Modified or Translated for the New System}

\par{The new system will take the raw data directly from the FRDR repository and present it to users. There are use cases in which the data will be processed for the purposes 
     of producing metrics or visualizations which is relevant to the functional requirements rather than this section. Thus, no data needs to be modified or translated for this new system.}

\section{Costs}
\lips
\section{User Documentation and Training}
\subsection{User Documentation Requirements}
\lips
\subsection{Training Requirements}
\lips

\section{Waiting Room}
\lips

\section{Ideas for Solution}
\lips

\newpage{}
\section*{Appendix --- Reflection}

The purpose of reflection questions is to give you a chance to assess your own
learning and that of your group as a whole, and to find ways to improve in the
future. Reflection is an important part of the learning process.  Reflection is
also an essential component of a successful software development process.  

Reflections are most interesting and useful when they're honest, even if the
stories they tell are imperfect. You will be marked based on your depth of
thought and analysis, and not based on the content of the reflections
themselves. Thus, for full marks we encourage you to answer openly and honestly
and to avoid simply writing ``what you think the evaluator wants to hear.''

Please answer the following questions.  Some questions can be answered on the
team level, but where appropriate, each team member should write their own
response:


\begin{enumerate}
  \item What went well while writing this deliverable? 
  \item What pain points did you experience during this deliverable, and how did
  you resolve them?
  \item How many of your requirements were inspired by speaking to your
  client(s) or their proxies (e.g. your peers, stakeholders, potential users)?
  \item Which of the courses you have taken, or are currently taking, will help
  your team to be successful with your capstone project.
  \item What knowledge and skills will the team collectively need to acquire to
  successfully complete this capstone project?  Examples of possible knowledge
  to acquire include domain specific knowledge from the domain of your
  application, or software engineering knowledge, mechatronics knowledge or
  computer science knowledge.  Skills may be related to technology, or writing,
  or presentation, or team management, etc.  You should look to identify at
  least one item for each team member.
  \item For each of the knowledge areas and skills identified in the previous
  question, what are at least two approaches to acquiring the knowledge or
  mastering the skill?  Of the identified approaches, which will each team
  member pursue, and why did they make this choice?
\end{enumerate}


\section{Nathan Perry Reflection:}

\begin{enumerate}
  \item What went well while writing this deliverable? 

  \par{ Understanding the expectations for this deliverable went well, especially relative to the previous deliverable. In the goals and problem statement
  document, I felt that we had a general idea of what we needed to do but not a concrete guide. Using the Volere template, we were able to find very helpful
  documentation which provided an overview of specifically what was expected to be put in each section. This made writing the document feel much more structured
  and provided confidence we were on the right track, especially since the document itself gives very little to go off of aside from section titles.}

  \item What pain points did you experience during this deliverable, and how did
  you resolve them?

  \par{ Far and away the biggest and nearly only pain point for this deliverable was the delegation of work among the members. A few of us like the frontload
  the effort over the deadline window while others like to backload the effort and thus, a few members ended up starting early and needing to take on more
  than their fair share of tasks as an inevitable scramble to complete the deliverable ensued. To try to resolve this, the members who did not start earlier offered
  to take on the rest of the work, telling the others they don't need to continue to contribute. This was somewhat helpful but the deliverable needs to get done
  so everyone ended up helping to some extent anyways. In the future, we decided to resolve this by delegating tasks before work begins. For the SRS
  we used a 'take-as-you-go' approach to issues whereas from now on we will be assigning everything ahead of time. }


  \item How many of your requirements were inspired by speaking to your
  client(s) or their proxies (e.g. your peers, stakeholders, potential users)?

\par{I cannot give a concrete number of requirements that were inspired by speaking to our client but we did have a meeting with Dr. Henry Szechtman
in the week leading up to the writing of our SRS document. The biggest requirement that he directly inspired was the idea to pre-package queries
and display them like a webstore so that users who may not even know exactly what they are looking for can derive value from our system. Furthermore,
our filtering requirements were guided by a reference website created by the National Cancer Institute which
was provided by our client as a reference. \href{https://portal.gdc.cancer.gov/analysis_page?app=CohortBuilder&tab=general.}. Requirements
to make the project maintainable by people other than the original developers and to be open source was also inspired by our client meeting as Dr. Szechtman wants the
project to carry on after we graduate.
}

  \item Which of the courses you have taken, or are currently taking, will help
  your team to be successful with your capstone project.

  \par{ Our project is very multi disciplinary as it is a full stack project and thus plenty of what we learned will go into
  our development. First and foremost, our databases class will provide the knowledge needed to design a database schema as a foundation
  for our entire system. Additionally, the Human-Computer interfaces class we are currently taking will be helpful in creating a user interface
  that is approachable for a non-technical audience. Furthermore, all of the very coding heavy courses including object oriented programming, system
  design and software development will help us to implement the core business logic of the application and to connect the different components.
  Finally, I will add that the our data structures class will help us to efficiently store, package and represent the data objects being transmitted
  in our system. }
\end{enumerate}

\section{Timothy Pokanai Reflection}

\begin{enumerate}
  \item What went well while writing the deliverable?
  \par{I would say that we were able to smoothly and consistently deliver our preliminary SRS document. What I mean is as we worked throughout the deliverable 
  our individual ideas and interpretations of our project description, through the forms of requirements, were aligned and consistent. To be fair we may have had 
  a commit or two where something I wrote was modified to better suit the context of our project, but that was the only occurence in terms of peer performed changes. 
  I would attribute our team's ability to work individually with seamless integration to the fact that we all acquired a great grasp of the scope of the project and 
  the constraints associated with it early on. A lot of this can be attributed to our supervisors with their involvement and a bit from the public nature of the project.}

  \item What pain points did you experience during this deliverable, and how did
  you resolve them?
  \par{What caused some turbulence while working on this deliverable was not related to the deliverable at all, it was how we delegated work throughout our team. 
  We used a first-come-first-serve approach when it came to picking up issues related to our deliverable, which had both its pros and cons. It allowed some 
  members of our team to pick what they wanted to do earlier on and often times they would do it early. This allowed some members to complete their 
  work earlier on in the project timeline which is definitely a pro. On the flip side, members that start their work later on don't have as much flexibility in 
  their choice of work issues. This resulted in an imbalance of workload between group members where some would be left with multiple issues with more weight in 
  work when compared to previously completed issues, and at times people who did their share of work needed to complete extra for the sake of time. Our resolution plan 
  moving forward is to delegate tasks in a different manner, which will be dividing sections so everyone has equal weight in work before anyone starts working 
  on a deliverable.
  }

  \item How many of your requirements were inspired by speaking to your
  client(s) or their proxies (e.g. your peers, stakeholders, potential users)?
  \par{Its hard to keep track of how many requirements exactly were inspired by our conversations with our clients and their proxies, but I'll provide the 
  general requirement section names which were mostly or fully completely based on our discussions. Our mandated constraints were heavily based on context 
  provided to us about the current and future system implementation. The functional requirements we created were directly inspired by what features our clients wanted 
  to see us deliver with the project. The requirement sections related to the usability and user experience of the software, which were look and feel 
  and usability and humanity requirements, were inspired by our clients goals of creating easy-to-use and globally accessible software. Lastly, since our clients are 
  actively involved with the current system in place, we were able to clearly interpret their requirements in the operational and environmental domain, 
  as well as security and compliance requirements.}

  \item Which of the courses you have taken, or are currently taking, will help
  your team to be successful with your capstone project.
  \par{Some elements from the project-based courses, most notably 1P13 and 2PX3, will definitely help with the pacing and iterative nature in terms of development and 
  documentation of our project. Also from those project based courses, we can use principles we have learned about working in a group and doing presentations which will 
  be more prevalent further in the project timeline. On the topic of workflows and development cycles, I think we will need to adopt one or two development cycles like 
  Agile Methodology coupled with Test Driven Development for example. We have each had much practice with some kind of development cycle from our Software Design I and III 
  courses where we focused on building larger-scale software during a time period, and we will definitely need to apply what we have learned from those for our documentation 
  and development. Furthermore, one of our main focus points for this project is to create a user friendly and accessible interface for users of our software. We are all 
  currently in a Human Computer Interfaces class where we will need to apply what we learn about user elicitation, design, and experience. Lastly, our project will heavily rely 
  on data that will be visualized and possible pre-processed by users. This will be a perfect opportunity to apply our learnings from our Database Design course and our Concurrent 
  System Design course, both of which go hand-in-hand when designing efficient, scalable, and safe databases.}
\end{enumerate}

\section{Jeremy Orr Reflection:}

\begin{enumerate}
 \item What went well while writing this deliverable?

    \par{I think our team worked effectively to complete a large and detailed document in an organized manner. We communicated clearly and consistently throughout the process, which helped us stay aligned with our goals. Personally, I was able to complete my assigned sections early and efficiently. As a group, we also did a good job discussing challenges openly and collaborating to resolve them.}

    \par{One specific event that went particularly well was our collaborative editing session before submission. During that session, we reviewed the entire document together, finalized remaining sections, and made meaningful improvements. This not only strengthened the overall quality and consistency of the deliverable but also ensured that everyone’s contributions were well integrated. Overall, our teamwork and communication during that session demonstrated how effectively we could work together under a deadline.}

  \item What pain points did you experience during this deliverable, and how did you resolve them?

    \par{One of the main challenges we faced was dividing the work fairly among team members. Initially, we used a first-come, first-served approach, which led to an uneven distribution of tasks. Additionally, I accidentally started working on a section that a team member had already begun, which caused some inefficiency and duplicated effort.}

    \par{To resolve these issues, we discussed a more structured approach for future work, agreeing to evenly divide tasks before starting on a deliverable. For my specific issue, I committed to reviewing team assignments and communicating with teammates before beginning work on any section, ensuring that efforts are coordinated and that overlap is avoided. This experience highlighted the importance of proactive communication and clear task assignment in collaborative projects, which we will carry forward into future deliverables.}

  \item How many of your requirements were inspired by speaking to your client(s) or their proxies (e.g., your peers, stakeholders, potential users)?

    \par{The primary requirement that was influenced by discussions with our client was clarifying the specific scope and objectives of the project. We held multiple meetings with the clients, during which they shared numerous ideas for future extensions and potential features. These conversations were important to ensure that our team focused only on the deliverables required for the current project and did not extend into future scope items. Other than clarifying the project scope, there were few additional requirements that needed direct input from the client, as much of the remaining work was based on established technical objectives and user needs identified through our research.}

  \item Which of the courses you have taken, or are currently taking, will help
  your team to be successful with your capstone project.

    \par{I believe that several courses will help our team succeed in this capstone project. These include SFWRENG 2AA4: Introduction to Software Development, SFWRENG 3RA3: Software Requirements, SFWRENG 3A04: Software Testing, and the Software Architecture course. These courses have provided us with great knowledge in software development practices, requirement gathering, system design.}


\end{enumerate}

\end{document}
