\documentclass[12pt, titlepage]{article}

\usepackage{booktabs}
\usepackage{tabularx}
\usepackage{hyperref}
\hypersetup{
    colorlinks,
    citecolor=blue,
    filecolor=black,
    linkcolor=red,
    urlcolor=blue
}
\usepackage[round]{natbib}
\usepackage{longtable}

%% Comments

\usepackage{color}

\newif\ifcomments\commentstrue %displays comments
%\newif\ifcomments\commentsfalse %so that comments do not display

\ifcomments
\newcommand{\authornote}[3]{\textcolor{#1}{[#3 ---#2]}}
\newcommand{\todo}[1]{\textcolor{red}{[TODO: #1]}}
\else
\newcommand{\authornote}[3]{}
\newcommand{\todo}[1]{}
\fi

\newcommand{\wss}[1]{\authornote{magenta}{SS}{#1}} 
\newcommand{\plt}[1]{\authornote{cyan}{TPLT}{#1}} %For explanation of the template
\newcommand{\an}[1]{\authornote{cyan}{Author}{#1}}

%% Common Parts

\newcommand{\progname}{Software Engineering} % PUT YOUR PROGRAM NAME HERE
\newcommand{\authname}{Team \#18, Gouda Engineers 
\\ Aidan Goodyer
\\ Jeremy Orr
\\ Leo Vugert
\\ Nathan Perry
\\ Tim Pokanai} % AUTHOR NAMES                  

\usepackage{hyperref}
    \hypersetup{colorlinks=true, linkcolor=blue, citecolor=blue, filecolor=blue,
                urlcolor=blue, unicode=false}
    \urlstyle{same}
                                


\begin{document}

\title{System Verification and Validation Plan for \progname{}} 
\author{\authname}
\date{\today}
	
\maketitle

\pagenumbering{roman}

\section*{Revision History}

\begin{tabularx}{\textwidth}{p{3cm}p{2cm}X}
\toprule {\bf Date} & {\bf Version} & {\bf Notes}\\
\midrule
Date 1 & 1.0 & Notes\\
Date 2 & 1.1 & Notes\\
\bottomrule
\end{tabularx}

~\\
\wss{The intention of the VnV plan is to increase confidence in the software.
However, this does not mean listing every verification and validation technique
that has ever been devised.  The VnV plan should also be a \textbf{feasible}
plan. Execution of the plan should be possible with the time and team available.
If the full plan cannot be completed during the time available, it can either be
modified to ``fake it'', or a better solution is to add a section describing
what work has been completed and what work is still planned for the future.}

\wss{The VnV plan is typically started after the requirements stage, but before
the design stage.  This means that the sections related to unit testing cannot
initially be completed.  The sections will be filled in after the design stage
is complete.  the final version of the VnV plan should have all sections filled
in.}

\newpage

\tableofcontents

\listoftables
\wss{Remove this section if it isn't needed}

\listoffigures
\wss{Remove this section if it isn't needed}

\newpage

\section{Symbols, Abbreviations, and Acronyms}

\renewcommand{\arraystretch}{1.2}
\begin{tabular}{l l} 
  \toprule		
  \textbf{symbol} & \textbf{description}\\
  \midrule 
  T & Test\\
  \bottomrule
\end{tabular}\\

\wss{symbols, abbreviations, or acronyms --- you can simply reference the SRS
  \citep{SRS} tables, if appropriate}

\wss{Remove this section if it isn't needed}

\newpage

\pagenumbering{arabic}

This document ... \wss{provide an introductory blurb and roadmap of the
  Verification and Validation plan}

\section{General Information}

\subsection{Summary}

\wss{Say what software is being tested.  Give its name and a brief overview of
  its general functions.}

\subsection{Objectives}

\wss{State what is intended to be accomplished.  The objective will be around
  the qualities that are most important for your project.  You might have
  something like: ``build confidence in the software correctness,''
  ``demonstrate adequate usability.'' etc.  You won't list all of the qualities,
  just those that are most important.}

\wss{You should also list the objectives that are out of scope.  You don't have 
the resources to do everything, so what will you be leaving out.  For instance, 
if you are not going to verify the quality of usability, state this.  It is also 
worthwhile to justify why the objectives are left out.}

\wss{The objectives are important because they highlight that you are aware of 
limitations in your resources for verification and validation.  You can't do everything, 
so what are you going to prioritize?  As an example, if your system depends on an 
external library, you can explicitly state that you will assume that external library 
has already been verified by its implementation team.}

\subsection{Challenge Level and Extras}

\wss{State the challenge level (advanced, general, basic) for your project.
Your challenge level should exactly match what is included in your problem
statement.  This should be the challenge level agreed on between you and the
course instructor.  You can use a pull request to update your challenge level
(in TeamComposition.csv or Repos.csv) if your plan changes as a result of the
VnV planning exercise.}

\wss{Summarize the extras (if any) that were tackled by this project.  Extras
can include usability testing, code walkthroughs, user documentation, formal
proof, GenderMag personas, Design Thinking, etc.  Extras should have already
been approved by the course instructor as included in your problem statement.
You can use a pull request to update your extras (in TeamComposition.csv or
Repos.csv) if your plan changes as a result of the VnV planning exercise.}

\subsection{Relevant Documentation}

\wss{Reference relevant documentation.  This will definitely include your SRS
  and your other project documents (design documents, like MG, MIS, etc).  You
  can include these even before they are written, since by the time the project
  is done, they will be written.  You can create BibTeX entries for your
  documents and within those entries include a hyperlink to the documents.}

\citet{SRS}

\wss{Don't just list the other documents.  You should explain why they are relevant and 
how they relate to your VnV efforts.}

\section{Plan}

\par{The following section will outline a verification and validation plan to follow throughout the project timeline. 
Since our development of the project is executed in multiple sequential steps, this section will outline the verification and validation 
plans, as well as the tools and methods used for each milestone component leading up to our final software solution. This section will also 
list the entities that will participate in the verification and validation steps. First we will outline who will be involved in the 
verification and validation team. Then we will discuss the verification plans of the SRS, design, VnV Plan, implementation, and 
the automatic tools we use for testing and verification. Lastly, we will discuss how we will perform the validation of our software.}

\wss{Introduce this section.  You can provide a roadmap of the sections to
  come.}

\subsection{Verification and Validation Team}

\wss{Your teammates.  Maybe your supervisor.
  You should do more than list names.  You should say what each person's role is
  for the project's verification.  A table is a good way to summarize this information.}

\begin{itemize}
  \item{\textbf{Aidan Goodyer},\textbf{Jeremy Orr},\textbf{Nathan Perry},\textbf{Timothy Pokanai},\textbf{Leo Vugert}:}
   \par{The five members of the capstone team will split the testing evenly among themselves. All members will be equally expected to
  contribute to the synthesis of both unit and system tests as well as setting up automated tests down the line. Since the author of the System Testing plan was \textbf{Jeremy Orr}
  and the author of the Unit Testing plan will be \textbf{Nathan Perry}, they will be considered the head of system testing and unit testing respectively. Although
  this does not require them to necessarily produce more tests, they will be expected to be prepared to provide guidance to other group members if they need it.}
  \item{\textbf{Dr. Henry Szechtman}, \textbf{Dr. Anna Dvorkin-Gheva}:} 
  \par{ The two supervisors of this project will be involved in the testing process as if they were early
  access users. Given that they represent a demographic very similar to what our actual users will look like (Albeit with a much better understanding of the data),
  their feedback will be extremely useful in understanding how well our system will be recieved by users. They will perform both User Acceptance Testing and
  and Usability testing on our system; trying out the various functionalities and commenting on if they think it is satisfactory and understandable for the
  general user base.}

\end{itemize}

\subsection{SRS Verification}

\wss{List any approaches you intend to use for SRS verification.  This may
  include ad hoc feedback from reviewers, like your classmates (like your
  primary reviewer), or you may plan for something more rigorous/systematic.}

\wss{If you have a supervisor for the project, you shouldn't just say they will
read over the SRS.  You should explain your structured approach to the review.
Will you have a meeting?  What will you present?  What questions will you ask?
Will you give them instructions for a task-based inspection?  Will you use your
issue tracker?}

\wss{Maybe create an SRS checklist?}

\subsubsection{SRS Verification Approaches}

\begin{itemize}

  \item{\textbf{Peer Reviewers: } Our first approach for SRS verification is reviewing the peer review issues created by our classmates. This
  is not the most rigorous or systematic but allows us to evaluate feedback on our SRS and based on these comments, determine if the SRS is consistent,
  correct, feasible and complete or if changes are needed.}
  \item{\textbf{ Unstructured Supervisor SRS Review: } Our second approach for SRS verification is to provide our SRS document to our supervisor, Dr. Henry Szechtman and
  allow him to make comments on it. This approach is also quite unstructured but is still useful in our verification process. By giving our supervisor
  the time to look over the SRS, it will elicit any large gaps between our understanding of his goals for the system and our interpretation of these goals
  during our meetings to elicit said requirements. In other words, this will show us any glaring issues in our requirements from the perspective of our
  supervisor.}
  \item{\textbf{Structured Supervisor SRS Review: } Our third approach for SRS verification also involves our supervisor but in a much more structured manner.
  Our approach to this review would be a meeting in which we present to him the key aspects of the SRS and ask him if he sees any issues with these key sections.
  The main sections (note these do not necessarily refer to specific sections in the SRS template itself) to be covered, in descending order of priority are: Functional Requirements and Use Cases, Project Constraints and Scope, Non-Functional Requirements, Project
  Drivers and Project Issues. During this review we would ask him things like: Is this section complete? Is there anything missing? Does the way we laid out
  these requirements seem correct to you? Do these requirements seem feasible to you? Is this in accordance with your conception of what the system will look like?}
  \item{\textbf{Structured Team SRS Review: } Our final approach for SRS verification will be a review in a much similar format to the structured supervisor
  review but only with the five members of our team. We would go over the same key aspects of the SRS and ask mostly the same questions regarding these sections
  as we would in our supervisor review. Doing an internal review like this will be helpful as although the SRS was generated collectively, the actual writing
  of the requirements was done in a very individual and compartmentalized way and this gives each member an opportunity to closely examine the key parts of our
  requirements and provide and criticisms, issues or additions that they feel are necessary to each section.}
\end{itemize}

\subsubsection{SRS Verification Checklist}

\par{Below is a general checklist of questions that our SRS should satisfy (i.e. the answer should be yes):}

\begin{itemize}

  \item{Are all of the functional requirements necessary, feasible, complete and provide unique value to the system?}
  \item{Do the functional requirements cover every expected function of the system?}
  \item{Are all of the use cases necessary, feasible and complete?}
  \item{Do the existing use cases cover every expected function of the system?}
  \item{Does the SRS cover every relevant project constraint such that there is no unaccounted for aspect of the environment that
  may prevent us from implementing some of the system's functionality?}
  \item{Are all project constraints accurately described such that it is clear what exactly is constrained and it is understandable how
  this may affect the system?}
  \item{Is the scope properly defined such that all project components have been accounted for in the scope?}
  \item{Have all relevant out of scope components been listed and rationale provided for why they are out of scope?}
  \item{Are all non-functional requirements necessary, feasible, complete and provide unique value to the system?}
  \item{For non-functional requirements that are not completely necessary, is their value large enough to justify their existence?}
  \item{Have all obligatory non-functional requirements been accounted for, specifically those relating to legality, security or compliance that
        are imposed by entities outside of the system?}

\end{itemize}

\subsection{Design Verification}

\wss{Plans for design verification}

\wss{The review will include reviews by your classmates}

\wss{Create a checklists?}

\subsection{Verification and Validation Plan Verification}

\wss{The verification and validation plan is an artifact that should also be
verified.  Techniques for this include review and mutation testing.}

\wss{The review will include reviews by your classmates}

\wss{Create a checklists?}

\subsection{Implementation Verification}

\wss{You should at least point to the tests listed in this document and the unit
  testing plan.}

\wss{In this section you would also give any details of any plans for static
  verification of the implementation.  Potential techniques include code
  walkthroughs, code inspection, static analyzers, etc.}

\wss{The final class presentation in CAS 741 could be used as a code
walkthrough.  There is also a possibility of using the final presentation (in
CAS741) for a partial usability survey.}

\subsection{Automated Testing and Verification Tools}

\wss{If you have already done this in the development plan, you can point to
that document.}

The following tools are to be used during the verification and validation process to facilitate automated testing and verification. Broadly, these tools are separated into frontend and backend categories:

\textbf{Frontend:}
\begin{itemize}
  \item Jest: Facilitates frontend unit testing and coverage 
  \item ESLint: A JavaScript/TypeScript linter with support for JSX
  \item Puppeteer: End to end testing and automated UI testing.
\end{itemize}

\textbf{Backend:}
\begin{itemize}
  \item flake8: Style guide enforcement for Python backend
  \item PyTest: Unit testing and integration testing. 
\end{itemize}

\textbf{Continuous Integration:}

We will use GitHub actions for CI purposes, which will be configured to run the complete test suite for the frontend and backend upon commit, and serve as a quality gate prior to PR merges. Jobs will also be configured to run static analysis and linting checks, as well as generate coverage reports. 


\subsection{Software Validation}

\wss{If there is any external data that can be used for validation, you should
  point to it here.  If there are no plans for validation, you should state that
  here.}

\wss{You might want to use review sessions with the stakeholder to check that
the requirements document captures the right requirements.  Maybe task based
inspection?}

\wss{For those capstone teams with an external supervisor, the Rev 0 demo should 
be used as an opportunity to validate the requirements.  You should plan on 
demonstrating your project to your supervisor shortly after the scheduled Rev 0 demo.  
The feedback from your supervisor will be very useful for improving your project.}

\wss{For teams without an external supervisor, user testing can serve the same purpose 
as a Rev 0 demo for the supervisor.}

\wss{This section might reference back to the SRS verification section.}

\section{System Tests}

This section defines the tests used to verify the system’s compliance with the
Functional Requirements specified in Section~9.1 of the SRS. Each test ensures
that the implemented features meet the expected behavior and fit criteria
described in the Software Requirements Specification. 

The subsections are organized according to the main areas of functionality of
the platform:
\begin{itemize}
    \item Data Filtering and Browsing (FUNC.R.1--R.3)
    \item Behavioral Analysis and Visualization (FUNC.R.4--R.5)
    \item Data Downloading and Accessibility (FUNC.R.6--R.7)
\end{itemize}

\subsection{Tests for Functional Requirements}

This section provides detailed tests for all functional requirements. Each test
includes a test ID, control type, initial conditions, inputs, expected outputs,
test case derivation, and execution description. References to the SRS are
included for traceability. 

\subsubsection{Data Filtering and Browsing}

These tests verify that users can filter, browse, and query data based on
predefined labels, natural language queries, and preset datasets. They validate
the backend (FastAPI + PostgreSQL) integration with the frontend filtering and
NLP modules.

\paragraph{Test 1: Verify Filtering by Predefined Labels}
\begin{itemize}
    \item \textbf{Related Requirement:} FUNC.R.1
    \item \textbf{Test ID:} FR1-Filter-Labels
    \item \textbf{Control:} Automatic (API + UI test)
    \item \textbf{Initial State:} The database contains labeled behavioral trials with fields such as \texttt{behavior\_label}, \texttt{trial\_id}, and \texttt{session\_id}.
    \item \textbf{Input:} User selects a filter, e.g., \texttt{behavior\_label = 'checking'}.
    \item \textbf{Output:} Returned dataset contains only trials where \texttt{behavior\_label == 'checking'}; 100\% match accuracy.
    \item \textbf{Test Case Derivation:} Since the fit criterion requires perfect accuracy between filter condition and output, the expected output is all rows meeting the filter condition.
    \item \textbf{How test will be performed:} Use an automated test script (e.g., Postman or PyTest) to send GET requests to \texttt{/api/filter?behavior\_label=checking}. Compare returned data with a query directly run on the database to verify no mismatched rows.
\end{itemize}

\paragraph{Test 2: Verify NLP-Based Query Returns Correct Dataset}
\begin{itemize}
    \item \textbf{Related Requirement:} FUNC.R.2
    \item \textbf{Test ID:} FR2-NLP-Query
    \item \textbf{Control:} Manual + Automatic hybrid (human semantic check)
    \item \textbf{Initial State:} NLP model and API are running. Database populated with varied behavioral patterns.
    \item \textbf{Input:} User enters: ``Show me trials with strong checking behavior after 5 injections.''
    \item \textbf{Output:} One dataset is returned that matches this semantic filter.
    \item \textbf{Test Case Derivation:} Based on the SRS fit criterion, one dataset should always be returned for a valid query, containing the subset matching the interpreted condition.
    \item \textbf{How test will be performed:} Send query via UI or API. Validate that the system outputs one dataset (non-empty). A human tester verifies semantic correctness (dataset contextually fits query meaning).
\end{itemize}

\paragraph{Test 3: Verify Predefined Data Categories Available for Browsing}
\begin{itemize}
    \item \textbf{Related Requirement:} FUNC.R.3
    \item \textbf{Test ID:} FR3-Predefined-Sets
    \item \textbf{Control:} Manual (UI check)
    \item \textbf{Initial State:} System deployed with at least 3 predefined sets (e.g., ``Exploration Trials'', ``Compulsive Behavior'', ``Control Trials'').
    \item \textbf{Input:} User opens the ``Browse'' interface.
    \item \textbf{Output:} At least 3 predefined datasets appear for selection.
    \item \textbf{Test Case Derivation:} Requirement specifies a minimum of 3 sets available for browsing.
    \item \textbf{How test will be performed:} Manually navigate to the Browse page and confirm at least 3 datasets are displayed and accessible.
\end{itemize}

\subsubsection{Behavioral Analysis and Visualization}

These tests ensure that behavioral categorizations and visual representations are correctly generated for user-selected data subsets.

\paragraph{Test 4: Verify Each Trial Has Behavioral Categorization}
\begin{itemize}
    \item \textbf{Related Requirement:} FUNC.R.4
    \item \textbf{Test ID:} FR4-Behavior-Metrics
    \item \textbf{Control:} Automatic (backend validation)
    \item \textbf{Initial State:} Database populated with trial records.
    \item \textbf{Input:} Query any trial dataset.
    \item \textbf{Output:} Each trial entry includes at least one behavioral category field (e.g., ``checking'', ``homebase'', etc.).
    \item \textbf{Test Case Derivation:} Requirement demands every trial have a behavior category, so absence of any \texttt{NULL} category field indicates success.
    \item \textbf{How test will be performed:} Run automated query validation across dataset ensuring no missing categorization values.
\end{itemize}

\paragraph{Test 5: Verify Visuals Are Generated for Each Result Set}
\begin{itemize}
    \item \textbf{Related Requirement:} FUNC.R.5
    \item \textbf{Test ID:} FR5-Visualization
    \item \textbf{Control:} Automatic (UI rendering check)
    \item \textbf{Initial State:} User has filtered dataset (non-empty result).
    \item \textbf{Input:} User clicks ``Generate Visualization''.
    \item \textbf{Output:} Visualization component renders trajectory or metric chart corresponding to the dataset.
    \item \textbf{Test Case Derivation:} The fit criterion specifies at least one visual per result set; therefore, successful rendering of a chart or trajectory fulfills this requirement.
    \item \textbf{How test will be performed:} Use UI test automation (e.g., Selenium) to trigger visualization and check that the chart container loads successfully (non-empty DOM element).
\end{itemize}

\subsubsection{Data Downloading and Accessibility}

These tests confirm that datasets and visuals can be downloaded correctly and that the platform maintains global accessibility.

\paragraph{Test 6: Verify Data Download and Integrity}
\begin{itemize}
    \item \textbf{Related Requirement:} FUNC.R.6
    \item \textbf{Test ID:} FR6-Download
    \item \textbf{Control:} Automatic
    \item \textbf{Initial State:} Dataset generated and displayed.
    \item \textbf{Input:} User clicks ``Download CSV'' or ``Download Visualization''.
    \item \textbf{Output:} File downloaded successfully; first 100 rows match database query results.
    \item \textbf{Test Case Derivation:} Fit criterion requires verification against the first 100 rows; thus, equality of the first 100 entries validates correctness.
    \item \textbf{How test will be performed:} Automated script downloads dataset and compares first 100 entries with direct database query output.
\end{itemize}

\paragraph{Test 7: Verify Global Accessibility and Performance}
\begin{itemize}
    \item \textbf{Related Requirement:} FUNC.R.7
    \item \textbf{Test ID:} FR7-Accessibility
    \item \textbf{Control:} Automatic (load testing + global simulation)
    \item \textbf{Initial State:} Deployed web system with CDN enabled.
    \item \textbf{Input:} Users (or simulated clients) from 3+ regions (e.g., North America, Europe, Asia) access the site.
    \item \textbf{Output:} Average page load time $<$ 4 seconds across all regions; no functional degradation.
    \item \textbf{Test Case Derivation:} Requirement defines measurable metric (4 seconds); success if timing threshold not exceeded.
    \item \textbf{How test will be performed:} Use tools such as Lighthouse, GTmetrix, or JMeter with geo-distributed endpoints to measure average response time and record metrics.
\end{itemize}


\subsection{Tests for Nonfunctional Requirements}

This section defines the test procedures for verifying the nonfunctional
requirements as defined in Sections~10--17 of the SRS. The tests include
evaluations of appearance, usability, performance, robustness, scalability,
maintainability, and compliance. 

Since nonfunctional requirements often describe qualities rather than discrete
functions, many tests in this section are qualitative or metric-based
evaluations. Several involve user studies, performance measurement, and static
reviews instead of strict pass/fail results.

\subsubsection{Look and Feel Requirements}

These tests ensure that the interface design aligns with the intended audience
(non-technical researchers) and provides a simple, familiar, and intuitive
user experience.

\paragraph{Test 1: Interface Resemblance to Webstore/Boutique}
\begin{itemize}
    \item \textbf{Related Requirement:} APP.R.1
    \item \textbf{Test ID:} NFR1-UI-Appearance
    \item \textbf{Type:} Static, Manual (Usability Review)
    \item \textbf{Initial State:} Functional web interface deployed.
    \item \textbf{Input/Condition:} UI inspected by 3 non-technical evaluators.
    \item \textbf{Output/Result:} 80\% of reviewers confirm the interface resembles a webstore-style interface with clear query “packages.”
    \item \textbf{How test will be performed:} Conduct a design walkthrough where users compare the layout to an e-commerce platform. Collect ratings via short usability survey (Appendix~A).
\end{itemize}

\paragraph{Test 2: Simplicity of Interface Presentation}
\begin{itemize}
    \item \textbf{Related Requirement:} APP.R.2
    \item \textbf{Test ID:} NFR2-UI-Simplicity
    \item \textbf{Type:} Static, Manual (Heuristic Evaluation)
    \item \textbf{Initial State:} All major features visible on homepage.
    \item \textbf{Input/Condition:} Evaluators perform inspection for UI density and feature visibility.
    \item \textbf{Output/Result:} Interface should not exceed 6 visible interactive elements on the main view; evaluators report no intimidation or confusion.
    \item \textbf{How test will be performed:} Use heuristic evaluation checklist for cognitive load and visual complexity; summarize reviewer feedback quantitatively.
\end{itemize}

\paragraph{Test 3: Familiar Filtering Presentation}
\begin{itemize}
    \item \textbf{Related Requirement:} APP.R.3
    \item \textbf{Test ID:} NFR3-UI-Familiarity
    \item \textbf{Type:} Static, Manual
    \item \textbf{Initial State:} Filtering and search panel implemented.
    \item \textbf{Input/Condition:} Users asked to find a data subset using filter controls.
    \item \textbf{Output/Result:} 90\% of participants find the filters intuitive and similar to an online storefront.
    \item \textbf{How test will be performed:} Conduct usability test with 5 participants. Record task completion time and perceived ease of use (Likert scale).
\end{itemize}

\subsubsection{Usability and Humanity Requirements}

These tests measure user learning time, ease of use, and understandability.

\paragraph{Test 4: Ease of Use (Floor and Ceiling)}
\begin{itemize}
    \item \textbf{Related Requirements:} EOU.R.1, EOU.R.2
    \item \textbf{Test ID:} NFR4-EaseOfUse
    \item \textbf{Type:} Dynamic, Manual (User Study)
    \item \textbf{Initial State:} Functional system with packaged queries and NLP interface.
    \item \textbf{Input/Condition:} New users complete a task set (select prepackaged query, perform NLP search, generate visualization).
    \item \textbf{Output/Result:} 100\% task completion; average total completion time $<$ 5 minutes for basic operations.
    \item \textbf{How test will be performed:} Conduct usability test with 5 non-technical participants. Record success rate and completion time per task.
\end{itemize}

\paragraph{Test 5: Learning Curve Assessment}
\begin{itemize}
    \item \textbf{Related Requirement:} LEA.R.1
    \item \textbf{Test ID:} NFR5-LearningCurve
    \item \textbf{Type:} Dynamic, Manual
    \item \textbf{Initial State:} Fully functional platform; user documentation accessible.
    \item \textbf{Input/Condition:} Participants with no prior system experience attempt to complete key workflows.
    \item \textbf{Output/Result:} Average learning curve $<$ 30 minutes; all users can independently perform search and visualization tasks.
    \item \textbf{How test will be performed:} Measure total time for users to reach proficiency (defined as completing 3 tasks unassisted).
\end{itemize}

\paragraph{Test 6: Understandability and Terminology Audit}
\begin{itemize}
    \item \textbf{Related Requirements:} UAP.R.1, UAP.R.2
    \item \textbf{Test ID:} NFR6-Understandability
    \item \textbf{Type:} Static, Manual (Content Review)
    \item \textbf{Initial State:} Interface text finalized.
    \item \textbf{Input/Condition:} Non-technical users read through all on-screen text.
    \item \textbf{Output/Result:} At least 90\% of all interface text rated “understandable” by users.
    \item \textbf{How test will be performed:} Conduct terminology walkthrough; classify all words/phrases as technical or intuitive. Replace technical terms as necessary.
\end{itemize}

\paragraph{Test 7: Accessibility Audit}
\begin{itemize}
    \item \textbf{Related Requirement:} ACC.R.1
    \item \textbf{Test ID:} NFR7-Accessibility
    \item \textbf{Type:} Static + Automated
    \item \textbf{Initial State:} Frontend deployed.
    \item \textbf{Input/Condition:} Run Lighthouse, Axe, or WAVE accessibility audit tools.
    \item \textbf{Output/Result:} Accessibility score $\geq$ 90/100; ARIA labels and alt-text coverage 100\%.
    \item \textbf{How test will be performed:} Automated scan using accessibility tools; manual confirmation of ARIA compliance and alt-text for media.
\end{itemize}

\subsubsection{Performance and Reliability Requirements}

These tests verify latency, throughput, fault-tolerance, and data integrity.

\paragraph{Test 8: Query Speed and Latency Benchmark}
\begin{itemize}
    \item \textbf{Related Requirements:} SAL.R.1, SAL.R.2
    \item \textbf{Test ID:} NFR8-Performance
    \item \textbf{Type:} Dynamic, Automatic
    \item \textbf{Initial State:} Server and database running under normal load.
    \item \textbf{Input/Condition:} Execute 100 queries of varying sizes up to 5,000 records.
    \item \textbf{Output/Result:} Average query time $<$ 2 seconds; request latency $<$ 100 ms.
    \item \textbf{How test will be performed:} Automated benchmark using JMeter or Locust. Generate graph of response time vs. query size.
\end{itemize}

\paragraph{Test 9: Fault-Tolerance and Logging Verification}
\begin{itemize}
    \item \textbf{Related Requirements:} ROFT.R.1, ROFT.R.2
    \item \textbf{Test ID:} NFR9-Robustness
    \item \textbf{Type:} Dynamic, Automatic
    \item \textbf{Initial State:} Backend running.
    \item \textbf{Input/Condition:} Submit invalid input payloads and simulate network interruptions.
    \item \textbf{Output/Result:} Errors logged without system crash; user receives appropriate feedback message.
    \item \textbf{How test will be performed:} Automated injection of malformed JSON; verify log entries and system stability.
\end{itemize}

\paragraph{Test 10: Capacity and Scalability Evaluation}
\begin{itemize}
    \item \textbf{Related Requirements:} CAP.R.1, CAP.R.2, SOE.R.2
    \item \textbf{Test ID:} NFR10-Capacity
    \item \textbf{Type:} Dynamic, Automatic
    \item \textbf{Initial State:} System deployed in cloud environment.
    \item \textbf{Input/Condition:} Simulate 250 concurrent users performing queries.
    \item \textbf{Output/Result:} Throughput $\geq$ 5,000 transactions/hour; performance efficiency $\geq$ 80\%.
    \item \textbf{How test will be performed:} Use load testing tools (Locust, K6) to simulate concurrent access and measure throughput metrics.
\end{itemize}

\subsubsection{Maintainability and Support Requirements}

\paragraph{Test 11: Maintainability Code Review}
\begin{itemize}
    \item \textbf{Related Requirements:} MAI.R.1, MAI.R.2
    \item \textbf{Test ID:} NFR11-Maintainability
    \item \textbf{Type:} Static (Code Walkthrough)
    \item \textbf{Initial State:} Source code and documentation complete.
    \item \textbf{Input/Condition:} Independent developer reviews repository structure and documentation.
    \item \textbf{Output/Result:} Reviewer able to build and extend system within 1 hour using provided documentation.
    \item \textbf{How test will be performed:} Conduct peer code walkthrough; independent participant attempts new dataset integration per SRS instructions.
\end{itemize}

\paragraph{Test 12: User Manual Review}
\begin{itemize}
    \item \textbf{Related Requirement:} SUP.R.1
    \item \textbf{Test ID:} NFR12-Documentation
    \item \textbf{Type:} Static, Manual
    \item \textbf{Initial State:} User manual drafted.
    \item \textbf{Input/Condition:} Review manual completeness and clarity.
    \item \textbf{Output/Result:} 90\% of user study participants rate the manual as “clear” or better.
    \item \textbf{How test will be performed:} Conduct documentation walkthrough with usability test participants; collect feedback ratings.
\end{itemize}

\subsubsection{Security and Compliance Requirements}

\paragraph{Test 13: Data Integrity and Abuse Prevention}
\begin{itemize}
    \item \textbf{Related Requirement:} INT.R.1, IMM.R.1, IMM.R.2
    \item \textbf{Test ID:} NFR13-Security
    \item \textbf{Type:} Dynamic, Automatic
    \item \textbf{Initial State:} Backend deployed behind HTTPS.
    \item \textbf{Input/Condition:} Attempt SQL injection, code injection, and rate-limit violations.
    \item \textbf{Output/Result:} All attacks blocked; valid error message returned; rate-limiting enforced at 100 requests/minute.
    \item \textbf{How test will be performed:} Perform penetration testing using OWASP ZAP; verify all malicious attempts logged and mitigated.
\end{itemize}

\paragraph{Test 14: Standards and Licensing Compliance}
\begin{itemize}
    \item \textbf{Related Requirements:} LEG.R.1, STA.R.1, STA.R.2
    \item \textbf{Test ID:} NFR14-Compliance
    \item \textbf{Type:} Static, Manual Review
    \item \textbf{Initial State:} Code repository finalized.
    \item \textbf{Input/Condition:} Review dependency list and license headers.
    \item \textbf{Output/Result:} All third-party libraries under compatible licenses; repository includes MIT license and citation requirements.
    \item \textbf{How test will be performed:} Conduct static code inspection; verify license documentation for all dependencies.
\end{itemize}


\subsection{Traceability Between Test Cases and Requirements}

Table~\ref{tab:traceability} provides a traceability matrix showing the correspondence between each Software Requirements Specification (SRS) requirement and the test cases defined in Sections~4.1 and~4.2. Each requirement identifier (FUNC.R.\#, APP.R.\#, etc.) is mapped to one or more test case identifiers that verify its implementation or satisfaction.

\renewcommand{\arraystretch}{1.2}
\begin{longtable}{|l|p{10cm}|}
\caption{Traceability Between Test Cases and Requirements}
\label{tab:traceability} \\
\hline
\textbf{Requirement ID} & \textbf{Associated Test Case(s)} \\
\hline
\endfirsthead
\hline
\textbf{Requirement ID} & \textbf{Associated Test Case(s)} \\
\hline
\endhead
\hline \multicolumn{2}{r}{{Continued on next page}} \\
\endfoot
\hline
\endlastfoot

FUNC.R.1 & ST-FR1.1 Filter Accuracy Test \\
FUNC.R.2 & ST-FR2.1 NLP Query Response Test \\
FUNC.R.3 & ST-FR3.1 Predefined Dataset Browsing Test \\
FUNC.R.4 & ST-FR4.1 Behavioural Categorization and Metrics Test \\
FUNC.R.5 & ST-FR5.1 Visualization Generation Test \\
FUNC.R.6 & ST-FR6.1 Data Download and Export Verification Test \\
FUNC.R.7 & ST-FR7.1 Global Accessibility and Latency Test \\
APP.R.1 & ST-NFR1.1 Interface Familiarity Inspection \\
APP.R.2 & ST-NFR1.2 Visual Simplicity Inspection \\
APP.R.3 & ST-NFR1.3 Filter Interface Familiarity Test \\
EOU.R.1 & ST-NFR2.1 Ease of Use (Low Floor) Usability Study \\
EOU.R.2 & ST-NFR2.2 Ease of Use (Low Ceiling) Usability Study \\
LEA.R.1 & ST-NFR3.1 Learning Curve Evaluation Test \\
LEA.R.2 & ST-NFR3.2 Visualization Guidance Test \\
UAP.R.1 & ST-NFR4.1 Technical Abstraction Inspection \\
UAP.R.2 & ST-NFR4.2 Terminology Familiarity Review \\
ACC.R.1 & ST-NFR5.1 Accessibility Compliance Test \\
SAL.R.1 & ST-NFR6.1 Query Response Time Benchmark \\
SAL.R.2 & ST-NFR6.2 Network Latency Benchmark \\
POA.R.1 & ST-NFR7.1 Data Accuracy Cross-Check \\
POA.R.2 & ST-NFR7.2 Database Consistency Test \\
ROFT.R.1 & ST-NFR8.1 Input Error Logging Test \\
ROFT.R.2 & ST-NFR8.2 Backend Fault Tolerance Test \\
CAP.R.1 & ST-NFR9.1 Concurrent User Load Test \\
CAP.R.2 & ST-NFR9.2 Data Volume Stress Test \\
SOE.R.1 & ST-NFR10.1 Module Integration Test \\
SOE.R.2 & ST-NFR10.2 Performance Under Load Test \\
LON.R.1 & ST-NFR11.1 Reliability and Maintenance Longevity Test \\
LON.R.2 & ST-NFR11.2 Cross-OS Compatibility Test \\
EPE.R.1 & ST-NFR12.1 Desktop Environment Performance Test \\
EPE.R.2 & ST-NFR12.2 Multi-OS Execution Verification Test \\
EPE.R.3 & ST-NFR12.3 Environmental Condition Simulation \\
WE.R.1 & ST-NFR13.1 Browser Compatibility Test \\
IWAS.R.1 & ST-NFR14.1 API Communication Protocol Test \\
PROD.R.1 & ST-NFR15.1 Docker/Kubernetes Deployment Test \\
PROD.R.2 & ST-NFR15.2 Repository Update Verification Test \\
REL.R.1 & ST-NFR16.1 Versioning Convention Test \\
REL.R.2 & ST-NFR16.2 Release Consistency Verification Test \\
MAI.R.1 & ST-NFR17.1 Documentation Maintainability Inspection \\
MAI.R.2 & ST-NFR17.2 Dataset Integration Maintenance Test \\
SUP.R.1 & ST-NFR18.1 User Manual Verification Test \\
SUP.R.2 & ST-NFR18.2 Self-Support Functionality Test \\
ADA.R.1 & ST-NFR19.1 Platform Adaptability Test \\
INT.R.1 & ST-NFR20.1 Data Integrity Protection Test \\
IMM.R.1 & ST-NFR21.1 Security Attack Resistance Test \\
IMM.R.2 & ST-NFR21.2 Rate Limiting and HTTPS Enforcement Test \\
LEG.R.1 & ST-NFR22.1 License Compliance Review \\
STA.R.1 & ST-NFR23.1 FRDR Standards Compliance Test \\
STA.R.2 & ST-NFR23.2 Ethical Use of Data Verification Test \\

\end{longtable}



\section{Unit Test Description}

\wss{This section should not be filled in until after the MIS (detailed design
  document) has been completed.}

\wss{Reference your MIS (detailed design document) and explain your overall
philosophy for test case selection.}  

\wss{To save space and time, it may be an option to provide less detail in this section.  
For the unit tests you can potentially layout your testing strategy here.  That is, you 
can explain how tests will be selected for each module.  For instance, your test building 
approach could be test cases for each access program, including one test for normal behaviour 
and as many tests as needed for edge cases.  Rather than create the details of the input 
and output here, you could point to the unit testing code.  For this to work, you code 
needs to be well-documented, with meaningful names for all of the tests.}

\subsection{Unit Testing Scope}

\wss{What modules are outside of the scope.  If there are modules that are
  developed by someone else, then you would say here if you aren't planning on
  verifying them.  There may also be modules that are part of your software, but
  have a lower priority for verification than others.  If this is the case,
  explain your rationale for the ranking of module importance.}

\subsection{Tests for Functional Requirements}

\wss{Most of the verification will be through automated unit testing.  If
  appropriate specific modules can be verified by a non-testing based
  technique.  That can also be documented in this section.}

\subsubsection{Module 1}

\wss{Include a blurb here to explain why the subsections below cover the module.
  References to the MIS would be good.  You will want tests from a black box
  perspective and from a white box perspective.  Explain to the reader how the
  tests were selected.}

\begin{enumerate}

\item{test-id1\\}

Type: \wss{Functional, Dynamic, Manual, Automatic, Static etc. Most will
  be automatic}
					
Initial State: 
					
Input: 
					
Output: \wss{The expected result for the given inputs}

Test Case Derivation: \wss{Justify the expected value given in the Output field}

How test will be performed: 
					
\item{test-id2\\}

Type: \wss{Functional, Dynamic, Manual, Automatic, Static etc. Most will
  be automatic}
					
Initial State: 
					
Input: 
					
Output: \wss{The expected result for the given inputs}

Test Case Derivation: \wss{Justify the expected value given in the Output field}

How test will be performed: 

\item{...\\}
    
\end{enumerate}

\subsubsection{Module 2}

...

\subsection{Tests for Nonfunctional Requirements}

\wss{If there is a module that needs to be independently assessed for
  performance, those test cases can go here.  In some projects, planning for
  nonfunctional tests of units will not be that relevant.}

\wss{These tests may involve collecting performance data from previously
  mentioned functional tests.}

\subsubsection{Module ?}
		
\begin{enumerate}

\item{test-id1\\}

Type: \wss{Functional, Dynamic, Manual, Automatic, Static etc. Most will
  be automatic}
					
Initial State: 
					
Input/Condition: 
					
Output/Result: 
					
How test will be performed: 
					
\item{test-id2\\}

Type: Functional, Dynamic, Manual, Static etc.
					
Initial State: 
					
Input: 
					
Output: 
					
How test will be performed: 

\end{enumerate}

\subsubsection{Module ?}

...

\subsection{Traceability Between Test Cases and Modules}

\wss{Provide evidence that all of the modules have been considered.}
				
\bibliographystyle{plainnat}

\bibliography{../../refs/References}

\newpage

\section{Appendix}

This is where you can place additional information.

\subsection{Symbolic Parameters}

The definition of the test cases will call for SYMBOLIC\_CONSTANTS.
Their values are defined in this section for easy maintenance.

\subsection{Usability Survey Questions?}

\wss{This is a section that would be appropriate for some projects.}

\newpage{}
\section*{Appendix --- Reflection}

\wss{This section is not required for CAS 741}

The information in this section will be used to evaluate the team members on the
graduate attribute of Lifelong Learning.

The purpose of reflection questions is to give you a chance to assess your own
learning and that of your group as a whole, and to find ways to improve in the
future. Reflection is an important part of the learning process.  Reflection is
also an essential component of a successful software development process.  

Reflections are most interesting and useful when they're honest, even if the
stories they tell are imperfect. You will be marked based on your depth of
thought and analysis, and not based on the content of the reflections
themselves. Thus, for full marks we encourage you to answer openly and honestly
and to avoid simply writing ``what you think the evaluator wants to hear.''

Please answer the following questions.  Some questions can be answered on the
team level, but where appropriate, each team member should write their own
response:


\begin{enumerate}
  \item What went well while writing this deliverable? 
  \item What pain points did you experience during this deliverable, and how
    did you resolve them?
  \item What knowledge and skills will the team collectively need to acquire to
  successfully complete the verification and validation of your project?
  Examples of possible knowledge and skills include dynamic testing knowledge,
  static testing knowledge, specific tool usage, Valgrind etc.  You should look to
  identify at least one item for each team member.
  \item For each of the knowledge areas and skills identified in the previous
  question, what are at least two approaches to acquiring the knowledge or
  mastering the skill?  Of the identified approaches, which will each team
  member pursue, and why did they make this choice?
\end{enumerate}

\end{document}
